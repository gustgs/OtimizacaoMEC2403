\documentclass[10pt, a4paper]{article}
% \usepackage[english]{babel}
\usepackage[brazilian]{babel}
\usepackage[utf8]{inputenc}
% \usepackage[T1]{fontenc}
\usepackage{lipsum}

% matlab code
% \usepackage{matlab-prettifier}
%\usepackage[numbered,framed]{matlab-prettifier}
\usepackage{pythonhighlight}
\renewcommand{\lstlistingname}{Anexo} % Listing->Code
\let\ph\mlplaceholder % shorter macro
\definecolor{codegreen}{rgb}{0,0.6,0}
\definecolor{codegray}{rgb}{0.5,0.5,0.5}
\definecolor{codepurple}{rgb}{0.58,0,0.82}
\definecolor{backcolour}{rgb}{0.95,0.95,0.92}
\lstdefinestyle{myStyle}{
    language=Matlab,
    breaklines=true,
    frame=single,
    numbers=none,
    basicstyle=\ttfamily\footnotesize,
%     basicstyle=\footnotesize\ttfamily,
    keywordstyle=\bfseries\color{magenta},
    commentstyle=\color{codegreen},
    identifierstyle=\color{blue},
    backgroundcolor=\color{backcolour},
    stringstyle=\color{codepurple},
}
\usepackage{adjustbox}

% For subfigure use
\usepackage[font=small,labelfont=bf]{caption}
\usepackage{subcaption}

% Set page size and margins
% Replace `letterpaper' with`a4paper' for UK/EU standard size
\usepackage[a4paper,top=2cm,bottom=2cm,left=2cm,right=2cm,marginparwidth=2cm]{geometry}

% tabelas
\usepackage{array}
\usepackage{tabularx}
\usepackage{booktabs}

\usepackage{float}

% Useful packages
\usepackage{amsmath}

\usepackage{graphicx}
%\graphicspath{{figures/}} %Setting the graphicspath
\usepackage[colorlinks=true, allcolors=blue]{hyperref}
\usepackage{cleveref}
\newcommand{\crefrangeconjunction}{--}
\DeclareMathOperator{\sen}{sen}


\begin{document}

\def\TITLE{Trabalho 2}
\def\SUBTITLE{Opção 01}
\def\DISCIPLINE{MEC 2403 - Otimização, Algoritmos e Aplicações na Engenharia Mecânica}
\def\PROFESSOR{Ivan Menezes}
\def\AUTHOR{Gustavo Henrique Gomes dos Santos}
\def\CONTACT{gustavohgs@gmail.com}
\def\DATE{junho de 2023}

\title{\textbf{\TITLE} \\ \DISCIPLINE}
\author{\AUTHOR}
\date{\DATE}

\begin{titlepage}
      \begin{center}
          \vspace*{1cm}

          \Huge
          \textbf{\TITLE}

          \Large
          \textbf{\SUBTITLE}

          \vspace{0.5cm}
          \LARGE
          \DISCIPLINE

          \vspace{1.5cm}

          \textbf{\AUTHOR \\ {\tt \CONTACT}}

          \vfill
          Professor: \PROFESSOR

          \vspace{0.8cm}

          \includegraphics[width=0.2\textwidth]{../general/puc.jpg}

          \Large
          Departamento de Engenharia Mecânica\\
          PUC-RJ Pontifícia Universidade Católica do Rio de Janeiro\\
          \DATE

      \end{center}
  \end{titlepage}

\maketitle

\section{Introdução}

\subsection{Objetivos}

Esse trabalho tem como objetivo a implementação e teste dos seguintes métodos indiretos de otimização com restrição :

\begin{enumerate}
  \item Penalidade
  \item Barreira
\end{enumerate}

Para a etapa de sequência de otimização sem restrição, da qual esses métodos fazem uso, foram utilizados os métodos
implementados no trabalho 1. São eles:

\begin{enumerate}
  \item Univariante
  \item Powell
  \item Steepest Descent
  \item Fletcher-Reeves
  \item BFGS
  \item Newton-Raphson
\end{enumerate}

Estes métodos, por sua vez, fazem uso dos seguintes algoritmos de busca unidimensional também implementados em trabalhos anteriores :

\begin{enumerate}
  \item Passo constante
  \item Seção áurea
\end{enumerate}

\section{Implementação}

Foi utilizada a linguagem de programação Python para elaboração deste trabalho.
A implementação consiste de um arquivo com o código principal, um segundo arquivo com os algoritmos dos métodos de otimização com restrição,
um terceiro arquivo que encapsula os métodos e a convergência da otimização sem restrição
e um quarto arquivo com os algoritmos da busca unidimensional.

\subsection{Código Principal}

O código principal trata das definições do ponto inicial, função a ser minimizada 
e as restrições de igualdade e desigualdade.
A escolha de quais métodos OCR e OSR serão utilizados e a definição dos parâmetros numéricos também são feitos no código principal.
Além disso, o controle de passos da otimização OC e a verificação de convergência.

\vspace{5mm}
Os seguintes pacotes foram utilizados na implementação do código principal, já inclusos os arquivos .py com
as implementações dos métodos OSR, busca unidimensional e métodos OCR.

\begin{python}
  import numpy as np
  import matplotlib.pyplot as plt
  import osr_methods as osr
  import line_search_methods as lsm
  import ocr_methods as ocr  
\end{python}

\vspace{5mm}
Definição do ponto inicial, que irá variar em cada teste realizado com diferentes funções e método OCR.

\begin{python}
  x = np.array([3., 2.])
\end{python}

\vspace{5mm}
Escolha dos métodos de OCR e OSR.

\begin{python}
  # Metodos OCR
  # 1 - Penalidade
  # 2 - Barreira
  metodo_ocr = 1

  if (metodo_ocr == 1):
      n_met_ocr = "Penalidade"
  elif (metodo_ocr == 2):
      n_met_ocr = "Barreira"

  # Metodos OSR
  # 1 - Univariante
  # 2 - Powell
  # 3 - Stepest Descent
  # 4 - Newton-Raphson
  # 5 - Fletcher-Reeves
  # 6 - BFGS
  metodo_osr = 4

  if (metodo_osr == 1):
      n_met = 'Univariante'
  elif (metodo_osr == 2):
      n_met = 'Powell'
  elif (metodo_osr == 3):
      n_met = 'Steepest Descent'
  elif (metodo_osr == 4):
      n_met = 'Newton-Raphson'
  elif (metodo_osr == 5):
      n_met = 'Fletcher-Reeves'
  elif (metodo_osr == 6):
      n_met = 'BFGS'  
\end{python}

\vspace{5mm}
Controle numérico

\begin{python}
  # numero maximo de iteracoes na OSR
  maxiter = 1000

  # tolerancia para convergencia do gradiente na OSR
  tol_conv = 1E-6

  # tolerancia para a busca unidirecional na OSR
  tol_search = 1E-7

  # delta alpha do passo constante na OSR
  line_step = 1E-2

  #epsilon da maquina
  eps = 1E-10

  #parametros ocr
  if metodo_ocr == 1:
      #penalidade
      r = 1
      beta = 10
  elif metodo_ocr == 2:
      #barreira
      r = 10
      beta = 0.1

  #tolerancia OCR
  tol = 1E-6

  ctrl_num_osr = [maxiter, tol_conv, tol_search, line_step, eps]  
\end{python}

\vspace{5mm}
Definição de $f(\overrightarrow{x})$, $\overrightarrow{\nabla} f(\overrightarrow{x})$ e $\textbf{H}(\overrightarrow{x})$.
As reticências no código abaixo serão substituiídas pelos valores adequados a cada função.

\begin{python}
  def f(x):
    return ...

  def grad_f(x):
    return ...

  def hess_f(x):
    hess = np.zeros((2,2), dtype=float)
    hess[0,:] = ...
    hess[1,:] = ...
    return hess
\end{python}

\vspace{5mm}
Definição das restrições $c_l(\overrightarrow{x})$, $\overrightarrow{\nabla}c_l(\overrightarrow{x})$,
$h_k(\overrightarrow{x})$, $\overrightarrow{\nabla}h_k(\overrightarrow{x})$, $\textbf{W}_{c_l}(\overrightarrow{x})$ e
$\textbf{W}_{h_k}(\overrightarrow{x})$, sendo W a hessiana da restrição.
As reticências no código abaixo serão substituídas pelos valores adequados a cada função. Podem ser definidas quantas restrições
forem desejadas, apesar do exemplo abaixo estar com apenas uma de desigualdade e uma de igualdade.

\begin{python}

  def h1(x):
    return ...

  def grad_h1(x):
    return ...

  def hess_h1(x):
    hess = np.zeros((2,2), dtype=float)
    hess[0,:] = ...
    hess[1,:] = ...
    return hess

  def c1(x):
    return ...

  def grad_c1(x):
    return ...

  def hess_c1(x):
    hess = np.zeros((2,2), dtype=float)
    hess[0,:] = ...
    hess[1,:] = ...
    return hess
\end{python}

\vspace{5mm}
Agrupamento das restrições em listas(ou arrays) para servirem de input para o restante do programa. 
E montagem de um array auxiliar para montagem da função $\phi$ no caso do método OCR de penalidade.

\begin{python}
  h_list = [h1]
  grad_h_list = [grad_h1]
  hess_h_list = [hess_h1]

  c_list = [c1]
  grad_c_list = [grad_c1]
  hess_c_list = [hess_c1]

  #para o metodo de penalidade
  #controle de quais cls irao montar a phi
  c_mont = []
  if metodo_ocr == 1:
      for c in c_list:
          if c(x) > 0:
              c_mont.append(1)
          else:
              c_mont.append(0)
              
  params = [f, grad_f, hess_f, h_list, grad_h_list, hess_h_list, c_list, grad_c_list, hess_c_list, c_mont]
\end{python}

\vspace{5mm}
Verificação de convergência.

\begin{enumerate}
  \item Penalidade : $\frac{1}{2}r_p^kp(\overrightarrow{x}^{k+1}) < tol$, sendo $p(\overrightarrow{x}) = \sum_{k=1}^{m} h_k^2(\overrightarrow{x}) + \sum_{l=1}^{p}\{max[0,c_l(\overrightarrow{x})]\}^2$
  \item Barreira   : $r_b^kb(\overrightarrow{x}^{k+1}) < tol$, sendo $b = \sum_{l=1}^{p} -\frac{1}{c_l(\overrightarrow{x})}$
\end{enumerate}

A cada iteração do loop no código abaixo, o valor de $r$ é atualizado pelo fator $\beta$, e os parâmetros necessários para montagem
da função $\phi$ pelos métodos da penalidade e da barreira também são atualizados. Após cada atualização e verificação de convergência, 
a otimização sem restrição é chamada, e esse processo é repetido até que seja atingido o critério de convergência da otimização com
restrição.

Para o método da barreira também é feita uma avaliação
se o ponto final da iteração continua dentro das restrições. Caso negativo, o passo é refeito com um passo reduzido no método do passo
constante da busca unidimensional.



\begin{python}
  if metodo_ocr == 1:
    parc = (1/2)*r*ocr.p_penal(x, params)
  elif metodo_ocr == 2:
    parc = r*ocr.b_bar(x, params)
    
  listP_OCR = []
  listP_OCR.append(x)

  listResultsOSR = []

  passos_OCR = 0
  redo = 0
  print(n_met)
  while(parc > tol):
    passos_OCR = passos_OCR + 1
    if passos_OCR > 1:
        r = beta*r
        if metodo_ocr == 1:
            params[-1] = []
            for c in c_list:
                if c(x) > 0:
                    params[-1].append(1)
                else:
                    params[-1].append(0)
    listP_OSR, passos_OSR, conv_OSR, flag_conv_OSR, tempoExec_OSR = osr.osr_ctrl(x, params, r, ctrl_num_osr, metodo_ocr, metodo_osr)
    
    if metodo_ocr == 2:
        redo = 0
        for c in c_list:
            if c(listP_OSR[-1]) > 0:
                redo = 1
                break
    if (redo == 0):
        ctrl_num_osr[3] = line_step
        x = listP_OSR[-1]
        listP_OCR.append(x)
        listResultsOSR.append([listP_OSR, params, r, metodo_ocr, metodo_osr])
        if metodo_ocr == 1:
            parc = (1/2)*r*ocr.p_penal(x, params)
        elif metodo_ocr == 2:
            parc = r*ocr.b_bar(x, params)
        print(f'{passos_OCR}: x={x}, r={r:.4e}, passos={passos_OSR}, conv_OCR={parc:.4e}, conv_OSR={conv_OSR:.4e}, tempo={tempoExec_OSR}')
    elif (redo == 1):
        print(f'Refazendo passo {passos_OCR} com delta alpha = {0.1*ctrl_num_osr[3]}')
        passos_OCR = passos_OCR - 1
        r = r/beta
        ctrl_num_osr[3] = 0.1*ctrl_num_osr[3]    
\end{python}

\subsection{Métodos OCR}

Os algoritmos referentes ao uso dos métdos OCR da penalidade e da barreira foram implementados em um arquivo denominado ocr\_methods.py.

\vspace{3mm}
O seguinte pacote é necessário nesse arquivo:

\begin{python}
  import numpy as np
\end{python}

\subsubsection{Método de Penalidade}

\textbf{Pseudo-Função Objetivo:}

\vspace{3mm}
$\phi(\overrightarrow{x}, r_p) = f(\overrightarrow{x}) + \frac{1}{2} r_p \sum_{k=1}^{m} h_k^2(\overrightarrow{x}) 
+ \frac{1}{2} r_p \sum_{l=1}^{p} \{max[0, c_l(\overrightarrow{x})]\}^2$

\vspace{5mm}
\textbf{Cálculo do gradiente:}

\vspace{3mm}
Sendo $p(\overrightarrow{x}) = \sum_{k=1}^{m} h_k^2(\overrightarrow{x}) + \sum_{l=1}^{p}\{max[0,c_l(\overrightarrow{x})]\}^2$, temos então que
$\overrightarrow{\nabla} \phi(\overrightarrow{x}, r_p) = \overrightarrow{\nabla} f(\overrightarrow{x}) + 
\frac{1}{2} r_p \overrightarrow{\nabla} p(\overrightarrow{x})$.

\vspace{3mm}
$\overrightarrow{\nabla} p(\overrightarrow{x}) = 2\sum_{k=1}^{m} \{h_k(\overrightarrow{x}) \overrightarrow{\nabla} h_k(\overrightarrow{x}) \}
+ 2\sum_{l=1}^{p} \{c_{l}^{f}(\overrightarrow{x}) c_l(\overrightarrow{x}) \overrightarrow{\nabla} c_l(\overrightarrow{x}) \} $,
sendo $c_l^f(\overrightarrow{x}) =
\begin{cases}
  1,& \text{se } c_l(\overrightarrow{x}) > 0 \\
  0,              & \text{se } c_l(\overrightarrow{x}) \leq 0
\end{cases}$

\vspace{5mm}
\textbf{Cálculo da Hessiana:}

\vspace{3mm}
$H_{\phi}(\overrightarrow{x}) = H_f(\overrightarrow{x}) + \frac{1}{2} r_p H_p(\overrightarrow{x})$

\vspace{3mm}
$H_{p_{i\text{x}j}} = \frac{\partial ^2 p}{\partial x_i \partial x_j} = 
2\sum_{k=1}^{m} \{\frac{\partial h_k}{\partial x_j} \frac{\partial h_k}{\partial x_i}\} 
+ 2\sum_{k=1}^{m} \{h_k \frac{\partial ^2 h_k}{\partial x_j \partial x_i}\}
+2\sum_{l=1}^{p} \{c_l^f \frac{\partial c_l}{\partial x_j} \frac{\partial c_l}{\partial x_i}\}
+ 2\sum_{l=1}^{p} \{c_l^f c_l \frac{\partial ^2 c_l}{\partial x_j \partial x_i}\}$

\vspace{3mm}
Interessante notar que $\frac{\partial c_l}{\partial x_j}$,
$\frac{\partial c_l}{\partial x_i}$,
$\frac{\partial h_k}{\partial x_j}$ e
 $\frac{\partial h_k}{\partial x_i}$ são componentes conhecidos de 
$\overrightarrow{\nabla} h_k$ e $\overrightarrow{\nabla} c_l$. Além disso,
$\frac{\partial ^2 c_l}{\partial x_j \partial x_i}$ 
e $\frac{\partial ^2 h_k}{\partial x_j \partial x_i}$
são termos das hessianas das restrições e que também são conhecidos.
Dessa forma, temos todos os inputs necessários para cálculo da hessiana de 
$p(\overrightarrow{x})$ e por conseguinte a hessiana de 
$\phi(\overrightarrow{x}, r)$.

\vspace{3mm}
\begin{python}
  #Metodo da Penalidade
  def p_penal(x, params):
      #leitura dos parametros
      h_list = params[3]
      c_list = params[6]
      c_mont = params[9]
      
      p = 0
      for h in h_list:
          p = p + (h(x))**2
      
      for i in np.arange(len(c_list)):
          p = p + c_mont[i]*c_list[i](x)**2
          
      return  p

  def phi_penal(x, params, r):
      #leitura dos parametros
      f = params[0]
      h_list = params[3]
      c_list = params[6]
      c_mont = params[9]
      
      p = 0
      for h in h_list:
          p = p + (h(x))**2
      
      for i in np.arange(len(c_list)):
          p = p + c_mont[i]*c_list[i](x)**2
          
      return f(x) + (1/2)*r*p

  def grad_phi_penal(x, params, r):
      #leitura dos parametros
      grad_f = params[1]
      h_list = params[3]
      grad_h_list = params[4]
      c_list = params[6]
      grad_c_list = params[7]
      c_mont = params[9]
      
      dimens = x.size
      grad_p = np.zeros(dimens, dtype=float)
      
      for i in np.arange(len(h_list)):
          grad_p = grad_p + 2*h_list[i](x)*grad_h_list[i](x)
      for j in np.arange(len(c_list)):
          grad_p = grad_p + 2*c_mont[j]*c_list[j](x)*grad_c_list[j](x)
          
      return grad_f(x) + (1/2)*r*grad_p

  def hess_phi_penal(x, params, r):
      #leitura dos parametros
      hess_f = params[2]
      h_list = params[3]
      grad_h_list = params[4]
      hess_h_list = params[5]
      c_list = params[6]
      grad_c_list = params[7]
      hess_c_list = params[8]
      c_mont = params[9]    
      
      dimens = x.size    
      hessian_p = np.zeros((dimens, dimens), dtype=float)
      
      for i in np.arange(dimens):    
          for j in np.arange(dimens):
              for k in np.arange(len(grad_h_list)):
                  hessian_p[i,j] = hessian_p[i,j] + 2*grad_h_list[k](x)[i]*grad_h_list[k](x)[j]
              for l in np.arange(len(grad_cl_list)):
                  hessian_p[i,j] = hessian_p[i,j] + 2*c_mont[l]*grad_c_list[l](x)[j]*grad_c_list[l](x)[i]
      
      for k in np.arange(len(h_list)):
          hessian_p = hessian_p + 2*h_list[k](x)*hess_h_list[k](x)
      
      for k in np.arange(len(c_list)):
          hessian_p = hessian_p + 2*c_mont[k]*c_list[k](x)*hess_c_list[k](x)
      
      return hess_f(x) + (1/2)*r*hessian_p
\end{python}

\subsubsection{Método de Barreira}

\textbf{Pseudo-Função Objetivo:}

\vspace{3mm}
$\phi(\overrightarrow{x}, r_b) = f(\overrightarrow{x}) + 
r_b \sum_{l=1}^{m} -\frac{1}{c_l(\overrightarrow{x})}$

\vspace{5mm}
\textbf{Cálculo do gradiente:}

\vspace{3mm}
Sendo $b(\overrightarrow{x}) = \sum_{l=1}^{m} -\frac{1}{c_l(\overrightarrow{x})}$
, temos então que
$\overrightarrow{\nabla} \phi(\overrightarrow{x}, r_b) = 
\overrightarrow{\nabla} f(\overrightarrow{x}) + 
r_b \overrightarrow{\nabla} b(\overrightarrow{x})$.

\vspace{3mm}
$\overrightarrow{\nabla} b =
\sum_{l=1}^{m} \frac{\overrightarrow{\nabla} c_l}{c_l^2}$

\vspace{5mm}
\textbf{Cálculo da Hessiana:}

\vspace{3mm}
$H_{\phi}(\overrightarrow{x}) = H_f(\overrightarrow{x}) 
+ r_b H_b(\overrightarrow{x})$

\vspace{3mm}
$H_{b_{i\text{x}j}} = -2 \sum_{l=1}^{m} \{ \frac{1}{c_l^3}\frac{\partial c_l}{\partial x_i} \frac{\partial c_l}{\partial x_j} \}
+ \sum_{l=1}^{m} \{ \frac{1}{c_l^2} \frac{\partial ^2 c_l}{\partial x_i \partial x_j}            \}$

\vspace{3mm}
$\frac{\partial c_l}{\partial x_j}$ e
$\frac{\partial c_l}{\partial x_i}$ são componentes conhecidos de $\overrightarrow{\nabla} c_l$.
Além disso,
$\frac{\partial^2 c_l}{\partial x_j \partial x_i}$ são termos da hessiana das restrições e que também são conhecidos.
Dessa forma, temos todos os inputs necessários para cálculo da hessiana de 
$b(\overrightarrow{x})$ e por conseguinte a hessiana de 
$\phi(\overrightarrow{x}, r)$.

\begin{python}
  #### Metodo da Barreira 
  def phi_bar(x, params, r):
      #leitura dos parametros
      f = params[0]
      c_list = params[6]
      
      b = 0    
      for c in c_list:
          b = b - 1/cl(x)
                  
      return f(x) + r*b

  def b_bar(x, params):
      #leitura dos parametros
      c_list = params[6]
      
      b = 0
      for c in c_list:
          b = b - 1/c(x)
              
      return b

  def grad_phi_bar(x, params, r):
      #leitura dos parametros
      grad_f = params[1]
      c_list = params[6]
      grad_c_list = params[7]
      
      dimens = x.size
      grad_b = np.zeros(dimens, dtype=float)
      
      for i in np.arange(len(c_list)):
          grad_b = grad_b + (c_list[i](x))**(-2)*grad_c_list[i](x)
              
      return grad_f(x) + r*grad_b

  def hess_phi_bar(x, params, r):
      #leitura dos parametros
      hess_f = params[2]
      c_list = params[6]
      grad_c_list = params[7]
      hess_c_list = params[8]
      
      dimens = x.size    
      hessian_b = np.zeros((dimens, dimens), dtype=float)
      
      for i in np.arange(dimens):    
          for j in np.arange(dimens):
              for k in np.arange(len(c_list)):
                  hessian_b[i,j] = hessian_b[i,j] - 2*((c_list[k](x))**(-3))*grad_c_list[k](x)[i]*grad_c_list[k](x)[j]
      
      for k in np.arange(len(cl_list)):
          hessian_b = hessian_b + ((c_list[k](x))**(-2))*hess_c_list[k](x)
      
      return hess_f(x) + r*hessian_b
\end{python}

\subsection{Métodos e Otimização OSR}

Os algoritmos de otimização sem restrição, Univariante, Powell, Steepest Descent, Newton-Raphson, Flecther-Reeves
e BFGS foram implementados em um arquivo denominado osr\_methods.py. Uma função chamada osr\_ctrl também está presente
nesse arquivo e faz a interface entre o código principal e os métodos de OSR.  O código principal chama essa função a cada iteração
OCR, e ela por sua vez faz todo o tratamento da OSR, chamando as funções $\phi$ do módulo OCR apresentado na seção anterior E
retornando os resultados para o código principal.

Esses códigos foram discutidos com mais detalhes no trabalho 1.

\begin{python}
  import numpy as np
  import osr_methods as osr
  import line_search_methods as lsm
  import ocr_methods as ocr
  from timeit import default_timer as timer

  def univariante(passo, dimens):
      #indice do vetor = (resto da divisao do passo pela dimensao) - 1
      #primeira posicao do vetor no python tem indice 0
      indice = passo%dimens - 1
      
      if (indice == -1) :
          #indice = -1 indica que se trata da ultima posicao do array
          #no pyton esse indice eh o tamanho do vetor - 1
          indice = dimens - 1
          
      #define a direcao canonica a ser utilizada
      ek = np.zeros(dimens)
      ek[indice] = 1
      
      return ek
      
  def powell(P, P0, direcoes, passos, ciclos, dimens):
      #indice do vetor = (resto da divisao do passo pela dimensao) - 1
      #primeira posicao do vetor no python tem indice 0
      indice = passos%(dimens + 1) - 1
      
      if (indice == -1):
          #indice = -1 indica que se trata da ultima posicao do array
          #no pyton esse indice eh o tamanho do vetor - 1
          #direcao n + 1 do ciclo = Patual - P0
          dir = P - P0
          direcoes[dimens - 1] = dir        
      elif (indice == 0):
          #indice = 0 significa que vamos usar a primeira direcao do conjunto
          #representa o inicio de um novo ciclo
          ciclos = ciclos + 1

          if (ciclos%(dimens+2) == 0):
              #se ciclo for multipl de dimens + 2, conjunto de direcoes = canonicas
              direcoes = np.eye(dimens, dtype=float)
          P0 = P.copy()
          dir = direcoes[indice].copy()
          
      else:
          dir = direcoes[indice].copy()
          direcoes[indice-1] = dir
    
      return dir, direcoes, P0, ciclos            

  def newtonRaphson(grad_P, hessian_f):
      return -np.linalg.inv(hessian_f).dot(grad_P)

  def steepestDescent(grad):
      return -grad

  def fletcherReeves(dir_last, grad, grad_last, passo):
      if passo == 1:
          grad_last = grad.copy()
          return -grad, grad_last
      else:
          beta = (np.linalg.norm(grad)/np.linalg.norm(grad_last))**2
          grad_last = grad.copy()
          return -grad + beta*dir_last, grad_last
      
  def bfgs(P, P_last, grad, grad_last, S_last, passo, dimens):
      if (passo == 1):
          dir = -S_last.dot(grad)
      else:
          delta_x_k = P - P_last
          delta_g_k = grad - grad_last
          
          #para o numpy, vetor 1-D linha e vetor coluna sao a mesma coisa (nao e necessrio transpor)
          #matrizes
          A = np.outer(delta_x_k, np.transpose(delta_x_k))
          B = S_last.dot(np.outer(delta_g_k, np.transpose(delta_x_k)))
          C = np.outer(delta_x_k, np.transpose(S_last.dot(delta_g_k)))
          
          #Escalares        
          d = np.transpose(delta_x_k).dot(delta_g_k)
          e = np.transpose(delta_g_k).dot(S_last.dot(delta_g_k))
                  
          S = S_last + (d + e)*A/(d**2) - (B + C)/d
          dir = -S.dot(grad)
          S_last = S.copy()
      P_last = P
      grad_last = grad
      return dir, P_last, grad_last, S_last

  def osr_ctrl(P0, params, r, ctrl_num, metodo_ocr, metodo_osr):
      #controle numerico
      maxiter = ctrl_num[0]
      tol_conv = ctrl_num[1]
      tol_search = ctrl_num[2]
      line_step = ctrl_num[3]
      eps = ctrl_num[4]
      
      metodo = metodo_osr
          
      #inicializacoes auxiliares dos metodos de OSR
      passos = 0
      dimens = P0.size
      Pmin = P0.copy()
      listPmin = []
      listPmin.append(Pmin)
      
      if metodo_ocr == 1:
          grad = ocr.grad_phi_penal(Pmin, params, r)
      elif metodo_ocr == 2:
          grad = ocr.grad_phi_bar(Pmin, params, r)
      
      norm_grad = np.linalg.norm(grad)
      flag_conv = True

      if (metodo == 2):
          direcoes = np.eye(dimens, dtype=float)
          ciclos = 0
          P1 = P0.copy()
      elif (metodo == 5):
          #o metodo recebe a direcao anterior 
          #inicializo a direcao com um vetor de zeros mas que nunca e usado
          #uso apenas para enviar como parametro na primeira iteracao do metodo, o qual atualiza o valor de dir para a iteracao seguinte
          dir = np.zeros((1, dimens))
          grad_last = grad.copy()
      elif(metodo == 6):
          S_last = np.eye(dimens)
          grad_last = grad.copy()
          P_last = P0.copy()
      
      #calculo do Pmin
      start = timer()
      while (norm_grad > tol_conv):
          if (passos == maxiter):
              flag_conv = False
              break
          passos = passos + 1
          if (metodo == 1):
              dir = osr.univariante(passos, dimens)
          elif (metodo == 2):
              dir, direcoes, P1, ciclos = osr.powell(Pmin, P1, direcoes,passos, ciclos, dimens)
          elif (metodo == 3):
              dir = osr.steepestDescent(grad)
          elif (metodo == 4):
              if metodo_ocr == 1:
                  hess = ocr.hess_phi_penal(Pmin, params, r)
              elif metodo_ocr == 2:
                  hess = ocr.hess_phi_bar(Pmin, params,r)
              dir = osr.newtonRaphson(grad, hess)
          elif (metodo == 5):
              dir, grad_last = osr.fletcherReeves(dir, grad, grad_last, passos)
          elif (metodo == 6):
              dir, P_last, grad_last, S_last = osr.bfgs(Pmin, P_last, grad, grad_last, S_last, passos, dimens)
          dir = dir/np.linalg.norm(dir)
          intervalo = lsm.passo_cte(dir, Pmin, params, r, metodo_ocr, eps, line_step)
          alpha = lsm.secao_aurea(intervalo, dir, Pmin, params, r, metodo_ocr, tol_search)
          Pmin = Pmin + alpha*dir
          listPmin.append(Pmin)
          
          
          if metodo_ocr == 1:
              grad = ocr.grad_phi_penal(Pmin, params, r)
          elif metodo_ocr == 2:
              grad = ocr.grad_phi_bar(Pmin, params, r)
              
          norm_grad = np.linalg.norm(grad)
          
      end = timer()
      tempoExec = end - start
      
      return listPmin, passos, norm_grad, flag_conv, tempoExec
\end{python}


\subsection{Busca Unidirecional}

Os algoritmos dos métodos do Passo Constante e da Seção Áurea foram implementados em um arquivo denominado line\_search\_methods.py.
Códigos discutidos no trabalho 1. Pequena adaptação realizada para trabalhar com as funções $\phi$ dos métodos OCR.


\begin{python}
  import ocr_methods as ocr
  import numpy as np
  
  def passo_cte(direcao, P0, params, r, metodo_ocr, eps = 1E-8, step = 0.01):
      #line search pelo metodo do passo constante
      
      #define o sentido correto de busca
      if metodo_ocr == 1:
          f1 = ocr.phi_penal(P0 - eps*(direcao/np.linalg.norm(direcao)), params, r)
          f2 = ocr.phi_penal(P0 + eps*(direcao/np.linalg.norm(direcao)), params, r)
      elif metodo_ocr == 2:
          f1 = ocr.phi_bar(P0 - eps*(direcao/np.linalg.norm(direcao)), params, r)
          f2 = ocr.phi_bar(P0 + eps*(direcao/np.linalg.norm(direcao)), params, r)
          
      if (f1 > f2):
          sentido_busca = direcao.copy()
          flag = 0
      else:
          sentido_busca = -direcao.copy()
          flag = 1
          
      P = P0.copy()
      P_next = P + step*sentido_busca
      alpha = 0
      
      if metodo_ocr == 1:
          f1 = ocr.phi_penal(P, params, r)
          f2 = ocr.phi_penal(P_next, params, r)
      elif metodo_ocr == 2:
          f1 = ocr.phi_bar(P, params, r)
          f2 = ocr.phi_bar(P_next, params, r)
        
      while (f1 > f2):           
          alpha = alpha + step
          P = P0 + alpha*sentido_busca
          P_next = P0 + (alpha+step)*sentido_busca        
          
          if metodo_ocr == 1:
              f1 = ocr.phi_penal(P, params, r)
              f2 = ocr.phi_penal(P_next, params, r)
              f_eps = ocr.phi_penal(P - eps*(sentido_busca/np.linalg.norm(sentido_busca)), params, r)
          elif metodo_ocr == 2:
              f1 = ocr.phi_bar(P, params, r)
              f2 = ocr.phi_bar(P_next, params, r)
              f_eps = ocr.phi_bar(P - eps*(sentido_busca/np.linalg.norm(sentido_busca)), params, r)
              
          if (f_eps < f1):
              alpha = alpha - step
              break
      
      intervalo = np.array([alpha, alpha + step])
      
      if(flag == 1):
          intervalo = -intervalo
          
      #retorna o intervalo de busca = [alpha min, alpha min + step]                 
      return intervalo
      
  def secao_aurea(intervalo, direcao, P0, params, r, metodo_ocr, tol=0.00001):
      #line search pelo metodo da secao aurea
      
      #verifica o sentido da busca
      if(intervalo[1] < 0):
          intervalo = -intervalo
          sentido_busca = -direcao.copy()
          flag = 1
      else:
          sentido_busca = direcao.copy()
          flag = 0
      
      #atribui os limites superior e inferior da busca a variaveis internas do metodo
      alpha_upper = intervalo[1]
      alpha_lower = intervalo[0]
      beta = alpha_upper - alpha_lower
      
      #razao aurea
      Ra = (np.sqrt(5)-1)/2
      
      # define os pontos de analise de f com base na razao aurea
      alpha_e = alpha_lower + (1-Ra)*beta
      alpha_d = alpha_lower + Ra*beta 
      
      #primeira iteracao avalia f nos 2 pontos selecionados pela razao aurea
      if metodo_ocr == 1:
          f1 = ocr.phi_penal(P0 + alpha_e*sentido_busca, params, r)
          f2 = ocr.phi_penal(P0 + alpha_d*sentido_busca, params, r)
      elif metodo_ocr == 2:
          f1 = ocr.phi_bar(P0 + alpha_e*sentido_busca, params, r)
          f2 = ocr.phi_bar(P0 + alpha_d*sentido_busca, params, r)
      
      #loop enquanto a convergencia nao for obtida
      while (beta > tol):
          if (f1 > f2):
              #caso positivo, define novo intervalo variando de alpha_e ate alpha_upper
              # e aproveita os valores anteriores de alpha_d e f2 como novos alpha_e e f1
              alpha_lower = alpha_e
              f1 = f2
              alpha_e = alpha_d
              
              #calcula novo alpha_d e f2=f(alpha_d)
              beta = alpha_upper - alpha_lower
              alpha_d = alpha_lower + Ra*beta
              
              if metodo_ocr == 1:
                  f2 = ocr.phi_penal(P0 + alpha_d*sentido_busca, params, r)
              elif metodo_ocr == 2:
                  f2 = ocr.phi_bar(P0 + alpha_d*sentido_busca, params, r) 
                  
          else:
              #caso negativo, define novo intervalo variando de alpha_lower ate alpha_d
              # e aproveita os valores anteriores de alpha_e e f1 como novos alpha_d e f2
              alpha_upper = alpha_d
              f2 = f1
              alpha_d = alpha_e
              
              #calcula novo alpha_e e f1=f(alpha_e)
              beta = alpha_upper - alpha_lower
              alpha_e = alpha_lower + (1-Ra)*beta
              
              if metodo_ocr == 1:
                  f1 = ocr.phi_penal(P0 + alpha_e*sentido_busca, params, r)
              elif metodo_ocr == 2:
                  f1 = ocr.phi_bar(P0 + alpha_e*sentido_busca, params, r)
              
      # calcula Pmin e alpha min apos convergencia
      alpha_med = (alpha_lower + alpha_upper)/2
      alpha_min = alpha_med
      
      if (flag == 1):
          alpha_min = -alpha_min
      
      return alpha_min
\end{python}


\section{Teste da Implementação}

\subsection{Problema 1}

\begin{center}
  $\begin{cases}
    \textbf{Min} \hspace{4mm}  f(x_1, x_2) = (x_1 - 2)^4 + (x_1 - 2x_2)^2 \\
    \textbf{s.t.:} \hspace{4mm} x_1^2 - x_2 \leq 0
  \end{cases}$
\end{center}
Obs.: Adotar $r_p^0 = 1$, $\beta = 10$ e $x^0 = \{3,2\}$ para o método de penalidade e
$r_b^0 = 10$, $\beta = 0.1$ e $x^0 = \{0,1\}$ para o método de barreira.

\vspace{3mm}
$\overrightarrow{\nabla} f (\overrightarrow{x}) = \{4(x_1 - 2)^3 + 2(x_1 - 2x_2), -4(x_1 - 2x_2)\}$

\vspace{3mm}
$H_{f_{1\text{x}1}} = 12(x_1 - 2)^2 + 2$, \hspace{4mm}
$H_{f_{1\text{x}2}} = -4$, \hspace{4mm}
$H_{f_{2\text{x}1}} = -4$, \hspace{4mm}
$H_{f_{2\text{x}2}} = 8$

\vspace{3mm}
\textbf{Definição da função, seu gradiente e sua hessiana, no código principal:}

\begin{python}
  def f(x):
      return (x[0]-2)**4 + (x[0] - 2*x[1])**2

  def grad_f(x):
      return np.array([4*(x[0]-2)**3 + 2*(x[0] - 2*x[1]), 2*(x[0] - 2*x[1])*(-2)])

  def hess_f(x):
      hess = np.zeros((2,2), dtype=float)
      hess[0,:] = np.array([12*(x[0]-2)**2 + 2, -4.])
      hess[1,:] = np.array([-4., 8.])
      return hess
\end{python}

\vspace{3mm}
$c(\overrightarrow{x}) = x_1^2 - x_2$

\vspace{2mm}
$\overrightarrow{\nabla} c (\overrightarrow{x}) = \{2x_1, -1\}$

\vspace{2mm}
$H_{c_{1\text{x}1}} = 2$, \hspace{4mm}
$H_{c_{1\text{x}2}} = 0$, \hspace{4mm}
$H_{c_{2\text{x}1}} = 0$, \hspace{4mm}
$H_{c_{2\text{x}2}} = 0$

\vspace{3mm}
\textbf{Definição da restrição, seu gradiente e sua hessiana, no código principal:}

\begin{python}
  def c1(x):
      return x[0]**2 - x[1]

  def grad_c1(x):
      return np.array([2*x[0], -1.])

  def hess_c1(x):
      hess = np.zeros((2,2), dtype=float)
      hess[0,:] = np.array([2., 0.])
      hess[1,:] = np.array([0., 0])
      return hess
\end{python}

\subsubsection{Penalidade - Prob. 1}

Definição do ponto inicial $x^0 = \{3,2\}$ no código principal :

\begin{python}
  x = np.array([3., 2.])
\end{python}

\vspace{3mm}
\textbf{Controle numérico e parâmetros:}

\begin{itemize}
  \item Máximas iterações na OSR : $1000$
  \item Tolerância OSR: $10^{-6}$
  \item Tolerância Seção Áurea: $10^{-7}$
  \item $\Delta \alpha$: $10^{-2}$
  \item $\epsilon$: $10^{-10}$
  \item Tolerância OCR: $10^{-6}$
  \item $r_p^0 = 1$
  \item $\beta = 10$
\end{itemize}

\vspace{5mm}
\begin{table}[H]
  \begin{center}
    \begin{tabular}{c|c|c|c|c|c|c}
      \multicolumn{7}{c}{\textbf{Prob. 1 - Penalidade - Univariante}}\\
      \hline
      \textbf{Iter} & \textbf{$P_{min}$} & \textbf{r} & \textbf{\# Passos} & \textbf{Conv\_OCR} & \textbf{Conv\_OSR} & \textbf{t(s)}\\
      \hline
        1& [1.25174114 0.7304246 ]& 1e+00& 27& 3.5e-01& 9.6e-07& 0.028   \\
        2& [1.02501305 0.81147611]& 1e+01& 1000& 2.9e-01& 3.7e-06& 0.230\\
        3& [0.95576688 0.88122316]& 1e+02& 1000& 5.2e-02& 3.5e-06& 0.189\\
        4& [0.94659013 0.89267782]& 1e+03& 1000& 5.6e-03& 1.6e-03& 0.214\\
        5& [0.94526659 0.89319248]& 1e+04& 1000& 5.7e-04& 1.5e-02& 0.190\\
        6& [0.94513054 0.89323808]& 1e+05& 1000& 5.7e-05& 1.5e-02& 0.195\\
        7& [0.94511456 0.89323814]& 1e+06& 1000& 5.7e-06& 3.3e-02& 0.180\\
        8& [0.94511298 0.8932382 ]& 1e+07& 1000& 6.2e-07& 3.0e-01& 0.228\\
    \end{tabular}
  \end{center}
  \caption{Resultados obtidos para o problema 1, método de penalidade, univariante para $x^0=\{3,2\}$}
\end{table}

\vspace{5mm}
\begin{table}[H]
  \begin{center}
    \begin{tabular}{c|c|c|c|c|c|c}
      \multicolumn{7}{c}{\textbf{Prob. 1 - Penalidade - Powell}}\\
      \hline
      \textbf{Iter} & \textbf{$P_{min}$} & \textbf{r} & \textbf{\# Passos} & \textbf{Conv\_OCR} & \textbf{Conv\_OSR} & \textbf{t(s)}\\
      \hline
        1& [1.25174105 0.73042442]& 1e+00& 6& 3.5e-01& 2.2e-07& 0.027     \\
        2& [1.02501313 0.81147619]& 1e+01& 7& 2.9e-01& 9.1e-07& 0.005\\
        3& [0.95576689 0.88122318]& 1e+02& 53& 5.2e-02& 5.1e-07& 0.023\\
        4& [0.94663397 0.89276033]& 1e+03& 38& 5.6e-03& 4.8e-07& 0.013\\
        5& [0.94568845 0.89398972]& 1e+04& 30& 5.7e-04& 2.8e-07& 0.009\\
        6& [0.94559354 0.89411344]& 1e+05& 78& 5.7e-05& 3.0e-08& 0.019\\
        7& [0.94558401 0.89412573]& 1e+06& 1000& 5.8e-06& 5.2e-02& 0.264\\
        8& [0.94558315 0.89412714]& 1e+07& 1000& 5.9e-07& 1.4e-01& 0.217\\
    \end{tabular}
  \end{center}
  \caption{Resultados obtidos para o problema 1, método de penalidade, powell para $x^0=\{3,2\}$}
\end{table}

\vspace{5mm}
\begin{table}[H]
  \begin{center}
    \begin{tabular}{c|c|c|c|c|c|c}
      \multicolumn{7}{c}{\textbf{Prob. 1 - Penalidade - Steepest Descent}}\\
      \hline
      \textbf{Iter} & \textbf{$P_{min}$} & \textbf{r} & \textbf{\# Passos} & \textbf{Conv\_OCR} & \textbf{Conv\_OSR} & \textbf{t(s)}\\
      \hline
        1& [1.2517411  0.73042453]& 1e+00& 25& 3.5e-01& 6.2e-07& 0.025    \\
        2& [1.02501316 0.81147628]& 1e+01& 1000& 2.9e-01& 1.6e-06& 0.254\\
        3& [0.9557669  0.88122324]& 1e+02& 1000& 5.2e-02& 1.2e-05& 0.248\\
        4& [0.94663395 0.89276024]& 1e+03& 1000& 5.6e-03& 1.1e-04& 0.210\\
        5& [0.94568838 0.89398961]& 1e+04& 1000& 5.7e-04& 1.3e-04& 0.250\\
        6& [0.94556416 0.89405783]& 1e+05& 1000& 5.7e-05& 1.1e-02& 0.203\\
        7& [0.94555192 0.89406498]& 1e+06& 1000& 6.0e-06& 1.8e-01& 0.239\\
        8& [0.94555068 0.89406568]& 1e+07& 1000& 8.9e-07& 1.8e+00& 0.212\\
    \end{tabular}
  \end{center}
  \caption{Resultados obtidos para o problema 1, método de penalidade, steepest descent para $x^0=\{3,2\}$}
\end{table}

\vspace{5mm}
\begin{table}[H]
  \begin{center}
    \begin{tabular}{c|c|c|c|c|c|c}
      \multicolumn{7}{c}{\textbf{Prob. 1 - Penalidade - Newton-Raphson}}\\
      \hline
      \textbf{Iter} & \textbf{$P_{min}$} & \textbf{r} & \textbf{\# Passos} & \textbf{Conv\_OCR} & \textbf{Conv\_OSR} & \textbf{t(s)}\\
      \hline
        1& [1.25174109 0.73042445]& 1e+00& 3& 3.5e-01& 2.7e-07& 0.035     \\
        2& [1.02501318 0.81147627]& 1e+01& 4& 2.9e-01& 8.9e-08& 0.009\\
        3& [0.95576692 0.88122322]& 1e+02& 4& 5.2e-02& 1.0e-07& 0.008\\
        4& [0.94663397 0.89276032]& 1e+03& 1000& 5.6e-03& 4.6e-06& 0.668\\
        5& [0.94568841 0.89398971]& 1e+04& 1000& 5.7e-04& 1.3e-03& 0.422\\
        6& [0.94559354 0.89411344]& 1e+05& 1000& 5.7e-05& 5.1e-04& 0.502\\
        7& [0.94558405 0.89412582]& 1e+06& 1000& 5.7e-06& 2.0e-04& 0.526\\
        8& [0.9455831  0.89412707]& 1e+07& 1000& 5.7e-07& 4.2e-03& 0.401\\
    \end{tabular}
  \end{center}
  \caption{Resultados obtidos para o problema 1, método de penalidade, Newton-Raphson para $x^0=\{3,2\}$}
\end{table}

\vspace{5mm}
\begin{table}[H]
  \begin{center}
    \begin{tabular}{c|c|c|c|c|c|c}
      \multicolumn{7}{c}{\textbf{Prob. 1 - Penalidade - Fletcher-Reeves}}\\
      \hline
      \textbf{Iter} & \textbf{$P_{min}$} & \textbf{r} & \textbf{\# Passos} & \textbf{Conv\_OCR} & \textbf{Conv\_OSR} & \textbf{t(s)}\\
      \hline
        1& [1.24978283 0.72770155]& 1e+00& 1000& 3.5e-01& 1.9e-02& 0.338\\
        2& [1.0247375  0.81081343]& 1e+01& 1000& 2.9e-01& 5.2e-03& 0.305\\
        3& [0.95598843 0.88169493]& 1e+02& 1000& 5.2e-02& 1.0e-02& 0.253\\
        4& [0.94673281 0.89294715]& 1e+03& 1000& 5.6e-03& 2.2e-03& 0.292\\
        5& [0.94569677 0.89400545]& 1e+04& 1000& 5.7e-04& 2.9e-04& 0.259\\
        6& [0.94557152 0.89407169]& 1e+05& 1000& 5.7e-05& 2.0e-02& 0.250\\
        7& [0.94555891 0.89407834]& 1e+06& 1000& 5.5e-06& 1.4e-01& 0.235\\
        8& [0.94555764 0.89407898]& 1e+07& 1000& 3.7e-07& 1.4e+00& 0.244\\
    \end{tabular}
  \end{center}
  \caption{Resultados obtidos para o problema 1, método de penalidade, Flecther-Reeves para $x^0=\{3,2\}$}
\end{table}

\vspace{5mm}
\begin{table}[H]
  \begin{center}
    \begin{tabular}{c|c|c|c|c|c|c}
      \multicolumn{7}{c}{\textbf{Prob. 1 - Penalidade - BFGS}}\\
      \hline
      \textbf{Iter} & \textbf{$P_{min}$} & \textbf{r} & \textbf{\# Passos} & \textbf{Conv\_OCR} & \textbf{Conv\_OSR} & \textbf{t(s)}\\
      \hline
        1& [1.25174106 0.73042443]& 1e+00& 6& 3.5e-01& 4.5e-08& 0.021      \\
        2& [1.02501318 0.81147628]& 1e+01& 4& 2.9e-01& 5.3e-08& 0.003\\
        3& [0.95576696 0.88122325]& 1e+02& 1000& 5.2e-02& 8.7e-06& 0.278\\
        4& [0.94663396 0.89276031]& 1e+03& 4& 5.6e-03& 6.6e-07& 0.001\\
        5& [0.94568843 0.8939897 ]& 1e+04& 1000& 5.7e-04& 6.2e-06& 0.244\\
        6& [0.94559353 0.89411343]& 1e+05& 1000& 5.7e-05& 3.1e-03& 0.259\\
        7& [0.94558406 0.89412585]& 1e+06& 1000& 5.7e-06& 1.0e-03& 0.276\\
        8& [0.94558308 0.89412702]& 1e+07& 1000& 5.7e-07& 7.1e-03& 0.271\\
    \end{tabular}
  \end{center}
  \caption{Resultados obtidos para o problema 1, método de penalidade, BFGS para $x^0=\{3,2\}$}
\end{table}




\subsubsection{Ponto inicial: $x^0 = \{2,2\}^t $}
Definição do ponto inicial no código principal:
\begin{python}
  P0 = np.array([2, 2])
\end{python}

\vspace{3mm}
\textbf{Principais resultados obtidos: }

\vspace{3mm}
A tabela resumo abaixo mostra que para essa função quadrática os métodos apresentaram resultados satisfatórios,
com os métodos de Powell, Fletcher-Reeves, BFGS e Newton-Rapshon respeitando o número máximo de passos para convergência.

\begin{table}[H]
  \begin{center}
    \begin{tabular}{c|c|c|c|c|c|c}
      \textbf{Iter} & \textbf{$P_{min}$} & \textbf{r} & \textbf{\# Passos} & \textbf{conv\_OCR} & \textbf{conv\_OSR} & \textbf{Tempo}\\
      \hline
        1:& [1.25174114 0.7304246 ]& 1e+00& 27& 3.5e-01& 9.6e-07& 0.028   \\
        2:& [1.02501305 0.81147611]& 1e+01& 1000& 2.9e-01& 3.7e-06& 0.230\\
        3:& [0.95576688 0.88122316]& 1e+02& 1000& 5.2e-02& 3.5e-06& 0.189\\
        4:& [0.94659013 0.89267782]& 1e+03& 1000& 5.6e-03& 1.6e-03& 0.214\\
        5:& [0.94526659 0.89319248]& 1e+04& 1000& 5.7e-04& 1.5e-02& 0.190\\
        6:& [0.94513054 0.89323808]& 1e+05& 1000& 5.7e-05& 1.5e-02& 0.195\\
        7:& [0.94511456 0.89323814]& 1e+06& 1000& 5.7e-06& 3.3e-02& 0.180\\
        8:& [0.94511298 0.8932382 ]& 1e+07& 1000& 6.2e-07& 3.0e-01& 0.228\\
    \end{tabular}
  \end{center}
  \caption{Resumo dos resultados obtidos na questão 1a para $x^0 = \{2,2\}^t$}
\end{table}

% \begin{figure}[H]
%   \centering
%   \includegraphics[scale=0.4]{figuras/q1a_passos_P1.PNG}
%   \caption{Número de passos por método OSR. Questão 1a e $x^0 = \{2,2\}^t$ }
% \end{figure}

\vspace{3mm}
A figura abaixo mostra, por método, o valor da função a cada iteração. A ideia é podermos comparar 
a rapidez com que cada método se aproxima do mínimo. Para essa função e ponto inicial, os métodos univariante e Powell
foram os que mais demoraram para se aproximar, levando por volta de 6 passos, enquanto os demais precisaram de no máximo
3 passos.

% \begin{figure}[H]
%   \centering
%   \includegraphics[scale=0.45]{figuras/q1a_fxpassos_P1.PNG}
%   \caption{Gráficos de $f(x_1,x_2)$ versus passo da minimização, por método. Questão 1a e $x^0 = \{2,2\}^t$}
% \end{figure}

As figuras abaixo representam as curvas de nível e o caminho de otimização percorrido por cada método. Como a função 
só possui um mínimo, todos os métodos convergiram o ponto correto (mínimo mais próximo).

% \begin{figure}[H]
%   \centering
%   \begin{subfigure}[b]{\textwidth}
%     \includegraphics[width=0.49\textwidth]{figuras/Q1.a_Univariante_P0=[2 2].pdf}
%     \includegraphics[width=0.49\textwidth]{figuras/Q1.a_Powell_P0=[2 2].pdf}
%   \end{subfigure}
%   \caption{Curvas de nível e os pontos obtidos pelo Univariante e Powell. Questão 1a e $x^0 = \{2,2\}^t$}
% \end{figure}

% \begin{figure}[H]
%   \centering
%   \begin{subfigure}[b]{\textwidth}
%     \includegraphics[width=0.49\textwidth]{figuras/Q1.a_Steepest Descent_P0=[2 2].pdf}
%     \includegraphics[width=0.49\textwidth]{figuras/Q1.a_Fletcher-Reeves_P0=[2 2].pdf}
%   \end{subfigure}
%   \caption{Curvas de nível e os pontos obtidos pelo Steepest Descent e Fletcher-Reeves. Questão 1a e $x^0 = \{2,2\}^t$}
% \end{figure}

% \begin{figure}[H]
%   \centering
%   \begin{subfigure}[b]{\textwidth}
%     \includegraphics[width=0.49\textwidth]{figuras/Q1.a_BFGS_P0=[2 2].pdf}
%     \includegraphics[width=0.49\textwidth]{figuras/Q1.a_Newton Raphson_P0=[2 2].pdf}
%   \end{subfigure}
%   \caption{Curvas de nível e os pontos obtidos pelo BFGS e Newton-Raphson. Questão 1a e $x^0 = \{2,2\}^t$}
% \end{figure}

\subsubsection{Ponto inicial : $x^0 = \{-1,-3\}^t$}
Definição do ponto inicial no código principal:
\begin{python}
  P0 = np.array([-1, -3])
\end{python}

\vspace{3mm}
\textbf{Principais resultados obtidos:}

\vspace{3mm}
Segue tabela resumo e figuras nos mesmos moldes do que foi apresentado para o outro ponto inicial.
As conclusões são basicamente as mesmas.

\begin{table}[H]
  \begin{center}
    \begin{tabular}{c|c|c|c}
      \textbf{Método} & \textbf{\# Passos} & \textbf{Tempo(s)} & \textbf{$P_{min}$}\\
      \hline
      Univariante & 48 & 0.04435 & $\{-0.7142935, -0.14286022\}^t$\\
      Powell & 6 & 0.02483 & $\{-0.71428418, -0.14285757\}^t$\\
      Steepest Descent & 7 & 0.00690 & $\{-0.71429411, -0.14286059\}^t$\\
      Fletcher-Reeves & 3 & 0.00467 & $\{-0.71428581, -0.14285718\}^t$\\
      BFGS & 2 & 0.00332 & $\{-0.71428583, -0.14285714\}^t$\\
      Newton-Raphson & 1 & 0.00219 & $\{-0.71428562, -0.1428562\}^t$\\
    \end{tabular}
  \end{center}
  \caption{Resumo dos resultados obtidos na questão 1a para $x^0 = \{-1,-3\}^t$}
\end{table}

% \begin{figure}[H]
%   \centering
%   \includegraphics[scale=0.4]{figuras/q1a_passos_P2.PNG}
%   \caption{Número de passos por método OSR. Questão 1a e $x^0 = \{-1,-3\}^t$ }
% \end{figure}

% \begin{figure}[H]
%   \centering
%   \includegraphics[scale=0.45]{figuras/q1a_fxpassos_P2.PNG}
%   \caption{Gráficos de $f(x_1,x_2)$ versus passo da minimização, por método. Questão 1a e $x^0 = \{-1,-3\}^t$}
% \end{figure}

% \begin{figure}[H]
%   \centering
%   \begin{subfigure}[b]{\textwidth}
%     \includegraphics[width=0.49\textwidth]{figuras/Q1.a_Univariante_P0=[-1 -3].pdf}
%     \includegraphics[width=0.49\textwidth]{figuras/Q1.a_Powell_P0=[-1 -3].pdf}
%   \end{subfigure}
%   \caption{Curvas de nível e os pontos obtidos pelo Univariante e Powell. Questão 1a e $x^0 = \{-1,-3\}^t$}
% \end{figure}

% \begin{figure}[H]
%   \centering
%   \begin{subfigure}[b]{\textwidth}
%     \includegraphics[width=0.49\textwidth]{figuras/Q1.a_Steepest Descent_P0=[-1 -3].pdf}
%     \includegraphics[width=0.49\textwidth]{figuras/Q1.a_Fletcher-Reeves_P0=[-1 -3].pdf}
%   \end{subfigure}
%   \caption{Curvas de nível e os pontos obtidos pelo Steepest Descent e Fletcher-Reeves. Questão 1a e $x^0 = \{-1,-3\}^t$}
% \end{figure}

% \begin{figure}[H]
%   \centering
%   \begin{subfigure}[b]{\textwidth}
%     \includegraphics[width=0.49\textwidth]{figuras/Q1.a_BFGS_P0=[-1 -3].pdf}
%     \includegraphics[width=0.49\textwidth]{figuras/Q1.a_Newton Raphson_P0=[-1 -3].pdf}
%   \end{subfigure}
%   \caption{Curvas de nível e os pontos obtidos pelo BFGS e Newton-Raphson. Questão 1a e $x^0 = \{-1,-3\}^t$}
% \end{figure}

\subsection{Questão 1 (b)}
Considerar $a = 10$ e $b = 1$. Utilizei o site Wolfram Alpha para cálculo do gradiente e da Hessiana.

\vspace{5mm}

\begin{center}
$f(x_1,x_2) = (1 + a -bx_1 - bx_2)^2 + (b + x_1 + ax_2 - bx_1x_2)^2$

\vspace{5mm}
$\overrightarrow{\nabla} f(x_1,x_2) = 
\begin{bmatrix}
  2(-a(bx_2^2 + b - x_2) + b^2x_1(x_2^2 + 1) -2bx_1x_2 + x_1) \\ -2b(2ax_1x_2 + x_1^2 + 1) + 2a(ax_2 + x_1) + 2b^2(x_1^2 +1)x_2
\end{bmatrix}$

\vspace{5mm}

$H_{1\text{x}1}(x_1,x_2) = 2b^2 + 2(1 - bx_2)^2$

$H_{1\text{x}2}(x_1,x_2) = -2b(ax_2 + b(-x_1)x_2 + b + x_1) + 2(1 - bx_2)(a - bx_1) + 2 b^2$

$H_{2\text{x}1}(x_1,x_2) = -2b(ax_2 + b(-x_1)x_2 + b + x_1) + 2(1 - bx_2) (a - bx_1) + 2 b^2$

$H_{2\text{x}2}(x_1,x_2) = 2(a - bx_1)^2 + 2b^2$

\end{center}

\vspace{5mm}

Definição da função, gradiente e Hessiana no código principal :

\begin{python}
  def f(Xn):
    a = 10
    b = 1
    return (1 + a - b*Xn[0] - b*Xn[1])**2 + (b + Xn[0] + a*Xn[1] - b*Xn[0]*Xn[1])**2 
    
  def grad_f(Xn):
      a = 10
      b = 1
      return np.array([2*(-a*(b*(Xn[1]**2) + b - Xn[1]) + (b**2)*Xn[0]*(Xn[1]**2 + 1) - 2*b*Xn[0]*Xn[1] + Xn[0]),
                      -2*b*(2*a*Xn[0]*Xn[1] + Xn[0]**2 + 1) + 2*a*(a*Xn[1] + Xn[0]) + 2*(b**2)*(Xn[0]**2 + 1)*Xn[1]])
  def hessian_f(Xn):
      a = 10
      b = 1
      hessian = np.zeros((2,2))
      hessian[0, 0] = 2*(b**2) + 2*((1 - b*Xn[1])**2)
      hessian[0, 1] = -2*b*(a*Xn[1] + b*(-Xn[0]*Xn[1]) + b + Xn[0]) + 2*(1-b*Xn[1])*(a - b*Xn[0]) + 2*(b**2)
      hessian[1, 0] = -2*b*(a*Xn[1] + b*(-Xn[0]*Xn[1]) + b + Xn[0]) + 2*(1-b*Xn[1])*(a-b*Xn[0]) + 2*(b**2)
      hessian[1, 1] = 2*((a-b*Xn[0])**2) + 2*(b**2)
      return hessian
      
  func = 2
\end{python}

\subsubsection{Ponto inicial : $x^0 = \{10,2\}^t $}

Definição do ponto inicial no código principal:
\begin{python}
  P0 = np.array([10, 2])
\end{python}

A tabela resumo abaixo mostra que para essa função, que não é quadrática, todos os métodos conseguiram convergir dentro
do limite especificado para o número máximo de iterações.

Os métodos de Powell, Fletcher-Reeves e BFGS não convergiram dentro
do número de passos esperado para funções quadráticas, o que era de certa forma esperado.  Newton-Rapshson convergiu em
1 passo, mas foi o único que encontrou um ponto diferente dos demais. Ele convergiu para um ponto de sela imediatamente
abaixo do ponto inicial, enquanto os demais métodos convergiram para o mínimo mais próximo, localizado à direita
do ponto inicial. Esses detalhes são melhores vistos na figura com as curvas de nível localizada mais abaixo neste documento.

\begin{table}[H]
  \begin{center}
    \begin{tabular}{c|c|c|c}
      \textbf{Método} & \textbf{\# Passos} & \textbf{Tempo(s)} & \textbf{$P_{min}$}\\
      \hline
      Univariante & 64 & 0.03673 & $\{13.00000142, 3.99999883\}^t$\\
      Powell & 15 & 0.02147 & $\{13.00000057, 3.99999962\}^t$\\
      Steepest Descent & 55 & 0.00927 & $\{13.00000099, 3.99999874\}^t$\\
      Fletcher-Reeves & 71 & 0.01208 & $\{12.99999937, 4.00000089\}^t$\\
      BFGS & 9 & 0.01836 & $\{13.00000042, 3.99999969\}^t$\\
      Newton-Raphson & 1 & 0.00272 & $\{10, 0.99999967\}^t$\\
    \end{tabular}
  \end{center}
  \caption{Resumo dos resultados obtidos na questão 1b para $x^0 = \{10,2\}^t$}
\end{table}

% \begin{figure}[H]
%   \centering
%   \includegraphics[scale=0.4]{figuras/q1b_passos_P1.PNG}
%   \caption{Número de passos por método OSR. Questão 1b e $x^0 = \{10,2\}^t$ }
% \end{figure}

A figura abaixo deixa claro como Newton-Raphson, apesar de ter convergido em 1 iteração, praticamente não minimizou
a função como os demais métodos, por ter encontrado um ponto de sela. Flecther-Reeves foi o método que mais demorou
para se aproximar do ponto mínimo, e o motivo ficará mais evidente mais abaixo quando eu apresentar o caminho percorrido
por este método.

% \begin{figure}[H]
%   \centering
%   \includegraphics[scale=0.4]{figuras/q1b_fxpassos_P1.PNG}
%   \caption{Gráficos de $f(x_1,x_2)$ versus passo da minimização, por método. Questão 1b e $x^0 = \{10,2\}^t$}
% \end{figure}

% \begin{figure}[H]
%   \centering
%   \begin{subfigure}[b]{\textwidth}
%     \includegraphics[width=0.49\textwidth]{figuras/Q1.b_Univariante_P0=[10e2].pdf}
%     \includegraphics[width=0.49\textwidth]{figuras/Q1.b_Powell_P0=[10e2].pdf}
%   \end{subfigure}
%   \caption{Curvas de nível e os pontos obtidos pelo Univariante e Powell. Questão 1b e $x^0 = \{10,2\}^t$}
% \end{figure}

% \begin{figure}[H]
%   \centering
%   \begin{subfigure}[b]{\textwidth}
%     \includegraphics[width=0.49\textwidth]{figuras/Q1.b_Steepest Descent_P0=[10e2].pdf}
%     \includegraphics[width=0.49\textwidth]{figuras/Q1.b_Fletcher-Reeves_P0=[10e2].pdf}
%   \end{subfigure}
%   \caption{Curvas de nível e os pontos obtidos pelo Steepest Descent e Fletcher-Reeves. Questão 1b e $x^0 = \{10,2\}^t$}
% \end{figure}

% \begin{figure}[H]
%   \centering
%   \begin{subfigure}[b]{\textwidth}
%     \includegraphics[width=0.49\textwidth]{figuras/Q1.b_BFGS_P0=[10e2].pdf}
%     \includegraphics[width=0.49\textwidth]{figuras/Q1.b_Newton Raphson_P0=[10e2].pdf}
%   \end{subfigure}
%   \caption{Curvas de nível e os pontos obtidos pelo BFGS e Newton-Raphson. Questão 1b e $x^0 = \{10,2\}^t$}
% \end{figure}

Um resultado que chama a atenção é o método Fletcher-Reeves ter levado 71 passos para convergir. Alterar a tolerância
de convergência global para $10^{-4}$ e a tolerância da seção áurea, na busca unidirecional, para $10^{-8}$ levou a uma
redução modesta para 60 iterações. Analisando as curvas de nível e o "caminho" de otimização gerado pelo método,
percebe-se que ocorre um movimento em espiral em torno do ponto mínimo. Ou seja, para essa função e ponto inicial,
as direções geradas por Fletcher-Reeves levam a um elevado número de passos.

Nenhum dos métodos levou ao ponto de mínimo à esquerda do ponto incicial.

% \begin{figure}[H]
%   \centering
%   \includegraphics[scale=0.5]{figuras/Q1.b_espiral_Fletcher-Reeves_P0=[10e2].pdf}
%   \caption{Movimento em espiral do método Fletcher-Reeves. Questão 1b e $x^0 = \{10,2\}^t$ }
% \end{figure}

\subsubsection{Ponto inicial : $x^0 = \{-2,-3\}^t $}

Definição do ponto inicial no código principal:
\begin{python}
  P0 = np.array([-2, -3])
\end{python}

Seguem resultados para o segundo ponto inicial proposto no enunciado. Pela posição deste ponto e da proximidade com 
o mínimo mais à esquerda da função, dessa vez, todos os métodos convergiram corretamente para o mínimo esperado.
Em relação aos resultados não há novas observações relevantes a serem feitas, com exceção do fato de todos os métodos
terem conseguido se aproximar muito rápido do mínimo.

\begin{table}[H]
  \begin{center}
    \begin{tabular}{c|c|c|c}
      \textbf{Método} & \textbf{\# Passos} & \textbf{Tempo(s)} & \textbf{$P_{min}$}\\
      \hline
      Univariante       & 61  & 0.04288 & $\{7.00000124, -2.00000132\}^t$\\
      Powell            & 15  & 0.13350 & $\{7.0000002, -1.99999995\}^t$\\
      Steepest Descent  & 45  & 0.01018 & $\{7.000001, -2.0000009\}^t$\\
      Fletcher-Reeves   & 21  & 0.00642 & $\{7.00000007, -2.00000017\}^t$\\
      BFGS              & 8   & 0.04286 & $\{7.00000021, -2.00000023\}^t$\\
      Newton-Raphson    & 6   & 0.01332 & $\{7.00000001, -2.00000001\}^t$\\
    \end{tabular}
  \end{center}
  \caption{Resumo dos resultados obtidos na questão 1b para $x^0 = \{-2,-3\}^t$}
\end{table}

% \begin{figure}[H]
%   \centering
%   \includegraphics[scale=0.4]{figuras/q1b_passos_P2.PNG}
%   \caption{Número de passos por método OSR. Questão 1b e $x^0 = \{-2,-3\}^t$ }
% \end{figure}

% \begin{figure}[H]
%   \centering
%   \includegraphics[scale=0.45]{figuras/q1b_fxpassos_P2.PNG}
%   \caption{Gráficos de $f(x_1,x_2)$ versus passo da minimização, por método. Questão 1b e $x^0 = \{-2,-3\}^t$}
% \end{figure}

% \begin{figure}[H]
%   \centering
%   \begin{subfigure}[b]{\textwidth}
%     \includegraphics[width=0.49\textwidth]{figuras/Q1.b_Univariante_P0=[-2e-3].pdf}
%     \includegraphics[width=0.49\textwidth]{figuras/Q1.b_Powell_P0=[-2e-3].pdf}
%   \end{subfigure}
%   \caption{Curvas de nível e os pontos obtidos pelo Univariante e Powell. Questão 1b e $x^0 = \{-2,-3\}^t$}
% \end{figure}

% \begin{figure}[H]
%   \centering
%   \begin{subfigure}[b]{\textwidth}
%     \includegraphics[width=0.49\textwidth]{figuras/Q1.b_Steepest Descent_P0=[-2e-3].pdf}
%     \includegraphics[width=0.49\textwidth]{figuras/Q1.b_Fletcher-Reeves_P0=[-2e-3].pdf}
%   \end{subfigure}
%   \caption{Curvas de nível e os pontos obtidos pelo Steepest Descent e Fletcher-Reeves. Questão 1b e $x^0 = \{-2,-3\}^t$}
% \end{figure}

% \begin{figure}[H]
%   \centering
%   \begin{subfigure}[b]{\textwidth}
%     \includegraphics[width=0.49\textwidth]{figuras/Q1.b_BFGS_P0=[-2e-3].pdf}
%     \includegraphics[width=0.49\textwidth]{figuras/Q1.b_Newton Raphson_P0=[-2e-3].pdf}
%   \end{subfigure}
%   \caption{Curvas de nível e os pontos obtidos pelo BFGS e Newton-Raphson. Questão 1b e $x^0 = \{-2,-3\}^t$}
% \end{figure}

% \section{Aplicação da Implementação}
% \subsection{Questão 2 (a)}
% \subsubsection{Enunciado}
% Determinar os deslocamentos $(u_A, v_A)$ do ponto $A$, que minimizam a Energia Potencial Total $(\Pi)$ do sitema de molas
% indicado na figura abaixo. Adotar o ponto inicial: $x^0 = \{0.01, -0.10\} ^t$

% \begin{figure}[H]
%   \centering
%   \includegraphics[scale=1]{figuras/enunciado_2a.png}
%   \caption{Sistema 2 molas. Fonte: Questão 2a do Trabalho 1. }
% \end{figure}

\subsubsection{Formulação}
A seguinte formulação foi apresentada em sala de aula :
\vspace{3mm}

$A = (x_A, y_A)$ = Posição inicial do ponto A

\vspace{3mm}
$A' = (x_A + u_A, y_A + v_A)$ = Posição final do ponto A

\vspace{3mm}
$\Pi = U - V$ = Energia Potencial total

\vspace{3mm}
$U = U_{mola1} + U_{mola2}$ = energia interna de deformação

\vspace{3mm}
$V$ = trabalho das forças externas

\vspace{5mm}
$U_i = \frac{1}{2}K_i\Delta L_i^2$

\vspace{3mm}
$L'_1 = \sqrt{(L_1 + u_A)^2 + v_A^2}$ , $L'_2 = \sqrt{(L_2 - u_A)^2 + v_A^2}$ e $W = \frac{1}{2}(\rho_1L_1 + \rho_2L_2)$

\vspace{3mm}
$V = Wv_A$

\vspace{5mm}
$\Pi = \frac{1}{2}\frac{EA_1}{L_1}(\sqrt{(L_1 + x_1)^2 + x_2^2} - L_1)^2 +
        \frac{1}{2}\frac{EA_2}{L_2}(\sqrt{(L_2 - x_1)^2 + x_2^2} - L_2)^2 - 
        (\frac{\rho_1L_1}{2} + \frac{\rho_2L_2}{2})x_2$

\vspace{3mm}
Substituindo os valores do enunciado e usando o site Wolfram ALpha para cálculo do gradiente e Hessiana :

\vspace{3mm}
$\Pi = 450(\sqrt{(30 + x_1)^2 + x_2^2} - 30)^2 +
        300(\sqrt{(30 - x_1)^2 + x_2^2} - 30)^2 - 
        360x_2$

\vspace{3mm}

 $\overrightarrow{\nabla}\Pi =
\begin{bmatrix}
  \frac{900 (x_1 + 30) (\sqrt{(x_1 + 30)^2 + x_2^2} - 30)}{\sqrt{(x_1 + 30)^2 + x_2^2}} - 
  \frac{600 (30 - x_1) (\sqrt{(x_1 - 30)^2 + x_2^2} - 30)}{\sqrt{(x_1 - 30)^2 + x_2^2}} \\[5mm]
  60 (x_2 (\frac{-450}{\sqrt{x_1^2 + 60 x_1 + x_2^2 + 900}} - \frac{300}{\sqrt{x_1^2 - 60 x_1 + x_2^2 + 900}} + 25) - 6)
\end{bmatrix}$

\vspace{5mm}

$H_{1\text{x}1} = -\frac{600 (30 - x_1)^2 (\sqrt{(30 - x_1)^2 + x_2^2} - 30)}{((30 - x_1)^2 + x_2^2)^{3/2}} +
\frac{600 (30 - x_1)^2}{(30 - x_1)^2 + x_2^2} +
\frac{600 (\sqrt{(30 - x_1)^2 + x_2^2} - 30)}{\sqrt{(30 - x_1)^2 + x_2^2}} +
\frac{900 (\sqrt{(x_1 + 30)^2 + x_2^2} - 30)}{\sqrt{(x_1 + 30)^2 + x_2^2}}$

\hspace{3em}
$-\frac{900 (x_1 + 30)^2 (\sqrt{(x_1 + 30)^2 + x_2^2} - 30)}{((x_1 + 30)^2 + x_2^2)^{3/2}} +
\frac{900 (x_1 + 30)^2}{(x_1 + 30)^2 + x_2^2}$

\vspace{5mm}

$H_{1\text{x}2} = \frac{600 (30 - x_1) x_2 (\sqrt{(30 - x_1)^2 + x_2^2} - 30)}{((30 - x_1)^2 + x_2^2)^{3/2}} -
\frac{900 (x_1 + 30) x_2 (\sqrt{(x_1 + 30)^2 + x_2^2} - 30)}{((x_1 + 30)^2 + x_2^2)^{3/2}} -
\frac{600 (30 - x_1) x_2}{(30 - x_1)^2 + x_2^2} +
\frac{900 (x_1 + 30) x_2}{(x_1 + 30)^2 + x_2^2}$

\vspace{5mm}

$H_{2\text{x}1} = \frac{600 (30 - x_1) x_2 (\sqrt{(30 - x_1)^2 + x_2^2} - 30)}{((30 - x_1)^2 + x_2^2)^{3/2}} -
\frac{900 (x_1 + 30) x_2 (\sqrt{(x_1 + 30)^2 + x_2^2} - 30)}{((x_1 + 30)^2 + x_2^2)^{3/2}} -
\frac{600 (30 - x_1) x_2}{(30 - x_1)^2 + x_2^2} +
\frac{900 (x_1 + 30) x_2}{(x_1 + 30)^2 + x_2^2}$

\vspace{5mm}

$H_{2\text{x}2} = -\frac{600 x_2^2 (\sqrt{(30 - x_1)^2 + x_2^2} - 30)}{((30 - x_1)^2 + x_2^2)^{3/2}} -
\frac{900 x_2^2 (\sqrt{(x_1 + 30)^2 + x_2^2} - 30)}{((x_1 + 30)^2 + x_2^2)^{3/2}} +
\frac{600 x_2^2}{(30 - x_1)^2 + x_2^2} +
\frac{900 x_2^2}{(x_1 + 30)^2 + x_2^2} +
\frac{600 (\sqrt{(30 - x_1)^2 + x_2^2} - 30)}{\sqrt{(30 - x_1)^2 + x_2^2}}$

\hspace{3em}
$+\frac{900 (\sqrt{(x_1 + 30)^2 + x_2^2} - 30)}{\sqrt{(x_1 + 30)^2 + x_2^2}}$

\vspace{5mm}
Definição da função, gradiente, Hessiana no código principal :

\begin{python}
  def f(Xn):
    return 450 *((np.sqrt((30 + Xn[0])**2 + Xn[1]**2) - 30 )**2) + 300 *((np.sqrt((30 - Xn[0])**2 + Xn[1]**2) - 30)**2) - 360*Xn[1]

  def grad_f(Xn):
      return np.array([(900*(Xn[0] + 30)*(np.sqrt((Xn[0] + 30)**2 + Xn[1]**2) - 30))/np.sqrt((Xn[0] + 30)**2 + Xn[1]**2) - (600*(30 - Xn[0])*(np.sqrt((Xn[0] - 30)**2 + Xn[1]**2) - 30))/np.sqrt((Xn[0] - 30)**2 + Xn[1]**2),
                      60*(Xn[1]*(-450/np.sqrt(Xn[0]**2 + 60*Xn[0] + Xn[1]**2 + 900) - 300/np.sqrt(Xn[0]**2 - 60*Xn[0] + Xn[1]**2 + 900) + 25) - 6)])
      
  def hessian_f(Xn):
      hessian = np.zeros((2,2))
      hessian[0, 0] = -(600*(30 - Xn[0])**2*(np.sqrt((30 - Xn[0])**2 + Xn[1]**2) - 30))/((30 - Xn[0])**2 + Xn[1]**2)**(3/2) + \
      (600*(30 - Xn[0])**2)/((30 - Xn[0])**2 + Xn[1]**2) + \
      (600*(np.sqrt((30 - Xn[0])**2 + Xn[1]**2) - 30))/np.sqrt((30 - Xn[0])**2 + Xn[1]**2) + \
      (900*(np.sqrt((Xn[0] + 30)**2 + Xn[1]**2) - 30))/np.sqrt((Xn[0] + 30)**2 + Xn[1]**2) - \
      (900*(Xn[0] + 30)**2*(np.sqrt((Xn[0] + 30)**2 + Xn[1]**2) - 30))/((Xn[0] + 30)**2 + Xn[1]**2)**(3/2) + \
      (900*(Xn[0] + 30)**2)/((Xn[0] + 30)**2 + Xn[1]**2)
      
      hessian[0, 1] = (600*(30 - Xn[0])*Xn[1]*(np.sqrt((30 - Xn[0])**2 + Xn[1]**2) - 30))/((30 - Xn[0])**2 + Xn[1]**2)**(3/2) - \
      (900*(Xn[0] + 30)*Xn[1]*(np.sqrt((Xn[0] + 30)**2 + Xn[1]**2) - 30))/((Xn[0] + 30)**2 + Xn[1]**2)**(3/2) - \
      (600*(30 - Xn[0])*Xn[1])/((30 - Xn[0])**2 + Xn[1]**2) + \
      (900*(Xn[0] + 30)*Xn[1])/((Xn[0] + 30)**2 + Xn[1]**2)
      
      hessian[1, 0] = (600*(30 - Xn[0])*Xn[1]*(np.sqrt((30 - Xn[0])**2 + Xn[1]**2) - 30))/((30 - Xn[0])**2 + Xn[1]**2)**(3/2) - \
      (900*(Xn[0] + 30)*Xn[1]*(np.sqrt((Xn[0] + 30)**2 + Xn[1]**2) - 30))/((Xn[0] + 30)**2 + Xn[1]**2)**(3/2) - \
      (600*(30 - Xn[0])*Xn[1])/((30 - Xn[0])**2 + Xn[1]**2) + \
      (900*(Xn[0] + 30)*Xn[1])/((Xn[0] + 30)**2 + Xn[1]**2)
      
      hessian[1, 1] = -(600*Xn[1]**2*(np.sqrt((30 - Xn[0])**2 + Xn[1]**2) - 30))/((30 - Xn[0])**2 + Xn[1]**2)**(3/2) - \
      (900*Xn[1]**2*(np.sqrt((Xn[0] + 30)**2 + Xn[1]**2) - 30))/((Xn[0] + 30)**2 + Xn[1]**2)**(3/2) + \
      (600*Xn[1]**2)/((30 - Xn[0])**2 + Xn[1]**2) + \
      (900*Xn[1]**2)/((Xn[0] + 30)**2 + Xn[1]**2) + \
      (600*(np.sqrt((30 - Xn[0])**2 + Xn[1]**2) - 30))/np.sqrt((30 - Xn[0])**2 + Xn[1]**2) + \
      (900*(np.sqrt((Xn[0] + 30)**2 + Xn[1]**2) - 30))/np.sqrt((Xn[0] + 30)**2 + Xn[1]**2)
      
      return hessian

  func = 3
\end{python}

\subsubsection{Resultados}

Definição do ponto inicial no código principal:
\begin{python}
  P0 = np.array([0.01, -0.1])
\end{python}

\textbf{Utilizei o seguinte controle numérico para resolução da questão 2(a):}
\begin{itemize}
  \item Número máximo de passos (ou iterações): 200
  \item Tolerância para convergência do gradiente: Sensibilidade com $10^{-5}$, $10^{-4}$ e $10^{-3}$
  \item Tolerância para convergência da busca unidirecional: $10^{-6}$
  \item $\Delta\alpha$ do passo constante: $10^{-2}$
\end{itemize}

\vspace{3mm}
\textbf{Principais resultados obtidos:}

\begin{table}[H]
  \begin{center}
    \begin{tabular}{c|c|c|c}
      \multicolumn{4}{c}{tol = $10^{-5}$}\\
      \hline
      \textbf{Método} & \textbf{\# Passos} & \textbf{Tempo(s)} & \textbf{$P_{min}$}\\
      \hline
      Univariante       & 200   & 0.07850 & $\{-0.20510911, 7.78899302\}^t$\\
      Powell            & 200   & 91.21617 & $\{-0.20510878, 7.78899254\}^t$\\
      Steepest Descent  & 20    & 0.00304 & $\{-0.20510889, 7.78899266\}^t$\\
      Fletcher-Reeves   & 200   & 0.02659 & $\{-0.20510878, 7.78899218\}^t$\\
      BFGS              & 200   & 0.03755 & $\{-0.20510893, 7.78899288\}^t$\\
      Newton-Raphson    & 200   & 0.03705 & $\{-0.2051089, 7.78899288\}^t$\\
    \end{tabular}
  \end{center}
  \caption{Resumo dos resultados obtidos na questão 2a para $x^0 = \{0.01,-0.1\}^t$ e tol = $10^{-5}$}
\end{table}

\begin{table}[H]
  \begin{center}
    \begin{tabular}{c|c|c|c}
      \multicolumn{4}{c}{tol = $10^{-4}$}\\
      \hline
      \textbf{Método} & \textbf{\# Passos} & \textbf{Tempo(s)} & \textbf{$P_{min}$}\\
      \hline
      Univariante       & 200   & 0.05557 & $\{-0.20510911, 7.78899302\}^t$\\
      Powell            & 9     & 92.83198 & $\{-0.20510884, 7.78899251\}^t$\\
      Steepest Descent  & 6     & 0.00090 & $\{-0.20510891, 7.78899276\}^t$\\
      Fletcher-Reeves   & 12    & 0.00123 & $\{-0.20510891, 7.78899332\}^t$\\
      BFGS              & 4     & 0.00324 & $\{-0.2051089,  7.78899268\}^t$\\
      Newton-Raphson    & 49    & 0.12737 & $\{-0.20510894, 7.78899296\}^t$\\
    \end{tabular}
  \end{center}
  \caption{Resumo dos resultados obtidos na questão 2a para $x^0 = \{0.01,-0.1\}^t$ e tol = $10^{-4}$}
\end{table}

\begin{table}[H]
  \begin{center}
    \begin{tabular}{c|c|c|c}
      \multicolumn{4}{c}{tol = $10^{-3}$}\\
      \hline
      \textbf{Método} & \textbf{\# Passos} & \textbf{Tempo(s)} & \textbf{$P_{min}$}\\
      \hline
      Univariante       & 9     & 0.04030 & $\{-0.20510877, 7.78899022\}^t$\\
      Powell            & 8     & 91.89477 & $\{-0.20510883, 7.78899446\}^t$\\
      Steepest Descent  & 5     & 0.00073 & $\{-0.20510863, 7.78899279\}^t$\\
      Fletcher-Reeves   & 10    & 0.00255 & $\{-0.2051088, 7.78899188\}^t$\\
      BFGS              & 4     & 0.00500 & $\{-0.20510894, 7.78899296\}^t$\\
      Newton-Raphson    & 3     & 0.00425 & $\{-0.20510893, 7.78899416\}^t$\\
    \end{tabular}
  \end{center}
  \caption{Resumo dos resultados obtidos na questão 2a para $x^0 = \{0.01,-0.1\}^t$ e tol = $10^{-3}$}
\end{table}

Pelas tabelas apresentadas acima, um aumento da tolerância da convergência do gradiente leva a uma redução drástica
do número de iterações e, aparentemente, sem grandes perdas de precisão no valor de ponto mínimo obtido. Para a 
tolerância de $10^{-5}$ apenas o método Steepest Descent convergiu, levando um total de 20 passos, enquanto os 
demais métodos usaram o máximo especificado de 200 iterações. Para uma tolerância de $10^{-4}$ apenas o método univariante
não convergiu em menos de 200 passos, enquanto que para uma tolerância de $10^{-3}$ todos os métodos convergiram em até 10
passos e com resultados satisfatórios.

Em termos de tempo de execução, o método de Powell é o que mais se destaca. Independente da tolerância, o mesmo leva por volta
de 90 segundos para convergir. Após depuração do código e dos resultados ficou claro que o motivo disso ocorrer é que, 
para essa função e ponto inicial, o método de Powell, no sexto passo global, terceiro do segundo ciclo, gera uma direção com módulo muito pequeno e que
que necessita de um $\alpha$(aproximadante 60000) muito grande para alcançar o mínimo. Isso faz com que o algoritmo leve o tempo apresentado.
Essa talvez seja uma grande indicação de que trabalhar com direções normalizadas na busca unidirecional
seja mais adequado, evitando esse tipo de problema que pode ocorrer com outros métodos também a depender da combimação 
dos parâmetros de entrada. Alterar o ponto inicial ajuda a resolver esse problema encontrado no método de Powell.

Para fins de apresentação dos demais resultados, irei utilizar como base o estudo feito com tolerância de $10^{-3}$.

% \begin{figure}[H]
%   \centering
%   \includegraphics[scale=0.4]{figuras/q2a_passos_P1.PNG}
%   \caption{Número de passos por método OSR. Questão 2a e $x^0 = \{0.01,-0.1\}^t$}
% \end{figure}

% \begin{figure}[H]
%   \centering
%   \includegraphics[scale=0.45]{figuras/q2a_fxpassos_P1.PNG}
%   \caption{Gráficos de $f(x_1,x_2)$ versus passo da minimização, por método. Questão 2a com $x^0 = \{0.01,-0.1\}^t$ e tol = $10^{-3}$}
% \end{figure}

% Seguem, abaixo, curvas de nível e caminhos obtidos pelos métodos para $x^0 = \{0.01,-0.1\}^t$ e tol = $10^{-3}$. 

% \begin{figure}[H]
%   \centering
%   \begin{subfigure}[b]{\textwidth}
%     \includegraphics[width=0.49\textwidth]{figuras/Q2.a_Univariante_P0=[0.01e-0.1].pdf}
%     \includegraphics[width=0.49\textwidth]{figuras/Q2.a_Powell_P0=[0.01e-0.1].pdf}
%   \end{subfigure}
%   \caption{Curvas de nível e os pontos obtidos pelo Univariante e Powell. Questão 2a com $x^0 = \{0.01,-0.1\}^t$ e tol = $10^{-3}$}
% \end{figure}

% \begin{figure}[H]
%   \centering
%   \begin{subfigure}[b]{\textwidth}
%     \includegraphics[width=0.49\textwidth]{figuras/Q2.a_Steepest Descent_P0=[0.01e-0.1].pdf}
%     \includegraphics[width=0.49\textwidth]{figuras/Q2.a_Fletcher-Reeves_P0=[0.01e-0.1].pdf}
%   \end{subfigure}
%   \caption{Curvas de nível e os pontos obtidos pelo Steepest Descent e Fletcher-Reeves. Questão 2a com $x^0 = \{0.01,-0.1\}^t$ e tol = $10^{-3}$}
% \end{figure}

% \begin{figure}[H]
%   \centering
%   \begin{subfigure}[b]{\textwidth}
%     \includegraphics[width=0.49\textwidth]{figuras/Q2.a_BFGS_P0=[0.01e-0.1].pdf}
%     \includegraphics[width=0.49\textwidth]{figuras/Q2.a_Newton Raphson_P0=[0.01e-0.1].pdf}
%   \end{subfigure}
%   \caption{Curvas de nível e os pontos obtidos pelo BFGS e Newton-Raphson. Questão 2a com $x^0 = \{0.01,-0.1\}^t$ e tol = $10^{-3}$}
% \end{figure}

% Abaixo seguem as curvas de nível para um ponto inicial diferente do proposto no enuncado. $x^0 = \{-2,10\}^t$

% \begin{figure}[H]
%   \centering
%   \begin{subfigure}[b]{\textwidth}
%     \includegraphics[width=0.49\textwidth]{figuras/Q2.a_Univariante_P0=[-2 10].pdf}
%     \includegraphics[width=0.49\textwidth]{figuras/Q2.a_Powell_P0=[-2 10].pdf}
%   \end{subfigure}
%   \caption{Curvas de nível e os pontos obtidos pelo Univariante e Powell. Questão 2a e $x^0 = \{-2,10\}^t$ e tol = $10^{-3}$}
% \end{figure}

% \begin{figure}[H]
%   \centering
%   \begin{subfigure}[b]{\textwidth}
%     \includegraphics[width=0.49\textwidth]{figuras/Q2.a_Steepest Descent_P0=[-2 10].pdf}
%     \includegraphics[width=0.49\textwidth]{figuras/Q2.a_Fletcher-Reeves_P0=[-2 10].pdf}
%   \end{subfigure}
%   \caption{Curvas de nível e os pontos obtidos pelo Steepest Descent e Fletcher-Reeves. Questão 2a e $x^0 = \{-2,10\}^t$ e tol = $10^{-3}$}
% \end{figure}

% \begin{figure}[H]
%   \centering
%   \begin{subfigure}[b]{\textwidth}
%     \includegraphics[width=0.49\textwidth]{figuras/Q2.a_BFGS_P0=[-2 10].pdf}
%     \includegraphics[width=0.49\textwidth]{figuras/Q2.a_Newton Raphson_P0=[-2 10].pdf}
%   \end{subfigure}
%   \caption{Curvas de nível e os pontos obtidos pelo BFGS e Newton-Raphson. Questão 2a e $x^0 = \{-2,10\}^t$ e tol = $10^{-3}$}
% \end{figure}

\subsection{Questão 2 (b)}
\subsubsection{Enunciado}
Desenvolver um estudo de convergência da solução do deslocamento do ponto A, do sistema de molas, para níveis crescentes
de discretização do modelo (ou seja, considerando o número de molas n = 2,4,6...). A rigidez de cada mola ($k_i$, i =1,...n)
é obtida como a razão entre o módulo de rigidez axial do material e o seu comprimento. Os valores $W_j$ (com j = 1,...n-1)
correspondem às cargas nodais equivalentes aos pesos das molas.

\subsubsection{Formulação}

Para a generalização do problema do sistema de molas, foi considerado que sempre existirá um número par de molas.

$n_{nodes} = n_{molas} -1$

Como cada nó possui duas variáveis (Deslocamento horizontal $(u_i)$ e Deslocamento vertical $(v_i)$), o número de dimensões
será sempre $n_{dimens} = 2n_{nodes}$.

Usando a mesma ideia do equacionamento dos comprimentos finais de cada mola para o sistema com apenas 2 molas, e
extrapolando para $n_{dimens}$, podemos escrever o seguinte :

\vspace{3mm}
Seja $Li_m = $ comprimeto inicial da mola m, com m=1,2,...$n_{molas}$

\vspace{3mm}
Seja $Lf_m = $ comprimeto final da mola m, com m=1,2,...$n_{molas}$

\vspace{3mm}
Apenas para facilidade do equacionamento e analogia com a formulação para 2 molas, 
considerei que cada nó livre (total $n_{molas} - 1$ nós livres) terá deslocamentos positivos tanto em $x$ quanto em $y$. Ou seja,
cada nó, após equilíbrio, estará à direita do seu ponto inicial e abaixo (considerando eixo $y$ positivo para baixo) do nó livre anterior.

\vspace{3mm}
Então:

\vspace{3mm}
$Lf_m = \sqrt{(Li_m + u_m - u_{m-1})^2 + (v_m - v_{m-1})^2}$, com $u_0 = v_0 = u_{n_{molas}} = v_{n_{molas}} = 0 $ e m=1,2,...$n_{molas}$

\vspace{3mm}
$U_m = \frac{1}{2}K_m\Delta L_m^2$, sendo $K_m = \frac{EA_m}{Li_m}$ e $\Delta L_m = Lf_m - Li_m$ = Energia Interna de deformação

\vspace{3mm}
$W_n = \frac{1}{2}(\rho_n Li_n + \rho_{n+1} Li_{n+1})$, com n=1,2,...$n_{nodes}$

\vspace{3mm}
$V_n = W_nv_n$ = Trabalho das forças externas

\vspace{3mm}
Finalmente :

\vspace{3mm}
$\Pi = \sum_{m=1}^{n_{molas}}{U_m} - \sum_{n=1}^{n_{nodes}}{V_n}$ = Energia Potencial Total

\vspace{3mm}
Para o cálculo do $\overrightarrow{\nabla} \Pi$, o seguinte raciocínio foi adotado :

A variável $u_n$ aparece apenas nos termos $U_n$ e $U_{n+1}$, enquanto $v_n$ aparece em $U_n$, $U_{n+1}$ e $V_n$.

Dessa forma $\frac{\partial \Pi}{\partial u_n} = \frac{\partial U_n}{\partial u_n} + \frac{\partial U_{n+1}}{\partial u_n}$ e 
$\frac{\partial \Pi}{\partial v_n} = \frac{\partial U_n}{\partial v_n} + \frac{\partial U_{n+1}}{\partial v_n} -
\frac{\partial V_n}{\partial v_n} $. Todos esses termos são possíveis de se calcular analiticamente usando as expressões
apresentadas acima para $Lf_m$, $U_m$ e $V_n$, e com isso conseguimos uma expressão para cálculo do gradiente da função e que
foi implementada no meu código.

O mesmo raciocínio poderia ser aplicado para cálculo da Hessiana da função, porém, como sua utilidade fica restrita ao
método de Newton-Raphson, para fins desse trabalho usei um pacote pronto para cálculo diferencial de forma numérica no Python 
(numdifftools) com resultados bem satisfatórios e coincidentes com todos os cálculos dos métodos
analíticos de gradiente e Hessiana implementados para as funções da questão 1 e questão 2a.

Na implementação, o parâmetro que controla o número de molas do sistema é o número de dimensões do ponto inicial. Ou seja,
para um sistema 2 molas, 1 nó livre, é necessário fornecer $x^0$ com 2 dimensões ($x^0 = \{u_1, v_1\}^t$). Para um
sistema 4 molas, 3 nós livres, é necessar informar $x^0 = \{u_1, v_1, u_2, v_2, u_3, v_3\}^t$, e assim por diante.

\vspace{5mm}
Definição da função, gradiente, Hessiana no código principal :

\begin{python}
  def f(Xn):
    dimens = Xn.size
    #numero de nos
    n = int(dimens/2)

    #numero de molas
    m = n + 1

    #Inicializacao dos vetores com as variaveis do problema
    Li = np.zeros(m, dtype=float) # comprimentos iniciais das molas
    EA = np.zeros(m, dtype=float)
    RHO = np.zeros(m, dtype=float)
    W = np.zeros(n, dtype=float) # peso em cada no
    
    #Atribuicao dos valores do problema
    #Cada mola mede inicialmente 60/n_molas
    #molas a esquerda possuem EA = 27000 e rho 8
    #molas a direita possuem EA = 18000 e rho 16
    Li = Li + 60/m

    EA[ : int(m/2)] = 27000
    EA[int(m/2) : ] = 18000
    RHO[ : int(m/2)] = 8
    RHO[int(m/2) : ] = 16
    
    #Calculo dos pesos atuando em cada no
    # W[j] = (1/2)*(RHO[j]*Li[j] + RHO[j+1]*Li[j+1])
    RHO_e = RHO[:m-1]
    RHO_d = RHO[1:m]
    Li_e = Li[:m-1]
    Li_d = Li[1:m]
    W = (1/2)*(RHO_e*Li_e + RHO_d*Li_d)
    
    Lf = np.zeros(m, dtype=float) # comprimentos finais das molas
    U = np.zeros(m, dtype=float) # energia elastica das molas0.01, -0.1    
    V = np.zeros(n, dtype=float) # trabalho em cada no (desloc vert)  
      
    #array com os deslocamentos horizontais do Xn
    dx = Xn[0::2].copy()
    #array com os deslocamentos verticais do Xn
    dy = Xn[1::2].copy()
       
    #Calculo dos comprimentos finais
    # Lf[0] = np.sqrt( (Li[0] + dx[0] )**2 + dy[0]**2 ) 
    # Lf[k] = np.sqrt(a**2 + b**2)
    # Lf[m-1] = np.sqrt((Li[m-1] - dx[n-1])**2 + dy[n-1]**2)
    dx_d = np.zeros(m, dtype= float)
    dx_d[1:] = dx.copy()    
    dx_e = np.zeros(m, dtype=float)
    dx_e[:m-1] = dx.copy()
    dy_d = np.zeros(m, dtype= float)
    dy_d[1:] = dy.copy()
    dy_e = np.zeros(m, dtype=float)
    dy_e[:m-1] = dy.copy()
    
    a = Li + dx_e - dx_d
    b = dy_e - dy_d
    
    Lf = np.sqrt(a**2 + b**2)
    
    #calculo da energia elastica em cada mola
    U = (1/2)*(EA/Li)*((Lf - Li)**2)
    
    #calculo do trabalho em cada no    
    V = W*dy
    
    #Calculo da Energia Total
    E = np.sum(U) - np.sum(V)
    
    return E

  def grad_f(Xn):
      dimens = Xn.size
      #numero de nos
      n = int(dimens/2)

      #numero de molas
      m = n + 1

      #Inicializacao dos vetores com as variaveis do problema
      Li = np.zeros(m, dtype=float) # comprimentos iniciais das molas
      EA = np.zeros(m, dtype=float)
      RHO = np.zeros(m, dtype=float)
      W = np.zeros(n, dtype=float) # peso em cada no
      
      #Atribuicao dos valores do problema
      #Cada mola mede inicialmente 60 sobre numero de molas
      #molas a esquerda possuem EA = 27000 e rho 8
      #molas a direita possuem EA = 18000 e rho 16
      Li = Li + 60/m

      EA[ : int(m/2)] = 27000
      EA[int(m/2) : ] = 18000
      RHO[ : int(m/2)] = 8
      RHO[int(m/2) : ] = 16
      
      #Calculo dos pesos atuando em cada no
      # W[j] = (1/2)*(RHO[j]*Li[j] + RHO[j+1]*Li[j+1])
      RHO_e = RHO[:m-1]
      RHO_d = RHO[1:m]
      Li_e = Li[:m-1]
      Li_d = Li[1:m]
      W = (1/2)*(RHO_e*Li_e + RHO_d*Li_d)

      Lf = np.zeros(m, dtype=float) # comprimentos finais das molas
          
      #array com os deslocamentos horizontais do Xn
      dx = Xn[0::2].copy()
      #array com os deslocamentos verticais do Xn
      dy = Xn[1::2].copy()
      
      #Calculo dos comprimentos finais
      # Lf[0] = np.sqrt( (Li[0] + dx[0] )**2 + dy[0]**2 ) 
      # Lf[k] = np.sqrt(a**2 + b**2)
      # Lf[m-1] = np.sqrt((Li[m-1] - dx[n-1])**2 + dy[n-1]**2)
      dx_d = np.zeros(m, dtype= float)
      dx_d[1:] = dx.copy()
      
      dx_e = np.zeros(m, dtype=float)
      dx_e[:m-1] = dx.copy()
      dy_d = np.zeros(m, dtype= float)
      dy_d[1:] = dy.copy()
      dy_e = np.zeros(m, dtype=float)
      dy_e[:m-1] = dy.copy()
      
      a = Li + dx_e - dx_d
      b = dy_e - dy_d
      
      Lf = np.sqrt(a**2 + b**2)
      
      #calculo gradiente
      gradx = np.zeros(n, dtype=float)
      grady = np.zeros(n, dtype=float)
      Li_e = np.zeros(n, dtype=float)
      Li_d = np.zeros(n, dtype=float)
      EA_e = np.zeros(n, dtype=float)
      EA_d = np.zeros(n, dtype=float)
      Lf_e = np.zeros(n, dtype=float)
      Lf_d = np.zeros(n, dtype=float)
      
      Li_e = Li[:m-1]
      Li_d = Li[1:m]
      EA_e = EA[:m-1]
      EA_d = EA[1:m]
      Lf_e = Lf[:m-1]
      Lf_d = Lf[1:m]
      
      dx_d = np.zeros(n, dtype=float)   
      dx_d[1:] = dx[:m-2]
      
      dy_d = np.zeros(n, dtype=float)  
      dy_d[1:] = dy[:m-2]
      
      dx_e = np.zeros(n, dtype=float)
      dx_e[:m-2] = dx[1:]
      
      dy_e = np.zeros(n, dtype=float)
      dy_e[:m-2] = dy[1:]
      
      deriv1 = np.zeros(n, dtype=float)
      deriv2 = np.zeros(n, dtype=float)
      deriv3 = np.zeros(n, dtype=float)
      deriv4 = np.zeros(n, dtype=float)
      deriv5 = np.zeros(n, dtype=float)
        
      deriv1 = (1/2)*((Li_e + dx - dx_d)**2 + (dy - dy_d)**2)**(-1/2)*(2*Li_e + 2*dx - 2*dx_d)        
      deriv2 = (1/2)*((Li_d + dx_e - dx)**2 + (dy_e - dy)**2)**(-1/2)*(-2*Li_d - 2*dx_e + 2*dx)    
      deriv3 = (1/2)*((Li_e + dx - dx_d)**2 + (dy - dy_d)**2)**(-1/2)*(2*dy - 2*dy_d)
      deriv4 = (1/2)*((Li_d + dx_e - dx)**2 + (dy_e - dy)**2)**(-1/2)*(-2*dy_e + 2*dy)
      deriv5 = W    
      
      # dUk/dxk + dU(k+1)/dxk
      gradx = (1/2)*(EA_e/Li_e)*2*(Lf_e - Li_e)*deriv1 + (1/2)*(EA_d/Li_d)*2*(Lf_d - Li_d)*deriv2
      
      #dUk/dyk + dU(k+1)/dyk - dVk/dyk
      grady = (1/2)*(EA_e/Li_e)*2*(Lf_e - Li_e)*deriv3 + (1/2)*(EA_d/Li_d)*2*(Lf_d - Li_d)*deriv4 - deriv5
      
      grad = np.zeros(2*n, dtype=float)
      grad[0::2] = gradx
      grad[1::2] = grady
          
      return grad

  def hessian_f(Xn):
      return nd.Hessian(f)(Xn)

  func = 4

\end{python}


\subsubsection{Resultados}

Para esse exercício, variei o número de molas no sistema de 2 a 20, e com isso precisei de
diferentes pontos iniciais para as rodadas dos métodos. Por simplificação, usei os mesmos delocamentos iniciais para 
todos os nós livres. Valores utilizados : $u_i = -1$ e $v_i = 5$, $\forall i$ com $i=1,2,...n_{nodes}$ 

Exemplo de definição do ponto inicial com a premissa acima para o caso 4 molas:

\begin{python}
  P0 = np.array([-1, 5, -1, 5, -1, 5])
\end{python}

\textbf{Utilizei o seguinte controle numérico para resolução da questão 2(b):}
\begin{itemize}
  \item Número máximo de passos (ou iterações): 200
  \item Tolerância para convergência do gradiente: $10^{-3}$
  \item Tolerância para convergência da busca unidirecional: $10^{-6}$
  \item $\Delta\alpha$ do passo constante: $10^{-2}$
\end{itemize}

O gráfico abaixo mostra o número de passos por método e por discretizaçao do número de molas. Nele é possível
constatar que os métodos Univariante e Steepest Descent já não conseguem convergir para um sistema de 4 molas em diante.
O método de Powell até converge para 4 molas, mas a partir de 6 molas já não consegue mais convergir. O método Fletcher-Reeves,
na sensibilidade feita nesse trabalho comvergiu até 10 molas, e no caso com 20 molas não convergiu, apesar de ter chegado
bem próximo dos resultados dos métodos que convergiram. Os métodos BFGS e
Newton-Raphson convergiram para todos os casos, com um número relativamente baixo de passos.

% \begin{figure}[H]
%   \centering
%   \includegraphics[scale=0.35]{figuras/q2b_passos.png}
%   \caption{Número de passos, por método OSR, para diferentes números de molas. Questão 2b.}
% \end{figure}

As tabelas abaixo resumem o estudo de convergência do valor para o deslocamento final do Ponto $A$, representado pelo
vetor $\{u_A, v_a\}^t$. Pode-se notar que os métodos que não convergem, apresentam resultados divergentes dos que convergiram,
como esperado. Também é possível notar a diferença entre posição prevista para o ponto $A$ caso a discretização seja muito
pequena, 2 molas por exemplo, e caso seja muito alta, 20 molas por exemplo, lembrando sempre de olhar o resultado do BFGS
e Newton-Raphson para essa comparação, dado que foram os métodos que atingiram convergência para todas as discretizações.

\begin{table}[H]
  \begin{center}
    \begin{tabular}{c|c|c|c}
      \multicolumn{4}{c}{Deslocamento do Ponto $A$ (nó central)}\\
      \hline
      \textbf{Método} & \textbf{\# 2 molas} & \textbf{4 molas} & \textbf{6 molas} \\
      \hline
      Univariante       & $\{-0.2051, 7.7890\}^t$ &	$\{-0.0867, 7.1702\}^t$ &	$\{-0.1927, 7.1189\}^t$ \\
      Powell	          & $\{-0.2051, 7.7890\}^t$ &	$\{-0.0863, 7.1700\}^t$ &	$\{-0.3735, 6.8136\}^t$ \\
      Steepest Descent  & $\{-0.2051, 7.7890\}^t$ &	$\{-0.0860, 7.1676\}^t$ &	$\{-0.0725, 6.9551\}^t$ \\
      Fletcher-Reeves   & $\{-0.2051, 7.7890\}^t$ &	$\{-0.0863, 7.1700\}^t$ &	$\{-0.0692, 7.0765\}^t$ \\
      BFGS	            & $\{-0.2051, 7.7890\}^t$ &	$\{-0.0863, 7.1700\}^t$ &	$\{-0.0692, 7.0765\}^t$ \\
      Newton-Raphson    & $\{-0.2051, 7.7890\}^t$ &	$\{-0.0863, 7.1700\}^t$ &	$\{-0.0692, 7.0765\}^t$
    \end{tabular}
  \end{center}
  \caption{Resumo dos resultados obtidos na questão 2b}
\end{table}

\begin{table}[H]
  \begin{center}
    \begin{tabular}{c|c|c|c}
      \multicolumn{4}{c}{Deslocamento do Ponto $A$ (nó central)}\\
      \hline
      \textbf{Método}   & \textbf{8 molas} & \textbf{10 molas} & \textbf{20 molas} \\
      \hline
      Univariante       & $\{-0.4678, 7.1400\}^t$	& $\{-0.6909, 7.2969\}^t$ &	$\{-1.0657, 6.7105\}^t$ \\
      Powell	          & $\{-0.5929, 7.5628\}^t$	& $\{-0.7793, 6.6867\}^t$ &	$\{-1.0733, 6.7864\}^t$ \\
      Steepest Descent  & $\{-0.1084, 6.6119\}^t$	& $\{-0.1652, 6.2609\}^t$ &	$\{-0.5561, 5.4032\}^t$ \\
      Fletcher-Reeves   & $\{-0.0635, 7.0450\}^t$	& $\{-0.0610, 7.0306\}^t$ &	$\{-0.0575, 7.0123\}^t$ \\
      BFGS	            & $\{-0.0635, 7.0450\}^t$	& $\{-0.0610, 7.0306\}^t$ &	$\{-0.0575, 7.0116\}^t$ \\
      Newton-Raphson    & $\{-0.0635, 7.0450\}^t$	& $\{-0.0610, 7.0306\}^t$ &	$\{-0.0575, 7.0116\}^t$
    \end{tabular}
  \end{center}
  \caption{Resumo dos resultados obtidos na questão 2b}
\end{table}

A figura abaixo resume o tempo de execução de cada método para cada discretização do número de molas.
Importante destacar o tempo maior para os métodos de Powell e Newton-Raphson. O tempo do método de Newton-Raphson 
acredito que possa ter influência do uso do pacote numdifftools para cálculo da Hessiana, e o tempo para cálculo de sua inversa.
Porém, o método de Powell sofreu novamente, principalmente no caso 4 molas, com uma direção com módulo pequeno e
consequente $\alpha$ grande do passo constante, \textbf{reforçando novamente a ideia de se normalizar as direções na busca unidirecional
 para evitar problemas desse tipo.}

% \begin{figure}[H]
%   \centering
%   \includegraphics[scale=0.4]{figuras/q2b_tempo.png}
%   \caption{Tempo de Execução, por método OSR, para diferentes números de molas. Questão 2b.}
% \end{figure}

A figura abaixo mostra o comportamento do valor da função a cada iteração dos métodos para o caso
com 20 molas, evidenciando o quão rápido Flecther-Reeves,
BFGS e Newton-Raphson se aproximam do mínimo.  

Um detalhe que também aparece, mas está sutil e pode passar despercebido dada a escala em $y$ utilizada,
é que o primeiro passo do Newton-Raphson está aumentando consideravelmente o valor da função. Após investigação do resultado,
notei que esse primeiro passo está usando uma direção com módulo muito grande, da ordem de $5.10^9$ e retornando um 
$\alpha$ da seção áurea da ordem de $3.10^{-7}$. Como o passo usado é $10^{-2}$, é possível constatar que o $\Delta \alpha$
absoluto, ao invés de ficar próximo ao ${10^{-2}}$ desejado, atinge valores acima de $1$, dadas as ordens de grandeza dos
parâmetros. Pode ser que o problema seja o cálculo da Hessiana nesse primeiro passo, apesar de que para os demais passos,
o Newton-Raphson se comporta adequadamente. \textbf{Esse exemplo é mais um reforço para que seja usado direções normalizadas.}

% \begin{figure}[H]
%   \centering
%   \includegraphics[scale=0.45]{figuras/q2b_fxpassos_20molas.PNG}
%   \caption{Gráficos de $f(x_1,x_2)$ versus passo da minimização, por método. Questão 2b com 20 molas}
% \end{figure}

% \begin{figure}[H]
%   \centering
%   \begin{subfigure}[b]{\textwidth}
%     \includegraphics[width=0.49\textwidth]{figuras/q2bUnivGraf.png}
%     \includegraphics[width=0.49\textwidth]{figuras/q2bPowellGraf.png}
%   \end{subfigure}
%   \begin{subfigure}[b]{\textwidth}
%     \includegraphics[width=0.49\textwidth]{figuras/q2bSteepDGraf.png}
%     \includegraphics[width=0.49\textwidth]{figuras/q2bFRGraf.png}
%   \end{subfigure}
%   \begin{subfigure}[b]{\textwidth}
%     \includegraphics[width=0.49\textwidth]{figuras/q2bBFGSGraf.png}
%     \includegraphics[width=0.49\textwidth]{figuras/q2bNRGraf.png}
%   \end{subfigure}
%   \caption{Posição dos nós, por método, para diferentes números de molas}
% \end{figure}

\end{document}