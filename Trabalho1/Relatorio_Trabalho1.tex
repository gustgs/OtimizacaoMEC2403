\documentclass[10pt, a4paper]{article}
% \usepackage[english]{babel}
\usepackage[brazilian]{babel}
\usepackage[utf8]{inputenc}
% \usepackage[T1]{fontenc}
\usepackage{lipsum}

% matlab code
% \usepackage{matlab-prettifier}
%\usepackage[numbered,framed]{matlab-prettifier}
\usepackage{pythonhighlight}
\renewcommand{\lstlistingname}{Anexo} % Listing->Code
\let\ph\mlplaceholder % shorter macro
\definecolor{codegreen}{rgb}{0,0.6,0}
\definecolor{codegray}{rgb}{0.5,0.5,0.5}
\definecolor{codepurple}{rgb}{0.58,0,0.82}
\definecolor{backcolour}{rgb}{0.95,0.95,0.92}
\lstdefinestyle{myStyle}{
    language=Matlab,
    breaklines=true,
    frame=single,
    numbers=none,
    basicstyle=\ttfamily\footnotesize,
%     basicstyle=\footnotesize\ttfamily,
    keywordstyle=\bfseries\color{magenta},
    commentstyle=\color{codegreen},
    identifierstyle=\color{blue},
    backgroundcolor=\color{backcolour},
    stringstyle=\color{codepurple},
}
\usepackage{adjustbox}

% For subfigure use
\usepackage[font=small,labelfont=bf]{caption}
\usepackage{subcaption}

% Set page size and margins
% Replace `letterpaper' with`a4paper' for UK/EU standard size
\usepackage[a4paper,top=2cm,bottom=2cm,left=2cm,right=2cm,marginparwidth=2cm]{geometry}

% tabelas
\usepackage{array}
\usepackage{tabularx}
\usepackage{booktabs}

\usepackage{float}

% Useful packages
\usepackage{amsmath}

\usepackage{graphicx}
%\graphicspath{{figures/}} %Setting the graphicspath
\usepackage[colorlinks=true, allcolors=blue]{hyperref}
\usepackage{cleveref}
\newcommand{\crefrangeconjunction}{--}
\DeclareMathOperator{\sen}{sen}


\begin{document}

\def\TITLE{Trabalho 1}
\def\DISCIPLINE{MEC 2403 - Otimização, Algoritmos e Aplicações na Engenharia Mecânica}
\def\PROFESSOR{Ivan Menezes}
\def\AUTHOR{Gustavo Henrique Gomes dos Santos}
\def\CONTACT{gustavohgs@gmail.com}
\def\DATE{maio de 2023}

\title{\textbf{\TITLE} \\ \DISCIPLINE}
\author{\AUTHOR}
\date{\DATE}

\begin{titlepage}
      \begin{center}
          \vspace*{1cm}

          \Huge
          \textbf{\TITLE}

          \vspace{0.5cm}
          \LARGE
          \DISCIPLINE

          \vspace{1.5cm}

          \textbf{\AUTHOR \\ {\tt \CONTACT}}

          \vfill
          Professor: \PROFESSOR

          \vspace{0.8cm}

          \includegraphics[width=0.2\textwidth]{../general/puc.jpg}

          \Large
          Departamento de Engenharia Mecânica\\
          PUC-RJ Pontifícia Universidade Católica do Rio de Janeiro\\
          \DATE

      \end{center}
  \end{titlepage}

\maketitle

\section{Introdução}

\subsection{Objetivos}

Esse trabalho tem como objetivo a implementação, em Python, o teste dessa implementação, em diferentes funções 
e pontos iniciais, bem como a aplicação da implementação na minimização da Energia Potencial Total de um sistema de molas,
dos seguintes métodos de otimização sem restrição :
\renewcommand{\theenumi}{\alph{enumi}}
\begin{enumerate}
  \item Univariante
  \item Powell
  \item Steepest Descent
  \item Fletcher-Reeves
  \item BFGS
  \item Newton-Raphson
\end{enumerate}

\section{Implementação}

A estratégia adotada neste trabalho foi de implementar algoritmos que, dados inputs específicos de cada método, retornam a próxima direção de busca. Para a busca unidirecional na direção especificada por cada método, 
foram aproveitados e melhorados os códigos do passo constante e da seção áurea utilizados na resolução da Lista-1. 

O código principal fica responsável pela chamada passo a passo dos métodos de OSR,
recebendo uma direção de busca e repassando a mesma para a busca unidirecional.
Com o valor de alpha obtido da seção áurea, um novo ponto é calculado e a convergência verificada.

Cabe destacar que o pacote numpy usado no python para cálculos vetoriais e matriciais não faz distinção,
para vetores 1-D, entre vetores coluna e vetores linha. Dessa forma, mesmo que possa parecer, nos códigos apresentados abaixo,
estarem sendo utilizados vetores linha para representar pontos, direções e gradientes, as especificidades das multiplicações
entre matrizes e vetores são tratadas de tal forma que o resultado seja coerente com o esperado entre a multiplicação
entre Matrizes e vetores e entre vetores e vetores.

\subsection{Busca Unidirecional}

Os algoritmos dos métodos do Passo Constante e da Seção Áurea foram implementados em um arquivo denominado line\_search\_methods.py.
O seguinte pacote é necessário nesse arquivo: 
\begin{python}
  import numpy as np
\end{python}

\subsubsection{Passo Constante}


Foi implementado um método que recebe como parâmetros de entrada um vetor direção ($\overrightarrow{d}$),
um ponto inicial ($\overrightarrow{x}^0$), 
a função que se deseja encontrar o mínimo ($f(\overrightarrow{x})$),
um valor opcional de $\epsilon$ da máquina ($\epsilon$ default com valor $10^{-8}$) e
um valor opcional de passo ($\Delta\alpha$ default com valor 0.01).

Para definir o sentido da busca, é feita uma comparação entre o valor de $f(\overrightarrow{x}^0-\epsilon\,\overrightarrow{d})$ e $f(\overrightarrow{x}^0+\epsilon\,\overrightarrow{d})$.
Caso este último valor seja maior do que o primeiro, o sentido de busca considerado é o oposto do vetor direção ($-\overrightarrow{d}$). Caso o primeiro seja
o maior valor, o sentido do vetor direção é mantido na busca ($\overrightarrow{d}$).

Com o passo default de 0.01 ou um qualquer outro passo desejado informado na passagem de parâmetro opcional,
o método percorre o sentido de busca definido até encontrar um valor de
$f(\overrightarrow{x}^0+ \alpha\,\overrightarrow{d})$ que seja inferior ao valor do próximo passo.
Quando essa condição é alcançada, esse último valor de $\alpha$ é definido como mínimo ($\alpha_{min}$). 

Para garantir que a cada incremento de $\alpha$ não tenha sido pulado um mínimo da função,
é feita uma comparação entre os valores de $f(\overrightarrow{x}-\epsilon\,\hat{d}_{unit})$ e $f(\overrightarrow{x})$, 
onde $\overrightarrow{x} = \overrightarrow{x}^0 + \alpha\,\overrightarrow{d}$.
Caso o último valor seja superior ao primeiro, o passo é desfeito e o $\alpha$ mínimo é considerado alcançado.

Caso uma das condições abaixo seja atingida, o algoritmo é interrompido e
retorna o intervalo [$\alpha$,  $\alpha + \Delta\alpha$],
fazendo os devidos ajustes para adequar os sinais de $\alpha$ de acordo com sentido de busca.

\begin{enumerate}
  \item $f(\overrightarrow{x}-\epsilon\,\hat{d}_{unit}) < f(\overrightarrow{x})$,
  com $\overrightarrow{x} = \overrightarrow{x}^0 + \alpha\,\overrightarrow{d}$ : retorna $\alpha = \alpha - \Delta \alpha$
  \item $f(\overrightarrow{x}^0+ \alpha\,\overrightarrow{d}) < f(\overrightarrow{x}^0 +
  (\alpha + \Delta\alpha)\,\overrightarrow{d})$ : retorna $\alpha$  
\end{enumerate}


\begin{python}
  def passo_cte(direcao, P0, f, eps = 1E-8, step = 0.01):
    #line search pelo metodo do passo constante
      
    #define o sentido correto de busca
    if (f(P0 - eps*(direcao/np.linalg.norm(direcao))) > f(P0 + eps*(direcao/np.linalg.norm(direcao)))):
        sentido_busca = direcao.copy()
        flag = 0
    else:
        sentido_busca = -direcao.copy()
        flag = 1
        
    P = P0.copy()
    P_next = P + step*sentido_busca
    alpha = 0
   
    while (f(P) > f(P_next)):           
        alpha = alpha + step
        P = P0 + alpha*sentido_busca
        P_next = P0 + (alpha+step)*sentido_busca
        if (f(P - eps*(sentido_busca/np.linalg.norm(sentido_busca))) < f(P)):
            alpha = alpha - step
            break
    
    intervalo = np.array([alpha, alpha + step])
    
    if(flag == 1):
        intervalo = -intervalo
        
    #retorna o intervalo de busca = [alpha min, alpha min + step]                 
    return intervalo
\end{python}

\subsubsection{Seção Áurea}

Implementado um método que recebe como parâmetros de entrada um intervalo de busca
($\overrightarrow{interv} = [\alpha^L, \alpha^U]$), um vetor direção de busca ($\overrightarrow{d}$),
um ponto inicial ($\overrightarrow{x}^0$), a função f ($f(\overrightarrow{x})$) 
e um parâmetro opcional para a tolerância de convergência com valor default de $10^{-5}$.

Resumidamente, o método utiliza a razão áurea ($R_a = \frac{\sqrt{5} - 1}{2}$) para comparar os 
valores de $f(\overrightarrow{x}^0 + \alpha_{E}\,\overrightarrow{d}$) e
$f(\overrightarrow{x}^0 + \alpha_{D}\,\overrightarrow{d})$,
onde $\alpha_{E} = \alpha^L + (1 - R_a )\beta$, $\alpha_{D} = \alpha^L + R_a\beta$ e $\beta = \alpha^U - \alpha^L$, 
para determinar em qual trecho, $[\alpha^L, \alpha_{D}]$ ou $[\alpha_{E}, \alpha^U]$, o ponto mínimo se encontra. Enquanto a
convergência não é alcançada, os valores de $\alpha^L, \alpha^U, \alpha_E$ e $\alpha_D$ vão sendo recalculados e atualizados.

Quando o comprimento do trecho a ser avaliado é inferior à tolerância, o método finaliza e retorna o seguinte valor:

\begin{itemize}
  \item $\alpha_{m}$, tal que $\alpha_{m} = \frac{\alpha^L + \alpha^U}{2}$ e $\alpha^L$, $\alpha^U$ são os extremos do
  intervalo do último passo do algortimo, quando a convergência foi obtida
\end{itemize} 

Importante destacar que o algortimo identifica o sentido de busca e
o sinal correto do valor de alpha através do valor do intervalo passado como parâmetro.


\begin{python}

def secao_aurea(intervalo, direcao, P0, f, tol=0.00001):
    #line search pelo metodo da secao aurea
    
    #verifica o sentido da busca
    if(intervalo[1] < 0):
        intervalo = -intervalo
        sentido_busca = -direcao.copy()
        flag = 1
    else:
        sentido_busca = direcao.copy()
        flag = 0
    
    #atribui os limites superior e inferior da busca a variaveis internas do metodo
    alpha_upper = intervalo[1]
    alpha_lower = intervalo[0]
    beta = alpha_upper - alpha_lower
    
    #razao aurea
    Ra = (np.sqrt(5)-1)/2
    
    # define os pontos de analise de f com base na razao aurea
    alpha_e = alpha_lower + (1-Ra)*beta
    alpha_d = alpha_lower + Ra*beta 
    
    #primeira iteracao avalia f nos 2 pontos selecionados pela razao aurea
    f1 = f(P0 + alpha_e*sentido_busca)
    f2 = f(P0 + alpha_d*sentido_busca)
    
    #loop enquanto a convergencia nao for obtida
    while (beta > tol):
        if (f1 > f2):
            #caso positivo, define novo intervalo variando de alpha_e ate alpha_upper
            # e aproveita os valores anteriores de alpha_d e f2 como novos alpha_e e f1
            alpha_lower = alpha_e
            f1 = f2
            alpha_e = alpha_d
            
            #calcula novo alpha_d e f2=f(alpha_d)
            beta = alpha_upper - alpha_lower
            #alpha_e = alpha_lower + (1-Ra)*beta
            alpha_d = alpha_lower + Ra*beta 
            f2 = f(P0 + alpha_d*sentido_busca)
        else:
            #caso negativo, define novo intervalo variando de alpha_lower ate alpha_d
            # e aproveita os valores anteriores de alpha_e e f1 como novos alpha_d e f2
            alpha_upper = alpha_d
            f2 = f1
            alpha_d = alpha_e
            
            #calcula novo alpha_e e f1=f(alpha_e)
            beta = alpha_upper - alpha_lower
            alpha_e = alpha_lower + (1-Ra)*beta
            #alpha_d = alpha_lower + Ra*beta 
            f1 = f(P0 + alpha_e*sentido_busca)
            
    # calcula Pmin e alpha min apos convergencia
    alpha_med = (alpha_lower + alpha_upper)/2
    alpha_min = alpha_med
    
    if (flag == 1):
        alpha_min = -alpha_min
    
    return alpha_min
\end{python}

\subsection{Métodos OSR}

Os algoritmos dos métodos Univariante, Powell, Steepest Descent, Fletcher-Reeves, BFGS e Newton-Raphson foram implementados em um arquivo denominado osr\_methods.py. O seguinte pacote é necessário nesse arquivo: 
\begin{python}
  import numpy as np
\end{python}

\subsubsection{Univariante}

O método univariante alterna entre as direções canônicas. A primeira vez que ele é chamado, retorna a primeira direção canônica. Na segunda vez, a segunda direção canônica, e assim por diante.
Quando todas as direções canônicas são utilizadas, reinicia-se pela primeira direção.

A implementação considerou como parâmetros de entrada o número de dimensões  e o passo em que a otimização se encontra. Ambos valores são facilmente calculados no código principal que irá chamar 
os métodos de OSR. O número de dimensões é extraído do ponto inicial e o passo é um valor controlado durante o processo de convergência que irá ser implementado no código principal.

Toda vez que o método é chamado, inicializa-se um vetor com n zeros, sendo n o número de dimensões. O índice do vetor cujo valor será alterado para 1 é calculado 
através de manipulações em cima do resto da divisão do número do passo pelo número de dimensões.

\begin{python}
  def univariante(passo, dimens):
    #indice do vetor = (resto da divisao do passo pela dimensao) - 1
    #primeira posicao do vetor no python tem indice 0
    indice = passo%dimens - 1
    
    if (indice == -1) :
        #indice = -1 indica que se trata da ultima posicao do array
        #no pyton esse indice eh o tamanho do vetor - 1
        indice = dimens - 1
        
    #define a direcao canonica a ser utilizada
    ek = np.zeros(dimens)
    ek[indice] = 1
    
    return ek
\end{python}

\subsubsection{Powell}

O método de Powell consiste de ciclos de n + 1 passos, onde n é o número de dimensões.
No primeiro ciclo e toda vez que o número do ciclo for múltiplo de n + 2, 
o conjunto de n direções é inicializado com as direções canônicas.
A direção n + 1 de todo ciclo será o igual a  $\overrightarrow{x}^n - \overrightarrow{x}^0$, 
onde $\overrightarrow{x}^n$ é o ponto encontrado na otimização até o passo n do ciclo atual e
$\overrightarrow{x}^0$ é o ponto inicial do primeiro passo do ciclo atual,
que por sua vez é igual ao ponto final do ciclo anterior.
Outra característica do método de Powell é que uma direção de um passo genérico k de um determinado ciclo será 
a direção do passo k - 1 do ciclo seguinte.

A implementação considerou como parâmetros de entrada o ponto mínimo encontrado no passo anterior ($\overrightarrow{x}$),
o ponto inicial do ciclo atual ($\overrightarrow{x}^0$), um array com o conjunto de direções,
um número representando o passo global da convergência no código principal,
um número representando o ciclo do método e o número de dimensões.

Na primeira chamada do método, ou seja, para o primeiro passo do primeiro ciclo,
o código principal envia o conjunto de direções canônicas e número de ciclo igual a zero.
Com manipulações com o resto da divisão do número do passo pelo número de dimensões,
o algoritmo consegue identificar qual direção utilizar do conjunto de direções
enviado como parâmetro e atualizar o mesmo para o ciclo seguinte.
O algoritmo também precisa atualizar o número do ciclo a cada n + 1 passos e
voltar para as direções canônicas a cada n + 2 ciclos.

A cada chamada do método, são retornadas a direção para o passo atual, o conjunto de direções atualizado,
o ponto inicial do ciclo atual  e o número de ciclos atual.
Esses valores precisam ser recebidos no código principal e utilizados para a chamada do método no passo seguinte.

\begin{python}
  def powell(P, P0, direcoes, passos, ciclos, dimens):
    #indice do vetor = (resto da divisao do passo pela dimensao) - 1
    #primeira posicao do vetor no python tem indice 0
    indice = passos%(dimens + 1) - 1
    
    if (indice == -1):
        #indice = -1 indica que se trata da ultima posicao do array
        #no pyton esse indice eh o tamanho do vetor - 1
        #direcao n + 1 do ciclo = Patual - P0
        dir = P - P0
        direcoes[dimens - 1] = dir        
    elif (indice == 0):
        #indice = 0 significa que vamos usar a primeira direcao do conjunto
        #representa o inicio de um novo ciclo
        ciclos = ciclos + 1

        if (ciclos%(dimens+2) == 0):
            #se ciclo for multipl de dimens + 2, conjunto de direcoes = canonicas
            direcoes = np.eye(dimens, dtype=float)
        P0 = P.copy()
        dir = direcoes[indice].copy()
        
    else:
        dir = direcoes[indice].copy()
        direcoes[indice-1] = dir
  
    return dir, direcoes, P0, ciclos 
\end{python}

\subsubsection{Steepest Descent}
O método Steepest Descent em tese recebe um ponto $\overrightarrow{x}$ e a função a ser minimizada e
retorna $\overrightarrow{d} = - \overrightarrow{\nabla} f(\overrightarrow{x})$.

Porém, a implementação considerou como parâmetro de entrada apenas o vetor gradiente da função no ponto $\overrightarrow{P}$,
uma vez que o código principal já faz o cálculo dessa variável para verificar convergência e
a mesma já está disponível no momento de chamada do método.

\begin{python}
 def steepestDescent(grad):
    return -grad
\end{python}

\subsubsection{Fletcher-Reeves}
O método Fletcher-Reeves utiliza $-\overrightarrow{\nabla}f$ no ponto $\overrightarrow{x}^0$ como primeira direção. A partir da segunda direção
ele utiliza uma correção $\beta$ que é a razão ao quadrado entre o gradiente da função no ponto encontrado no passo anterior e o gradiente da função no ponto
inicial do passo anterior. Ou seja, $\beta^k = \frac{(\lvert\overrightarrow{\nabla f}(\overrightarrow{x}^{k+1})\rvert)^2}{(\lvert\overrightarrow{\nabla f}(\overrightarrow{x}^k)\rvert)^2}$.
A correção é então utilizada em conjunto com a direção do último passo e o gradiente da função no ponto obtido no último passo. Dessa forma,
$\overrightarrow{d}^{k+1} = -\overrightarrow{\nabla}f(\overrightarrow{x}^{k+1}) + \beta^k\,\overrightarrow{d^k}$

A implementação considerou como parâmetros de entrada a direção utilizada no passo anterior, o gradiente da função no ponto obtido no passo anterior, o gradiente 
da função no ponto inicial do passo anterior e o número do passo. Para o primeiro passo do método, o código principal envia o gradiente da função no ponto inicial.
O método retorna a direção a ser utilizada e o gradiente a ser usado no passo seguinte como gradiente da função f no ponto inicial do passo anterior. O código principal,
recebe esses valores e utiliza nas chamadas subsequentes.

\begin{python}
 def fletcherReeves(dir_last, grad, grad_last, passo):
    if passo == 1:
        grad_last = grad.copy()
        return -grad, grad_last
    else:
        beta = (np.linalg.norm(grad)/np.linalg.norm(grad_last))**2
        grad_last = grad.copy()
        return -grad + beta*dir_last, grad_last
\end{python}

\subsubsection{BFGS}

O método BFGS retorna $\overrightarrow{d}^k = -\bar{\bar{S}}^k\,\overrightarrow{\nabla}f(\overrightarrow{x}^k)$ como direção,
sendo $\bar{\bar{S}}^0 = \bar{\bar{I}}$.

\begin{figure}[htpb]
  \centering
  \includegraphics[scale=0.5]{figuras/bfgs_auxiliares.PNG}
  \caption{Variáveis Auxiliares. Fonte: Aula-3\_MetodosOSR.pdf}
\end{figure}

\begin{figure}[htpb]
  \centering
  \includegraphics[scale=0.5]{figuras/bfgs_sk.PNG}
  \caption{Cálculo de $S^{k+1}$. Fonte: Aula-3\_MetodosOSR.pdf}
\end{figure}

A implementação considerou como parâmetros de entrada os pontos inicial e final do passo anterior,
os gradientes da função nos pontos inicial e final do passo anterior,
a matriz $\bar{\bar{S}}$ calculada no último passo, o número do passo e o número de dimensões.
O algortimo retorna a direção atual a ser utilizada na busca,
bem como os demais parâmetros atualizados necessários para a chamada subsequente do método,
que precisarão ser lidos no código principal.
O código principal se encarrega de fornecer o valor de $\bar{\bar{S}}^0$ na chamada do primeiro passo.

\begin{python}
  def bfgs(P, P_last, grad, grad_last, S_last, passo, dimens):
    if (passo == 1):
        dir = -S_last.dot(grad)
    else:
        delta_x_k = P - P_last
        delta_g_k = grad - grad_last
        
        #para o numpy, vetor 1-D linha e vetor coluna sao a mesma coisa (nao e necessrio transpor)
        #matrizes
        A = np.outer(delta_x_k, np.transpose(delta_x_k))
        B = S_last.dot(np.outer(delta_g_k, np.transpose(delta_x_k)))
        C = np.outer(delta_x_k, np.transpose(S_last.dot(delta_g_k)))
        
        #Escalares        
        d = np.transpose(delta_x_k).dot(delta_g_k)
        e = np.transpose(delta_g_k).dot(S_last.dot(delta_g_k))
                
        S = S_last + (d + e)*A/(d**2) - (B + C)/d
        dir = -S.dot(grad)
        S_last = S.copy()
    P_last = P
    grad_last = grad
    return dir, P_last, grad_last, S_last
\end{python}

\subsubsection{Newton-Raphson}
O método Newton-raphson retorna
$\overrightarrow{d}^k = -\bar{\bar{H}}(\overrightarrow{x}^k)^{-1}\,\overrightarrow{\nabla} f(\overrightarrow{x}^k)$.

A implementação considerou como parâmetros de entrada o ponto inicial ou ponto obtido no passo anterior,
o gradiente da função nesse ponto e o método que caclula a matriz Hessiana de f.

\begin{python}
  def newtonRaphson(P, grad_P, hessian_f):
    return -np.linalg.inv(hessian_f(P)).dot(grad_P)
\end{python}

\subsection{Código Principal}

Os seguintes pacotes foram utilizados na implementação do código principal, já inclusos os arquivos .py com as implementações dos métodos OSR e busca unidimensional.

\begin{python}
  import numpy as np
  import osr_methods as osr
  import line_search_methods as lsm
  import numdifftools as nd
  import matplotlib.pyplot as plt
  from timeit import default_timer as timer
\end{python}

Criada uma seção com as variáveis responsáveis pelo controle numérico da minimização das funções.

\begin{python}
  # numero maximo de iteracoes 
  maxiter = 200

  # tolerancia para convergencia do gradiente
  tol_conv = 1E-5

  # tolerancia para a busca unidirecional
  tol_search = 1E-5

  # delta alpha do passo constante
  line_step = 1E-2

  #epsilon da maquina
  eps = 1E-8
\end{python}

Seção para escolha do método a ser utilizado na minimização.

\begin{python}
  # 1 - Univariante
  # 2 - Powell
  # 3 - Steepest Descent
  # 4 - Newton Raphson
  # 5 - Fletcher Reeves
  # 6 - BFGS

  metodo = 1

  if (metodo == 1):
      n_met = 'Univariante'
  elif (metodo == 2):
      n_met = 'Powell'
  elif (metodo == 3):
      n_met = 'Steepest Descent'
  elif (metodo == 4):
      n_met = 'Newton Raphson'
  elif (metodo == 5):
      n_met = 'Fletcher-Reeves'
  elif (metodo == 6):
      n_met = 'BFGS'
\end{python}

Seção reservada para definição da função f, do gradiente de f, da Hessiana de f e ponto inicial. Nesse momento deixei em branco as definições, mas na seção de cada 
questão com sua função específica as definições serão apresentadas.

\begin{python}
  def f(Xn):
    return ...
      
   def grad_f(Xn):
    return ...
      
   def hessian_f(Xn):
    return ...   
  
  P0 = ...
  
  #numero da funcao  - apenas para automatizacao do plot de contorno
  #1 = Questao 1 letra a
  #2 = Questao 1 letra b
  #3 = Questao 2 letra a
  #4 = Questao 2 lebra b
  func = ...

  # nome da questao e letra - apenas para automatizcao do plot
  if (func == 1):
    n_func = 'Q1.a'
  elif (func == 2):
    n_func = 'Q1.b'
  elif (func == 3):
    n_func = 'Q2.a'
  elif (func == 4):
    n_func = 'Q2.b'
\end{python}



Seção para inicialização de variáveis auxiliares para chamada inicial dos métodos de otimização OSR.

\begin{python}
  passos = 0
  dimens = P0.size
  Pmin = P0.copy()
  listPmin = []
  listPmin.append(Pmin)
  grad = grad_f(Pmin)
  norm_grad = np.linalg.norm(grad)

  if (metodo == 2):
      direcoes = np.eye(dimens, dtype=float)
      ciclos = 0
      P1 = P0.copy()
  elif (metodo == 5):
      #o metodo recebe a direcao anterior 
      #inicializo a direcao com um vetor de zeros mas que nunca e usado
      #uso apenas para enviar como parametro na primeira iteracao do metodo, o qual atualiza o valor de dir para a iteracao seguinte
      dir = np.zeros((1, dimens))
      grad_last = grad.copy()
  elif(metodo == 6):
      S_last = np.eye(dimens)
      grad_last = grad.copy()
      P_last = P0.copy()
\end{python}

Loop para chamada dos métodos e convergência da otimização. A cada iteração armazeno o valor do ponto encontrado na variável
$P_{min}$ e adiciono o ponto numa lista(vetor) para acesso futuro e plot do caminho do método nas curvas de nível.

\begin{python}
  start = timer()

  while (norm_grad > tol_conv):
      if (passos == maxiter):
          print('Nao convergiu')
          break
      passos = passos + 1
      if (metodo == 1):
          dir = osr.univariante(passos, dimens)
      elif (metodo == 2):
          dir, direcoes, P1, ciclos = osr.powell(Pmin, P1, direcoes,passos, ciclos, dimens)
      elif (metodo == 3):
          dir = osr.steepestDescent(grad)
      elif (metodo == 4):
          dir = osr.newtonRaphson(Pmin, grad, hessian_f)
      elif (metodo == 5):
          dir, grad_last = osr.fletcherReeves(dir, grad, grad_last, passos)
      elif (metodo == 6):
          dir, P_last, grad_last, S_last = osr.bfgs(Pmin, P_last, grad, grad_last, S_last, passos, dimens)
      
      intervalo = lsm.passo_cte(dir, Pmin, f, eps, line_step)
      alpha = lsm.secao_aurea(intervalo, dir, Pmin, f, tol_search)
      Pmin = Pmin + alpha*dir
      listPmin.append(Pmin)
      grad = grad_f(Pmin)
      norm_grad = np.linalg.norm(grad)    

  end = timer()

  tempoExec = end - start
\end{python}

Seção para geração automática do gráfico de configuração dos nós da questão 2 letra b e das curvas de nível para as demais questões.

\begin{python}
  if (func < 4):
    if (func == 1):
        x1 = np.linspace(-6, 3, 100)
        x2 = np.linspace(-4, 2.5, 100)
        X1, X2 = np.meshgrid(x1, x2)
        X3 = f([X1, X2])
        niveis = plt.contour(X1, X2, X3, [0, 1, 3, 8, 15, 25, 40], colors='black')
    elif (func == 2):
        x1 = np.linspace(-5, 25, 100)
        x2 = np.linspace(-10, 10, 100)
        X1, X2 = np.meshgrid(x1, x2)
        X3 = f([X1, X2])
        niveis = plt.contour(X1, X2, X3,[50, 100,200,500, 1000, 2000, 5000], colors='black')
    elif (func == 3):
        x1 = np.linspace(-3, 3, 100)
        x2 = np.linspace(-6, 15, 100)
        X1, X2 = np.meshgrid(x1, x2)
        X3 = f([X1, X2])
        niveis = plt.contour(X1, X2, X3,[-2000,-1500, -1000, -500,500,2000], colors='black')
        
    plt.clabel(niveis, inline=1, fontsize=10)
    x = []
    y = []
    for P in listPmin:
        x.append(P[0])
        y.append(P[1])
    plt.plot(x, y, color='g', linewidth='3')
    plt.xlabel('$x_1$', fontsize='16')
    plt.ylabel('$x_2$', fontsize='16')
    plt.grid(linestyle='--')
    titulo = n_func + ' ' + n_met
    plt.title(titulo, fontsize='16')
    file_name = n_func + '_' + n_met + '_P0=' + np.array2string(P0, precision = 2, separator=' ') + '.pdf'
    plt.savefig(file_name, format="pdf")
    plt.show()
  elif (func == 4):
      x = []
      y = []
      n = int(dimens/2)
      m = n + 1
      Li = np.zeros(m, dtype=float) # comprimentos iniciais das molas
      Li = Li + 60/m
      print(Li)
      x.append(0)
      y.append(0)
      comp = 0
      for k in np.arange(n):
          comp = comp + Li[k]
          x.append(comp + Pmin[2*k])
          y.append(Pmin[2*k + 1])
      x.append(60)
      y.append(0)
      
      fig, ax = plt.subplots()
      ax.tick_params(top=True, labeltop=True, bottom=False, labelbottom=False)
      ax.xaxis.set_label_position('top')
      ax.spines['left'].set_position(('data', 0))
      ax.spines['top'].set_position(('data', 0))

          
      plt.plot(x, y, marker = 'o', color='g', linewidth='3')
      plt.ylim([0, 8])
      plt.xlim([0, 60])
      plt.xlabel('$x$', fontsize='16')
      plt.ylabel('$y$', fontsize='16')
      plt.grid()
      plt.gca().invert_yaxis()
      plt.show()
\end{python}

\section{Teste da Implementação}
A questão 1 solicita a busca dos pontos de mínimo de duas funções distintas como forma de testar a implementação
dos métodos de OSR e busca unidirecional apresentados na seção anterior.
Uma das funções é quadrática e a outra não. Para funções qudráticas,
é esperado que o método de Powell atinja convergência em até $(n + 1)^2$ passos,
o método de Fletcher-Reeves em até $n + 1$ passos, o método BFGS em até $n + 1$ passos e
o método Newton-Raphson em até $n$ passos, sendo n o número de dimensões (ou variáveis) da função.

\vspace{5mm}
\textbf{O seguinte controle numérico foi utilizado para a convergência das funções da questão 1,
 independente do ponto inicial:}

\begin{itemize}
  \item Número máximo de passos (ou iterações): 200
  \item Tolerância para convergência do gradiente: $10^{-5}$
  \item Tolerância para convergência da busca unidirecional: $10^{-6}$
  \item $\Delta\alpha$ do passo constante: $10^{-2}$
\end{itemize}

\subsection{Questão 1 (a)}

\vspace{5mm}

\begin{center}
$f(x_1,x_2) = x_1^2 - 3x_1x_2 + 4x_2^2 + x_1 - x_2$

\vspace{5mm}
$\overrightarrow{\nabla} f(x_1,x_2) = 
\begin{bmatrix}
  2x_1 - 3x_2 + 1 \\ -3x_1 + 8x_2 -1
\end{bmatrix}$

\vspace{5mm}

$H(x_1,x_2) = 
\begin{bmatrix}
  \phantom{-}2 & -3 \\ -3 & \phantom{-}8
\end{bmatrix}$

\end{center}

\vspace{5mm}

Definição da função, gradiente e Hessiana no código principal :

\begin{python}
  def f(Xn):
    return Xn[0]**2 - 3*Xn[0]*Xn[1] + 4*(Xn[1]**2) + Xn[0] - Xn[1]
    
  def grad_f(Xn):
      return np.array([2*Xn[0] - 3*Xn[1] + 1, -3*Xn[0] + 8*Xn[1] - 1])
      
  def hessian_f(Xn):
      return np.array([[2, -3],
                      [-3, 8]], dtype=float)
    
  func = 1
\end{python}


\subsubsection{Ponto inicial: $x^0 = \{2,2\}^t $}
Definição do ponto inicial no código principal:
\begin{python}
  P0 = np.array([2, 2])
\end{python}

\vspace{3mm}
\textbf{Principais resultados obtidos: }

\vspace{3mm}
A tabela resumo abaixo mostra que para essa função quadrática os métodos apresentaram resultados satisfatórios,
com os métodos de Powell, Fletcher-Reeves, BFGS e Newton-Rapshon respeitando o número máximo de passos para convergência.

\begin{table}[H]
  \begin{center}
    \begin{tabular}{c|c|c|c}
      \textbf{Método} & \textbf{\# Passos} & \textbf{Tempo(s)} & \textbf{$P_{min}$}\\
      \hline
      Univariante & 46 & 0.03397 & $\{-0.71427545, -0.14285347\}^t$\\
      Powell & 6 & 0.09025 & $\{-0.71428536, -0.14285781\}^t$\\
      Steepest Descent & 31 & 0.03088 & $\{-0.71427858, -0.14285438\}^t$\\
      Fletcher-Reeves & 3 & 0.00991 & $\{-0.71428613, -0.14285732\}^t$\\
      BFGS & 2 & 0.00495 & $\{-0.71428573, -0.14285724\}^t$\\
      Newton-Raphson & 1 & 0.00241 & $\{-0.71428661, -0.14285785\}^t$\\
    \end{tabular}
  \end{center}
  \caption{Resumo dos resultados obtidos na questão 1a para $x^0 = \{2,2\}^t$}
\end{table}

\begin{figure}[H]
  \centering
  \includegraphics[scale=0.4]{figuras/q1a_passos_P1.PNG}
  \caption{Número de passos por método OSR. Questão 1a e $x^0 = \{2,2\}^t$ }
\end{figure}

\vspace{3mm}
A figura abaixo mostra, por método, o valor da função a cada iteração. A ideia é podermos comparar 
a rapidez com que cada método se aproxima do mínimo. Para essa função e ponto inicial, os métodos univariante e Powell
foram os que mais demoraram para se aproximar, levando por volta de 6 passos, enquanto os demais precisaram de no máximo
3 passos.

\begin{figure}[H]
  \centering
  \includegraphics[scale=0.45]{figuras/q1a_fxpassos_P1.PNG}
  \caption{Gráficos de $f(x_1,x_2)$ versus passo da minimização, por método. Questão 1a e $x^0 = \{2,2\}^t$}
\end{figure}

As figuras abaixo representam as curvas de nível e o caminho de otimização percorrido por cada método. Como a função 
só possui um mínimo, todos os métodos convergiram o ponto correto (mínimo mais próximo).

\begin{figure}[H]
  \centering
  \begin{subfigure}[b]{\textwidth}
    \includegraphics[width=0.49\textwidth]{figuras/Q1.a_Univariante_P0=[2 2].pdf}
    \includegraphics[width=0.49\textwidth]{figuras/Q1.a_Powell_P0=[2 2].pdf}
  \end{subfigure}
  \caption{Curvas de nível e os pontos obtidos pelo Univariante e Powell. Questão 1a e $x^0 = \{2,2\}^t$}
\end{figure}

\begin{figure}[H]
  \centering
  \begin{subfigure}[b]{\textwidth}
    \includegraphics[width=0.49\textwidth]{figuras/Q1.a_Steepest Descent_P0=[2 2].pdf}
    \includegraphics[width=0.49\textwidth]{figuras/Q1.a_Fletcher-Reeves_P0=[2 2].pdf}
  \end{subfigure}
  \caption{Curvas de nível e os pontos obtidos pelo Steepest Descent e Fletcher-Reeves. Questão 1a e $x^0 = \{2,2\}^t$}
\end{figure}

\begin{figure}[H]
  \centering
  \begin{subfigure}[b]{\textwidth}
    \includegraphics[width=0.49\textwidth]{figuras/Q1.a_BFGS_P0=[2 2].pdf}
    \includegraphics[width=0.49\textwidth]{figuras/Q1.a_Newton Raphson_P0=[2 2].pdf}
  \end{subfigure}
  \caption{Curvas de nível e os pontos obtidos pelo BFGS e Newton-Raphson. Questão 1a e $x^0 = \{2,2\}^t$}
\end{figure}

\subsubsection{Ponto inicial : $x^0 = \{-1,-3\}^t$}
Definição do ponto inicial no código principal:
\begin{python}
  P0 = np.array([-1, -3])
\end{python}

\vspace{3mm}
\textbf{Principais resultados obtidos:}

\vspace{3mm}
Segue tabela resumo e figuras nos mesmos moldes do que foi apresentado para o outro ponto inicial.
As conclusões são basicamente as mesmas.

\begin{table}[H]
  \begin{center}
    \begin{tabular}{c|c|c|c}
      \textbf{Método} & \textbf{\# Passos} & \textbf{Tempo(s)} & \textbf{$P_{min}$}\\
      \hline
      Univariante & 48 & 0.04435 & $\{-0.7142935, -0.14286022\}^t$\\
      Powell & 6 & 0.02483 & $\{-0.71428418, -0.14285757\}^t$\\
      Steepest Descent & 7 & 0.00690 & $\{-0.71429411, -0.14286059\}^t$\\
      Fletcher-Reeves & 3 & 0.00467 & $\{-0.71428581, -0.14285718\}^t$\\
      BFGS & 2 & 0.00332 & $\{-0.71428583, -0.14285714\}^t$\\
      Newton-Raphson & 1 & 0.00219 & $\{-0.71428562, -0.1428562\}^t$\\
    \end{tabular}
  \end{center}
  \caption{Resumo dos resultados obtidos na questão 1a para $x^0 = \{-1,-3\}^t$}
\end{table}

\begin{figure}[H]
  \centering
  \includegraphics[scale=0.4]{figuras/q1a_passos_P2.PNG}
  \caption{Número de passos por método OSR. Questão 1a e $x^0 = \{-1,-3\}^t$ }
\end{figure}

\begin{figure}[H]
  \centering
  \includegraphics[scale=0.45]{figuras/q1a_fxpassos_P2.PNG}
  \caption{Gráficos de $f(x_1,x_2)$ versus passo da minimização, por método. Questão 1a e $x^0 = \{-1,-3\}^t$}
\end{figure}

\begin{figure}[H]
  \centering
  \begin{subfigure}[b]{\textwidth}
    \includegraphics[width=0.49\textwidth]{figuras/Q1.a_Univariante_P0=[-1 -3].pdf}
    \includegraphics[width=0.49\textwidth]{figuras/Q1.a_Powell_P0=[-1 -3].pdf}
  \end{subfigure}
  \caption{Curvas de nível e os pontos obtidos pelo Univariante e Powell. Questão 1a e $x^0 = \{-1,-3\}^t$}
\end{figure}

\begin{figure}[H]
  \centering
  \begin{subfigure}[b]{\textwidth}
    \includegraphics[width=0.49\textwidth]{figuras/Q1.a_Steepest Descent_P0=[-1 -3].pdf}
    \includegraphics[width=0.49\textwidth]{figuras/Q1.a_Fletcher-Reeves_P0=[-1 -3].pdf}
  \end{subfigure}
  \caption{Curvas de nível e os pontos obtidos pelo Steepest Descent e Fletcher-Reeves. Questão 1a e $x^0 = \{-1,-3\}^t$}
\end{figure}

\begin{figure}[H]
  \centering
  \begin{subfigure}[b]{\textwidth}
    \includegraphics[width=0.49\textwidth]{figuras/Q1.a_BFGS_P0=[-1 -3].pdf}
    \includegraphics[width=0.49\textwidth]{figuras/Q1.a_Newton Raphson_P0=[-1 -3].pdf}
  \end{subfigure}
  \caption{Curvas de nível e os pontos obtidos pelo BFGS e Newton-Raphson. Questão 1a e $x^0 = \{-1,-3\}^t$}
\end{figure}

\subsection{Questão 1 (b)}
Considerar $a = 10$ e $b = 1$. Utilizei o site Wolfram Alpha para cálculo do gradiente e da Hessiana.

\vspace{5mm}

\begin{center}
$f(x_1,x_2) = (1 + a -bx_1 - bx_2)^2 + (b + x_1 + ax_2 - bx_1x_2)^2$

\vspace{5mm}
$\overrightarrow{\nabla} f(x_1,x_2) = 
\begin{bmatrix}
  2(-a(bx_2^2 + b - x_2) + b^2x_1(x_2^2 + 1) -2bx_1x_2 + x_1) \\ -2b(2ax_1x_2 + x_1^2 + 1) + 2a(ax_2 + x_1) + 2b^2(x_1^2 +1)x_2
\end{bmatrix}$

\vspace{5mm}

$H_{1\text{x}1}(x_1,x_2) = 2b^2 + 2(1 - bx_2)^2$

$H_{1\text{x}2}(x_1,x_2) = -2b(ax_2 + b(-x_1)x_2 + b + x_1) + 2(1 - bx_2)(a - bx_1) + 2 b^2$

$H_{2\text{x}1}(x_1,x_2) = -2b(ax_2 + b(-x_1)x_2 + b + x_1) + 2(1 - bx_2) (a - bx_1) + 2 b^2$

$H_{2\text{x}2}(x_1,x_2) = 2(a - bx_1)^2 + 2b^2$

\end{center}

\vspace{5mm}

Definição da função, gradiente e Hessiana no código principal :

\begin{python}
  def f(Xn):
    a = 10
    b = 1
    return (1 + a - b*Xn[0] - b*Xn[1])**2 + (b + Xn[0] + a*Xn[1] - b*Xn[0]*Xn[1])**2 
    
  def grad_f(Xn):
      a = 10
      b = 1
      return np.array([2*(-a*(b*(Xn[1]**2) + b - Xn[1]) + (b**2)*Xn[0]*(Xn[1]**2 + 1) - 2*b*Xn[0]*Xn[1] + Xn[0]),
                      -2*b*(2*a*Xn[0]*Xn[1] + Xn[0]**2 + 1) + 2*a*(a*Xn[1] + Xn[0]) + 2*(b**2)*(Xn[0]**2 + 1)*Xn[1]])
  def hessian_f(Xn):
      a = 10
      b = 1
      hessian = np.zeros((2,2))
      hessian[0, 0] = 2*(b**2) + 2*((1 - b*Xn[1])**2)
      hessian[0, 1] = -2*b*(a*Xn[1] + b*(-Xn[0]*Xn[1]) + b + Xn[0]) + 2*(1-b*Xn[1])*(a - b*Xn[0]) + 2*(b**2)
      hessian[1, 0] = -2*b*(a*Xn[1] + b*(-Xn[0]*Xn[1]) + b + Xn[0]) + 2*(1-b*Xn[1])*(a-b*Xn[0]) + 2*(b**2)
      hessian[1, 1] = 2*((a-b*Xn[0])**2) + 2*(b**2)
      return hessian
      
  func = 2
\end{python}

\subsubsection{Ponto inicial : $x^0 = \{10,2\}^t $}

Definição do ponto inicial no código principal:
\begin{python}
  P0 = np.array([10, 2])
\end{python}

A tabela resumo abaixo mostra que para essa função, que não é quadrática, todos os métodos conseguiram convergir dentro
do limite especificado para o número máximo de iterações.

Os métodos de Powell, Fletcher-Reeves e BFGS não convergiram dentro
do número de passos esperado para funções quadráticas, o que era de certa forma esperado.  Newton-Rapshson convergiu em
1 passo, mas foi o único que encontrou um ponto diferente dos demais. Ele convergiu para um ponto de sela imediatamente
abaixo do ponto inicial, enquanto os demais métodos convergiram para o mínimo mais próximo, localizado à direita
do ponto inicial. Esses detalhes são melhores vistos na figura com as curvas de nível localizada mais abaixo neste documento.

\begin{table}[H]
  \begin{center}
    \begin{tabular}{c|c|c|c}
      \textbf{Método} & \textbf{\# Passos} & \textbf{Tempo(s)} & \textbf{$P_{min}$}\\
      \hline
      Univariante & 64 & 0.03673 & $\{13.00000142, 3.99999883\}^t$\\
      Powell & 15 & 0.02147 & $\{13.00000057, 3.99999962\}^t$\\
      Steepest Descent & 55 & 0.00927 & $\{13.00000099, 3.99999874\}^t$\\
      Fletcher-Reeves & 71 & 0.01208 & $\{12.99999937, 4.00000089\}^t$\\
      BFGS & 9 & 0.01836 & $\{13.00000042, 3.99999969\}^t$\\
      Newton-Raphson & 1 & 0.00272 & $\{10, 0.99999967\}^t$\\
    \end{tabular}
  \end{center}
  \caption{Resumo dos resultados obtidos na questão 1b para $x^0 = \{10,2\}^t$}
\end{table}

\begin{figure}[H]
  \centering
  \includegraphics[scale=0.4]{figuras/q1b_passos_P1.PNG}
  \caption{Número de passos por método OSR. Questão 1b e $x^0 = \{10,2\}^t$ }
\end{figure}

A figura abaixo deixa claro como Newton-Raphson, apesar de ter convergido em 1 iteração, praticamente não minimizou
a função como os demais métodos, por ter encontrado um ponto de sela. Flecther-Reeves foi o método que mais demorou
para se aproximar do ponto mínimo, e o motivo ficará mais evidente mais abaixo quando eu apresentar o caminho percorrido
por este método.

\begin{figure}[H]
  \centering
  \includegraphics[scale=0.4]{figuras/q1b_fxpassos_P1.PNG}
  \caption{Gráficos de $f(x_1,x_2)$ versus passo da minimização, por método. Questão 1b e $x^0 = \{10,2\}^t$}
\end{figure}

\begin{figure}[H]
  \centering
  \begin{subfigure}[b]{\textwidth}
    \includegraphics[width=0.49\textwidth]{figuras/Q1.b_Univariante_P0=[10e2].pdf}
    \includegraphics[width=0.49\textwidth]{figuras/Q1.b_Powell_P0=[10e2].pdf}
  \end{subfigure}
  \caption{Curvas de nível e os pontos obtidos pelo Univariante e Powell. Questão 1b e $x^0 = \{10,2\}^t$}
\end{figure}

\begin{figure}[H]
  \centering
  \begin{subfigure}[b]{\textwidth}
    \includegraphics[width=0.49\textwidth]{figuras/Q1.b_Steepest Descent_P0=[10e2].pdf}
    \includegraphics[width=0.49\textwidth]{figuras/Q1.b_Fletcher-Reeves_P0=[10e2].pdf}
  \end{subfigure}
  \caption{Curvas de nível e os pontos obtidos pelo Steepest Descent e Fletcher-Reeves. Questão 1b e $x^0 = \{10,2\}^t$}
\end{figure}

\begin{figure}[H]
  \centering
  \begin{subfigure}[b]{\textwidth}
    \includegraphics[width=0.49\textwidth]{figuras/Q1.b_BFGS_P0=[10e2].pdf}
    \includegraphics[width=0.49\textwidth]{figuras/Q1.b_Newton Raphson_P0=[10e2].pdf}
  \end{subfigure}
  \caption{Curvas de nível e os pontos obtidos pelo BFGS e Newton-Raphson. Questão 1b e $x^0 = \{10,2\}^t$}
\end{figure}

Um resultado que chama a atenção é o método Fletcher-Reeves ter levado 71 passos para convergir. Alterar a tolerância
de convergência global para $10^{-4}$ e a tolerância da seção áurea, na busca unidirecional, para $10^{-8}$ levou a uma
redução modesta para 60 iterações. Analisando as curvas de nível e o "caminho" de otimização gerado pelo método,
percebe-se que ocorre um movimento em espiral em torno do ponto mínimo. Ou seja, para essa função e ponto inicial,
as direções geradas por Fletcher-Reeves levam a um elevado número de passos.

Nenhum dos métodos levou ao ponto de mínimo à esquerda do ponto incicial.

\begin{figure}[H]
  \centering
  \includegraphics[scale=0.5]{figuras/Q1.b_espiral_Fletcher-Reeves_P0=[10e2].pdf}
  \caption{Movimento em espiral do método Fletcher-Reeves. Questão 1b e $x^0 = \{10,2\}^t$ }
\end{figure}

\subsubsection{Ponto inicial : $x^0 = \{-2,-3\}^t $}

Definição do ponto inicial no código principal:
\begin{python}
  P0 = np.array([-2, -3])
\end{python}

Seguem resultados para o segundo ponto inicial proposto no enunciado. Pela posição deste ponto e da proximidade com 
o mínimo mais à esquerda da função, dessa vez, todos os métodos convergiram corretamente para o mínimo esperado.
Em relação aos resultados não há novas observações relevantes a serem feitas, com exceção do fato de todos os métodos
terem conseguido se aproximar muito rápido do mínimo.

\begin{table}[H]
  \begin{center}
    \begin{tabular}{c|c|c|c}
      \textbf{Método} & \textbf{\# Passos} & \textbf{Tempo(s)} & \textbf{$P_{min}$}\\
      \hline
      Univariante       & 61  & 0.04288 & $\{7.00000124, -2.00000132\}^t$\\
      Powell            & 15  & 0.13350 & $\{7.0000002, -1.99999995\}^t$\\
      Steepest Descent  & 45  & 0.01018 & $\{7.000001, -2.0000009\}^t$\\
      Fletcher-Reeves   & 21  & 0.00642 & $\{7.00000007, -2.00000017\}^t$\\
      BFGS              & 8   & 0.04286 & $\{7.00000021, -2.00000023\}^t$\\
      Newton-Raphson    & 6   & 0.01332 & $\{7.00000001, -2.00000001\}^t$\\
    \end{tabular}
  \end{center}
  \caption{Resumo dos resultados obtidos na questão 1b para $x^0 = \{-2,-3\}^t$}
\end{table}

\begin{figure}[H]
  \centering
  \includegraphics[scale=0.4]{figuras/q1b_passos_P2.PNG}
  \caption{Número de passos por método OSR. Questão 1b e $x^0 = \{-2,-3\}^t$ }
\end{figure}

\begin{figure}[H]
  \centering
  \includegraphics[scale=0.45]{figuras/q1b_fxpassos_P2.PNG}
  \caption{Gráficos de $f(x_1,x_2)$ versus passo da minimização, por método. Questão 1b e $x^0 = \{-2,-3\}^t$}
\end{figure}

\begin{figure}[H]
  \centering
  \begin{subfigure}[b]{\textwidth}
    \includegraphics[width=0.49\textwidth]{figuras/Q1.b_Univariante_P0=[-2e-3].pdf}
    \includegraphics[width=0.49\textwidth]{figuras/Q1.b_Powell_P0=[-2e-3].pdf}
  \end{subfigure}
  \caption{Curvas de nível e os pontos obtidos pelo Univariante e Powell. Questão 1b e $x^0 = \{-2,-3\}^t$}
\end{figure}

\begin{figure}[H]
  \centering
  \begin{subfigure}[b]{\textwidth}
    \includegraphics[width=0.49\textwidth]{figuras/Q1.b_Steepest Descent_P0=[-2e-3].pdf}
    \includegraphics[width=0.49\textwidth]{figuras/Q1.b_Fletcher-Reeves_P0=[-2e-3].pdf}
  \end{subfigure}
  \caption{Curvas de nível e os pontos obtidos pelo Steepest Descent e Fletcher-Reeves. Questão 1b e $x^0 = \{-2,-3\}^t$}
\end{figure}

\begin{figure}[H]
  \centering
  \begin{subfigure}[b]{\textwidth}
    \includegraphics[width=0.49\textwidth]{figuras/Q1.b_BFGS_P0=[-2e-3].pdf}
    \includegraphics[width=0.49\textwidth]{figuras/Q1.b_Newton Raphson_P0=[-2e-3].pdf}
  \end{subfigure}
  \caption{Curvas de nível e os pontos obtidos pelo BFGS e Newton-Raphson. Questão 1b e $x^0 = \{-2,-3\}^t$}
\end{figure}

\section{Aplicação da Implementação}
\subsection{Questão 2 (a)}
\subsubsection{Enunciado}
Determinar os deslocamentos $(u_A, v_A)$ do ponto $A$, que minimizam a Energia Potencial Total $(\Pi)$ do sitema de molas
indicado na figura abaixo. Adotar o ponto inicial: $x^0 = \{0.01, -0.10\} ^t$

\begin{figure}[H]
  \centering
  \includegraphics[scale=1]{figuras/enunciado_2a.png}
  \caption{Sistema 2 molas. Fonte: Questão 2a do Trabalho 1. }
\end{figure}

\subsubsection{Formulação}
A seguinte formulação foi apresentada em sala de aula :
\vspace{3mm}

$A = (x_A, y_A)$ = Posição inicial do ponto A

\vspace{3mm}
$A' = (x_A + u_A, y_A + v_A)$ = Posição final do ponto A

\vspace{3mm}
$\Pi = U - V$ = Energia Potencial total

\vspace{3mm}
$U = U_{mola1} + U_{mola2}$ = energia interna de deformação

\vspace{3mm}
$V$ = trabalho das forças externas

\vspace{5mm}
$U_i = \frac{1}{2}K_i\Delta L_i^2$

\vspace{3mm}
$L'_1 = \sqrt{(L_1 + u_A)^2 + v_A^2}$ , $L'_2 = \sqrt{(L_2 - u_A)^2 + v_A^2}$ e $W = \frac{1}{2}(\rho_1L_1 + \rho_2L_2)$

\vspace{3mm}
$V = Wv_A$

\vspace{5mm}
$\Pi = \frac{1}{2}\frac{EA_1}{L_1}(\sqrt{(L_1 + x_1)^2 + x_2^2} - L_1)^2 +
        \frac{1}{2}\frac{EA_2}{L_2}(\sqrt{(L_2 - x_1)^2 + x_2^2} - L_2)^2 - 
        (\frac{\rho_1L_1}{2} + \frac{\rho_2L_2}{2})x_2$

\vspace{3mm}
Substituindo os valores do enunciado e usando o site Wolfram ALpha para cálculo do gradiente e Hessiana :

\vspace{3mm}
$\Pi = 450(\sqrt{(30 + x_1)^2 + x_2^2} - 30)^2 +
        300(\sqrt{(30 - x_1)^2 + x_2^2} - 30)^2 - 
        360x_2$

\vspace{3mm}

 $\overrightarrow{\nabla}\Pi =
\begin{bmatrix}
  \frac{900 (x_1 + 30) (\sqrt{(x_1 + 30)^2 + x_2^2} - 30)}{\sqrt{(x_1 + 30)^2 + x_2^2}} - 
  \frac{600 (30 - x_1) (\sqrt{(x_1 - 30)^2 + x_2^2} - 30)}{\sqrt{(x_1 - 30)^2 + x_2^2}} \\[5mm]
  60 (x_2 (\frac{-450}{\sqrt{x_1^2 + 60 x_1 + x_2^2 + 900}} - \frac{300}{\sqrt{x_1^2 - 60 x_1 + x_2^2 + 900}} + 25) - 6)
\end{bmatrix}$

\vspace{5mm}

$H_{1\text{x}1} = -\frac{600 (30 - x_1)^2 (\sqrt{(30 - x_1)^2 + x_2^2} - 30)}{((30 - x_1)^2 + x_2^2)^{3/2}} +
\frac{600 (30 - x_1)^2}{(30 - x_1)^2 + x_2^2} +
\frac{600 (\sqrt{(30 - x_1)^2 + x_2^2} - 30)}{\sqrt{(30 - x_1)^2 + x_2^2}} +
\frac{900 (\sqrt{(x_1 + 30)^2 + x_2^2} - 30)}{\sqrt{(x_1 + 30)^2 + x_2^2}}$

\hspace{3em}
$-\frac{900 (x_1 + 30)^2 (\sqrt{(x_1 + 30)^2 + x_2^2} - 30)}{((x_1 + 30)^2 + x_2^2)^{3/2}} +
\frac{900 (x_1 + 30)^2}{(x_1 + 30)^2 + x_2^2}$

\vspace{5mm}

$H_{1\text{x}2} = \frac{600 (30 - x_1) x_2 (\sqrt{(30 - x_1)^2 + x_2^2} - 30)}{((30 - x_1)^2 + x_2^2)^{3/2}} -
\frac{900 (x_1 + 30) x_2 (\sqrt{(x_1 + 30)^2 + x_2^2} - 30)}{((x_1 + 30)^2 + x_2^2)^{3/2}} -
\frac{600 (30 - x_1) x_2}{(30 - x_1)^2 + x_2^2} +
\frac{900 (x_1 + 30) x_2}{(x_1 + 30)^2 + x_2^2}$

\vspace{5mm}

$H_{2\text{x}1} = \frac{600 (30 - x_1) x_2 (\sqrt{(30 - x_1)^2 + x_2^2} - 30)}{((30 - x_1)^2 + x_2^2)^{3/2}} -
\frac{900 (x_1 + 30) x_2 (\sqrt{(x_1 + 30)^2 + x_2^2} - 30)}{((x_1 + 30)^2 + x_2^2)^{3/2}} -
\frac{600 (30 - x_1) x_2}{(30 - x_1)^2 + x_2^2} +
\frac{900 (x_1 + 30) x_2}{(x_1 + 30)^2 + x_2^2}$

\vspace{5mm}

$H_{2\text{x}2} = -\frac{600 x_2^2 (\sqrt{(30 - x_1)^2 + x_2^2} - 30)}{((30 - x_1)^2 + x_2^2)^{3/2}} -
\frac{900 x_2^2 (\sqrt{(x_1 + 30)^2 + x_2^2} - 30)}{((x_1 + 30)^2 + x_2^2)^{3/2}} +
\frac{600 x_2^2}{(30 - x_1)^2 + x_2^2} +
\frac{900 x_2^2}{(x_1 + 30)^2 + x_2^2} +
\frac{600 (\sqrt{(30 - x_1)^2 + x_2^2} - 30)}{\sqrt{(30 - x_1)^2 + x_2^2}}$

\hspace{3em}
$+\frac{900 (\sqrt{(x_1 + 30)^2 + x_2^2} - 30)}{\sqrt{(x_1 + 30)^2 + x_2^2}}$

\vspace{5mm}
Definição da função, gradiente, Hessiana no código principal :

\begin{python}
  def f(Xn):
    return 450 *((np.sqrt((30 + Xn[0])**2 + Xn[1]**2) - 30 )**2) + 300 *((np.sqrt((30 - Xn[0])**2 + Xn[1]**2) - 30)**2) - 360*Xn[1]

  def grad_f(Xn):
      return np.array([(900*(Xn[0] + 30)*(np.sqrt((Xn[0] + 30)**2 + Xn[1]**2) - 30))/np.sqrt((Xn[0] + 30)**2 + Xn[1]**2) - (600*(30 - Xn[0])*(np.sqrt((Xn[0] - 30)**2 + Xn[1]**2) - 30))/np.sqrt((Xn[0] - 30)**2 + Xn[1]**2),
                      60*(Xn[1]*(-450/np.sqrt(Xn[0]**2 + 60*Xn[0] + Xn[1]**2 + 900) - 300/np.sqrt(Xn[0]**2 - 60*Xn[0] + Xn[1]**2 + 900) + 25) - 6)])
      
  def hessian_f(Xn):
      hessian = np.zeros((2,2))
      hessian[0, 0] = -(600*(30 - Xn[0])**2*(np.sqrt((30 - Xn[0])**2 + Xn[1]**2) - 30))/((30 - Xn[0])**2 + Xn[1]**2)**(3/2) + \
      (600*(30 - Xn[0])**2)/((30 - Xn[0])**2 + Xn[1]**2) + \
      (600*(np.sqrt((30 - Xn[0])**2 + Xn[1]**2) - 30))/np.sqrt((30 - Xn[0])**2 + Xn[1]**2) + \
      (900*(np.sqrt((Xn[0] + 30)**2 + Xn[1]**2) - 30))/np.sqrt((Xn[0] + 30)**2 + Xn[1]**2) - \
      (900*(Xn[0] + 30)**2*(np.sqrt((Xn[0] + 30)**2 + Xn[1]**2) - 30))/((Xn[0] + 30)**2 + Xn[1]**2)**(3/2) + \
      (900*(Xn[0] + 30)**2)/((Xn[0] + 30)**2 + Xn[1]**2)
      
      hessian[0, 1] = (600*(30 - Xn[0])*Xn[1]*(np.sqrt((30 - Xn[0])**2 + Xn[1]**2) - 30))/((30 - Xn[0])**2 + Xn[1]**2)**(3/2) - \
      (900*(Xn[0] + 30)*Xn[1]*(np.sqrt((Xn[0] + 30)**2 + Xn[1]**2) - 30))/((Xn[0] + 30)**2 + Xn[1]**2)**(3/2) - \
      (600*(30 - Xn[0])*Xn[1])/((30 - Xn[0])**2 + Xn[1]**2) + \
      (900*(Xn[0] + 30)*Xn[1])/((Xn[0] + 30)**2 + Xn[1]**2)
      
      hessian[1, 0] = (600*(30 - Xn[0])*Xn[1]*(np.sqrt((30 - Xn[0])**2 + Xn[1]**2) - 30))/((30 - Xn[0])**2 + Xn[1]**2)**(3/2) - \
      (900*(Xn[0] + 30)*Xn[1]*(np.sqrt((Xn[0] + 30)**2 + Xn[1]**2) - 30))/((Xn[0] + 30)**2 + Xn[1]**2)**(3/2) - \
      (600*(30 - Xn[0])*Xn[1])/((30 - Xn[0])**2 + Xn[1]**2) + \
      (900*(Xn[0] + 30)*Xn[1])/((Xn[0] + 30)**2 + Xn[1]**2)
      
      hessian[1, 1] = -(600*Xn[1]**2*(np.sqrt((30 - Xn[0])**2 + Xn[1]**2) - 30))/((30 - Xn[0])**2 + Xn[1]**2)**(3/2) - \
      (900*Xn[1]**2*(np.sqrt((Xn[0] + 30)**2 + Xn[1]**2) - 30))/((Xn[0] + 30)**2 + Xn[1]**2)**(3/2) + \
      (600*Xn[1]**2)/((30 - Xn[0])**2 + Xn[1]**2) + \
      (900*Xn[1]**2)/((Xn[0] + 30)**2 + Xn[1]**2) + \
      (600*(np.sqrt((30 - Xn[0])**2 + Xn[1]**2) - 30))/np.sqrt((30 - Xn[0])**2 + Xn[1]**2) + \
      (900*(np.sqrt((Xn[0] + 30)**2 + Xn[1]**2) - 30))/np.sqrt((Xn[0] + 30)**2 + Xn[1]**2)
      
      return hessian

  func = 3
\end{python}

\subsubsection{Resultados}

Definição do ponto inicial no código principal:
\begin{python}
  P0 = np.array([0.01, -0.1])
\end{python}

\textbf{Utilizei o seguinte controle numérico para resolução da questão 2(a):}
\begin{itemize}
  \item Número máximo de passos (ou iterações): 200
  \item Tolerância para convergência do gradiente: Sensibilidade com $10^{-5}$, $10^{-4}$ e $10^{-3}$
  \item Tolerância para convergência da busca unidirecional: $10^{-6}$
  \item $\Delta\alpha$ do passo constante: $10^{-2}$
\end{itemize}

\vspace{3mm}
\textbf{Principais resultados obtidos:}

\begin{table}[H]
  \begin{center}
    \begin{tabular}{c|c|c|c}
      \multicolumn{4}{c}{tol = $10^{-5}$}\\
      \hline
      \textbf{Método} & \textbf{\# Passos} & \textbf{Tempo(s)} & \textbf{$P_{min}$}\\
      \hline
      Univariante       & 200   & 0.07850 & $\{-0.20510911, 7.78899302\}^t$\\
      Powell            & 200   & 91.21617 & $\{-0.20510878, 7.78899254\}^t$\\
      Steepest Descent  & 20    & 0.00304 & $\{-0.20510889, 7.78899266\}^t$\\
      Fletcher-Reeves   & 200   & 0.02659 & $\{-0.20510878, 7.78899218\}^t$\\
      BFGS              & 200   & 0.03755 & $\{-0.20510893, 7.78899288\}^t$\\
      Newton-Raphson    & 200   & 0.03705 & $\{-0.2051089, 7.78899288\}^t$\\
    \end{tabular}
  \end{center}
  \caption{Resumo dos resultados obtidos na questão 2a para $x^0 = \{0.01,-0.1\}^t$ e tol = $10^{-5}$}
\end{table}

\begin{table}[H]
  \begin{center}
    \begin{tabular}{c|c|c|c}
      \multicolumn{4}{c}{tol = $10^{-4}$}\\
      \hline
      \textbf{Método} & \textbf{\# Passos} & \textbf{Tempo(s)} & \textbf{$P_{min}$}\\
      \hline
      Univariante       & 200   & 0.05557 & $\{-0.20510911, 7.78899302\}^t$\\
      Powell            & 9     & 92.83198 & $\{-0.20510884, 7.78899251\}^t$\\
      Steepest Descent  & 6     & 0.00090 & $\{-0.20510891, 7.78899276\}^t$\\
      Fletcher-Reeves   & 12    & 0.00123 & $\{-0.20510891, 7.78899332\}^t$\\
      BFGS              & 4     & 0.00324 & $\{-0.2051089,  7.78899268\}^t$\\
      Newton-Raphson    & 49    & 0.12737 & $\{-0.20510894, 7.78899296\}^t$\\
    \end{tabular}
  \end{center}
  \caption{Resumo dos resultados obtidos na questão 2a para $x^0 = \{0.01,-0.1\}^t$ e tol = $10^{-4}$}
\end{table}

\begin{table}[H]
  \begin{center}
    \begin{tabular}{c|c|c|c}
      \multicolumn{4}{c}{tol = $10^{-3}$}\\
      \hline
      \textbf{Método} & \textbf{\# Passos} & \textbf{Tempo(s)} & \textbf{$P_{min}$}\\
      \hline
      Univariante       & 9     & 0.04030 & $\{-0.20510877, 7.78899022\}^t$\\
      Powell            & 8     & 91.89477 & $\{-0.20510883, 7.78899446\}^t$\\
      Steepest Descent  & 5     & 0.00073 & $\{-0.20510863, 7.78899279\}^t$\\
      Fletcher-Reeves   & 10    & 0.00255 & $\{-0.2051088, 7.78899188\}^t$\\
      BFGS              & 4     & 0.00500 & $\{-0.20510894, 7.78899296\}^t$\\
      Newton-Raphson    & 3     & 0.00425 & $\{-0.20510893, 7.78899416\}^t$\\
    \end{tabular}
  \end{center}
  \caption{Resumo dos resultados obtidos na questão 2a para $x^0 = \{0.01,-0.1\}^t$ e tol = $10^{-3}$}
\end{table}

Pelas tabelas apresentadas acima, um aumento da tolerância da convergência do gradiente leva a uma redução drástica
do número de iterações e, aparentemente, sem grandes perdas de precisão no valor de ponto mínimo obtido. Para a 
tolerância de $10^{-5}$ apenas o método Steepest Descent convergiu, levando um total de 20 passos, enquanto os 
demais métodos usaram o máximo especificado de 200 iterações. Para uma tolerância de $10^{-4}$ apenas o método univariante
não convergiu em menos de 200 passos, enquanto que para uma tolerância de $10^{-3}$ todos os métodos convergiram em até 10
passos e com resultados satisfatórios.

Em termos de tempo de execução, o método de Powell é o que mais se destaca. Independente da tolerância, o mesmo leva por volta
de 90 segundos para convergir. Após depuração do código e dos resultados ficou claro que o motivo disso ocorrer é que, 
para essa função e ponto inicial, o método de Powell, no sexto passo global, terceiro do segundo ciclo, gera uma direção com módulo muito pequeno e que
que necessita de um $\alpha$(aproximadante 60000) muito grande para alcançar o mínimo. Isso faz com que o algoritmo leve o tempo apresentado.
Essa talvez seja uma grande indicação de que trabalhar com direções normalizadas na busca unidirecional
seja mais adequado, evitando esse tipo de problema que pode ocorrer com outros métodos também a depender da combimação 
dos parâmetros de entrada. Alterar o ponto inicial ajuda a resolver esse problema encontrado no método de Powell.

Para fins de apresentação dos demais resultados, irei utilizar como base o estudo feito com tolerância de $10^{-3}$.

\begin{figure}[H]
  \centering
  \includegraphics[scale=0.4]{figuras/q2a_passos_P1.PNG}
  \caption{Número de passos por método OSR. Questão 2a e $x^0 = \{0.01,-0.1\}^t$}
\end{figure}

\begin{figure}[H]
  \centering
  \includegraphics[scale=0.45]{figuras/q2a_fxpassos_P1.PNG}
  \caption{Gráficos de $f(x_1,x_2)$ versus passo da minimização, por método. Questão 2a com $x^0 = \{0.01,-0.1\}^t$ e tol = $10^{-3}$}
\end{figure}

Seguem, abaixo, curvas de nível e caminhos obtidos pelos métodos para $x^0 = \{0.01,-0.1\}^t$ e tol = $10^{-3}$. 

\begin{figure}[H]
  \centering
  \begin{subfigure}[b]{\textwidth}
    \includegraphics[width=0.49\textwidth]{figuras/Q2.a_Univariante_P0=[0.01e-0.1].pdf}
    \includegraphics[width=0.49\textwidth]{figuras/Q2.a_Powell_P0=[0.01e-0.1].pdf}
  \end{subfigure}
  \caption{Curvas de nível e os pontos obtidos pelo Univariante e Powell. Questão 2a com $x^0 = \{0.01,-0.1\}^t$ e tol = $10^{-3}$}
\end{figure}

\begin{figure}[H]
  \centering
  \begin{subfigure}[b]{\textwidth}
    \includegraphics[width=0.49\textwidth]{figuras/Q2.a_Steepest Descent_P0=[0.01e-0.1].pdf}
    \includegraphics[width=0.49\textwidth]{figuras/Q2.a_Fletcher-Reeves_P0=[0.01e-0.1].pdf}
  \end{subfigure}
  \caption{Curvas de nível e os pontos obtidos pelo Steepest Descent e Fletcher-Reeves. Questão 2a com $x^0 = \{0.01,-0.1\}^t$ e tol = $10^{-3}$}
\end{figure}

\begin{figure}[H]
  \centering
  \begin{subfigure}[b]{\textwidth}
    \includegraphics[width=0.49\textwidth]{figuras/Q2.a_BFGS_P0=[0.01e-0.1].pdf}
    \includegraphics[width=0.49\textwidth]{figuras/Q2.a_Newton Raphson_P0=[0.01e-0.1].pdf}
  \end{subfigure}
  \caption{Curvas de nível e os pontos obtidos pelo BFGS e Newton-Raphson. Questão 2a com $x^0 = \{0.01,-0.1\}^t$ e tol = $10^{-3}$}
\end{figure}

Abaixo seguem as curvas de nível para um ponto inicial diferente do proposto no enuncado. $x^0 = \{-2,10\}^t$

\begin{figure}[H]
  \centering
  \begin{subfigure}[b]{\textwidth}
    \includegraphics[width=0.49\textwidth]{figuras/Q2.a_Univariante_P0=[-2 10].pdf}
    \includegraphics[width=0.49\textwidth]{figuras/Q2.a_Powell_P0=[-2 10].pdf}
  \end{subfigure}
  \caption{Curvas de nível e os pontos obtidos pelo Univariante e Powell. Questão 2a e $x^0 = \{-2,10\}^t$ e tol = $10^{-3}$}
\end{figure}

\begin{figure}[H]
  \centering
  \begin{subfigure}[b]{\textwidth}
    \includegraphics[width=0.49\textwidth]{figuras/Q2.a_Steepest Descent_P0=[-2 10].pdf}
    \includegraphics[width=0.49\textwidth]{figuras/Q2.a_Fletcher-Reeves_P0=[-2 10].pdf}
  \end{subfigure}
  \caption{Curvas de nível e os pontos obtidos pelo Steepest Descent e Fletcher-Reeves. Questão 2a e $x^0 = \{-2,10\}^t$ e tol = $10^{-3}$}
\end{figure}

\begin{figure}[H]
  \centering
  \begin{subfigure}[b]{\textwidth}
    \includegraphics[width=0.49\textwidth]{figuras/Q2.a_BFGS_P0=[-2 10].pdf}
    \includegraphics[width=0.49\textwidth]{figuras/Q2.a_Newton Raphson_P0=[-2 10].pdf}
  \end{subfigure}
  \caption{Curvas de nível e os pontos obtidos pelo BFGS e Newton-Raphson. Questão 2a e $x^0 = \{-2,10\}^t$ e tol = $10^{-3}$}
\end{figure}

\subsection{Questão 2 (b)}
\subsubsection{Enunciado}
Desenvolver um estudo de convergência da solução do deslocamento do ponto A, do sistema de molas, para níveis crescentes
de discretização do modelo (ou seja, considerando o número de molas n = 2,4,6...). A rigidez de cada mola ($k_i$, i =1,...n)
é obtida como a razão entre o módulo de rigidez axial do material e o seu comprimento. Os valores $W_j$ (com j = 1,...n-1)
correspondem às cargas nodais equivalentes aos pesos das molas.

\subsubsection{Formulação}

Para a generalização do problema do sistema de molas, foi considerado que sempre existirá um número par de molas.

$n_{nodes} = n_{molas} -1$

Como cada nó possui duas variáveis (Deslocamento horizontal $(u_i)$ e Deslocamento vertical $(v_i)$), o número de dimensões
será sempre $n_{dimens} = 2n_{nodes}$.

Usando a mesma ideia do equacionamento dos comprimentos finais de cada mola para o sistema com apenas 2 molas, e
extrapolando para $n_{dimens}$, podemos escrever o seguinte :

\vspace{3mm}
Seja $Li_m = $ comprimeto inicial da mola m, com m=1,2,...$n_{molas}$

\vspace{3mm}
Seja $Lf_m = $ comprimeto final da mola m, com m=1,2,...$n_{molas}$

\vspace{3mm}
Apenas para facilidade do equacionamento e analogia com a formulação para 2 molas, 
considerei que cada nó livre (total $n_{molas} - 1$ nós livres) terá deslocamentos positivos tanto em $x$ quanto em $y$. Ou seja,
cada nó, após equilíbrio, estará à direita do seu ponto inicial e abaixo (considerando eixo $y$ positivo para baixo) do nó livre anterior.

\vspace{3mm}
Então:

\vspace{3mm}
$Lf_m = \sqrt{(Li_m + u_m - u_{m-1})^2 + (v_m - v_{m-1})^2}$, com $u_0 = v_0 = u_{n_{molas}} = v_{n_{molas}} = 0 $ e m=1,2,...$n_{molas}$

\vspace{3mm}
$U_m = \frac{1}{2}K_m\Delta L_m^2$, sendo $K_m = \frac{EA_m}{Li_m}$ e $\Delta L_m = Lf_m - Li_m$ = Energia Interna de deformação

\vspace{3mm}
$W_n = \frac{1}{2}(\rho_n Li_n + \rho_{n+1} Li_{n+1})$, com n=1,2,...$n_{nodes}$

\vspace{3mm}
$V_n = W_nv_n$ = Trabalho das forças externas

\vspace{3mm}
Finalmente :

\vspace{3mm}
$\Pi = \sum_{m=1}^{n_{molas}}{U_m} - \sum_{n=1}^{n_{nodes}}{V_n}$ = Energia Potencial Total

\vspace{3mm}
Para o cálculo do $\overrightarrow{\nabla} \Pi$, o seguinte raciocínio foi adotado :

A variável $u_n$ aparece apenas nos termos $U_n$ e $U_{n+1}$, enquanto $v_n$ aparece em $U_n$, $U_{n+1}$ e $V_n$.

Dessa forma $\frac{\partial \Pi}{\partial u_n} = \frac{\partial U_n}{\partial u_n} + \frac{\partial U_{n+1}}{\partial u_n}$ e 
$\frac{\partial \Pi}{\partial v_n} = \frac{\partial U_n}{\partial v_n} + \frac{\partial U_{n+1}}{\partial v_n} -
\frac{\partial V_n}{\partial v_n} $. Todos esses termos são possíveis de se calcular analiticamente usando as expressões
apresentadas acima para $Lf_m$, $U_m$ e $V_n$, e com isso conseguimos uma expressão para cálculo do gradiente da função e que
foi implementada no meu código.

O mesmo raciocínio poderia ser aplicado para cálculo da Hessiana da função, porém, como sua utilidade fica restrita ao
método de Newton-Raphson, para fins desse trabalho usei um pacote pronto para cálculo diferencial de forma numérica no Python 
(numdifftools) com resultados bem satisfatórios e coincidentes com todos os cálculos dos métodos
analíticos de gradiente e Hessiana implementados para as funções da questão 1 e questão 2a.

Na implementação, o parâmetro que controla o número de molas do sistema é o número de dimensões do ponto inicial. Ou seja,
para um sistema 2 molas, 1 nó livre, é necessário fornecer $x^0$ com 2 dimensões ($x^0 = \{u_1, v_1\}^t$). Para um
sistema 4 molas, 3 nós livres, é necessar informar $x^0 = \{u_1, v_1, u_2, v_2, u_3, v_3\}^t$, e assim por diante.

\vspace{5mm}
Definição da função, gradiente, Hessiana no código principal :

\begin{python}
  def f(Xn):
    dimens = Xn.size
    #numero de nos
    n = int(dimens/2)

    #numero de molas
    m = n + 1

    #Inicializacao dos vetores com as variaveis do problema
    Li = np.zeros(m, dtype=float) # comprimentos iniciais das molas
    EA = np.zeros(m, dtype=float)
    RHO = np.zeros(m, dtype=float)
    W = np.zeros(n, dtype=float) # peso em cada no
    
    #Atribuicao dos valores do problema
    #Cada mola mede inicialmente 60/n_molas
    #molas a esquerda possuem EA = 27000 e rho 8
    #molas a direita possuem EA = 18000 e rho 16
    Li = Li + 60/m

    EA[ : int(m/2)] = 27000
    EA[int(m/2) : ] = 18000
    RHO[ : int(m/2)] = 8
    RHO[int(m/2) : ] = 16
    
    #Calculo dos pesos atuando em cada no
    # W[j] = (1/2)*(RHO[j]*Li[j] + RHO[j+1]*Li[j+1])
    RHO_e = RHO[:m-1]
    RHO_d = RHO[1:m]
    Li_e = Li[:m-1]
    Li_d = Li[1:m]
    W = (1/2)*(RHO_e*Li_e + RHO_d*Li_d)
    
    Lf = np.zeros(m, dtype=float) # comprimentos finais das molas
    U = np.zeros(m, dtype=float) # energia elastica das molas0.01, -0.1    
    V = np.zeros(n, dtype=float) # trabalho em cada no (desloc vert)  
      
    #array com os deslocamentos horizontais do Xn
    dx = Xn[0::2].copy()
    #array com os deslocamentos verticais do Xn
    dy = Xn[1::2].copy()
       
    #Calculo dos comprimentos finais
    # Lf[0] = np.sqrt( (Li[0] + dx[0] )**2 + dy[0]**2 ) 
    # Lf[k] = np.sqrt(a**2 + b**2)
    # Lf[m-1] = np.sqrt((Li[m-1] - dx[n-1])**2 + dy[n-1]**2)
    dx_d = np.zeros(m, dtype= float)
    dx_d[1:] = dx.copy()    
    dx_e = np.zeros(m, dtype=float)
    dx_e[:m-1] = dx.copy()
    dy_d = np.zeros(m, dtype= float)
    dy_d[1:] = dy.copy()
    dy_e = np.zeros(m, dtype=float)
    dy_e[:m-1] = dy.copy()
    
    a = Li + dx_e - dx_d
    b = dy_e - dy_d
    
    Lf = np.sqrt(a**2 + b**2)
    
    #calculo da energia elastica em cada mola
    U = (1/2)*(EA/Li)*((Lf - Li)**2)
    
    #calculo do trabalho em cada no    
    V = W*dy
    
    #Calculo da Energia Total
    E = np.sum(U) - np.sum(V)
    
    return E

  def grad_f(Xn):
      dimens = Xn.size
      #numero de nos
      n = int(dimens/2)

      #numero de molas
      m = n + 1

      #Inicializacao dos vetores com as variaveis do problema
      Li = np.zeros(m, dtype=float) # comprimentos iniciais das molas
      EA = np.zeros(m, dtype=float)
      RHO = np.zeros(m, dtype=float)
      W = np.zeros(n, dtype=float) # peso em cada no
      
      #Atribuicao dos valores do problema
      #Cada mola mede inicialmente 60 sobre numero de molas
      #molas a esquerda possuem EA = 27000 e rho 8
      #molas a direita possuem EA = 18000 e rho 16
      Li = Li + 60/m

      EA[ : int(m/2)] = 27000
      EA[int(m/2) : ] = 18000
      RHO[ : int(m/2)] = 8
      RHO[int(m/2) : ] = 16
      
      #Calculo dos pesos atuando em cada no
      # W[j] = (1/2)*(RHO[j]*Li[j] + RHO[j+1]*Li[j+1])
      RHO_e = RHO[:m-1]
      RHO_d = RHO[1:m]
      Li_e = Li[:m-1]
      Li_d = Li[1:m]
      W = (1/2)*(RHO_e*Li_e + RHO_d*Li_d)

      Lf = np.zeros(m, dtype=float) # comprimentos finais das molas
          
      #array com os deslocamentos horizontais do Xn
      dx = Xn[0::2].copy()
      #array com os deslocamentos verticais do Xn
      dy = Xn[1::2].copy()
      
      #Calculo dos comprimentos finais
      # Lf[0] = np.sqrt( (Li[0] + dx[0] )**2 + dy[0]**2 ) 
      # Lf[k] = np.sqrt(a**2 + b**2)
      # Lf[m-1] = np.sqrt((Li[m-1] - dx[n-1])**2 + dy[n-1]**2)
      dx_d = np.zeros(m, dtype= float)
      dx_d[1:] = dx.copy()
      
      dx_e = np.zeros(m, dtype=float)
      dx_e[:m-1] = dx.copy()
      dy_d = np.zeros(m, dtype= float)
      dy_d[1:] = dy.copy()
      dy_e = np.zeros(m, dtype=float)
      dy_e[:m-1] = dy.copy()
      
      a = Li + dx_e - dx_d
      b = dy_e - dy_d
      
      Lf = np.sqrt(a**2 + b**2)
      
      #calculo gradiente
      gradx = np.zeros(n, dtype=float)
      grady = np.zeros(n, dtype=float)
      Li_e = np.zeros(n, dtype=float)
      Li_d = np.zeros(n, dtype=float)
      EA_e = np.zeros(n, dtype=float)
      EA_d = np.zeros(n, dtype=float)
      Lf_e = np.zeros(n, dtype=float)
      Lf_d = np.zeros(n, dtype=float)
      
      Li_e = Li[:m-1]
      Li_d = Li[1:m]
      EA_e = EA[:m-1]
      EA_d = EA[1:m]
      Lf_e = Lf[:m-1]
      Lf_d = Lf[1:m]
      
      dx_d = np.zeros(n, dtype=float)   
      dx_d[1:] = dx[:m-2]
      
      dy_d = np.zeros(n, dtype=float)  
      dy_d[1:] = dy[:m-2]
      
      dx_e = np.zeros(n, dtype=float)
      dx_e[:m-2] = dx[1:]
      
      dy_e = np.zeros(n, dtype=float)
      dy_e[:m-2] = dy[1:]
      
      deriv1 = np.zeros(n, dtype=float)
      deriv2 = np.zeros(n, dtype=float)
      deriv3 = np.zeros(n, dtype=float)
      deriv4 = np.zeros(n, dtype=float)
      deriv5 = np.zeros(n, dtype=float)
        
      deriv1 = (1/2)*((Li_e + dx - dx_d)**2 + (dy - dy_d)**2)**(-1/2)*(2*Li_e + 2*dx - 2*dx_d)        
      deriv2 = (1/2)*((Li_d + dx_e - dx)**2 + (dy_e - dy)**2)**(-1/2)*(-2*Li_d - 2*dx_e + 2*dx)    
      deriv3 = (1/2)*((Li_e + dx - dx_d)**2 + (dy - dy_d)**2)**(-1/2)*(2*dy - 2*dy_d)
      deriv4 = (1/2)*((Li_d + dx_e - dx)**2 + (dy_e - dy)**2)**(-1/2)*(-2*dy_e + 2*dy)
      deriv5 = W    
      
      # dUk/dxk + dU(k+1)/dxk
      gradx = (1/2)*(EA_e/Li_e)*2*(Lf_e - Li_e)*deriv1 + (1/2)*(EA_d/Li_d)*2*(Lf_d - Li_d)*deriv2
      
      #dUk/dyk + dU(k+1)/dyk - dVk/dyk
      grady = (1/2)*(EA_e/Li_e)*2*(Lf_e - Li_e)*deriv3 + (1/2)*(EA_d/Li_d)*2*(Lf_d - Li_d)*deriv4 - deriv5
      
      grad = np.zeros(2*n, dtype=float)
      grad[0::2] = gradx
      grad[1::2] = grady
          
      return grad

  def hessian_f(Xn):
      return nd.Hessian(f)(Xn)

  func = 4

\end{python}


\subsubsection{Resultados}

Para esse exercício, variei o número de molas no sistema de 2 a 20, e com isso precisei de
diferentes pontos iniciais para as rodadas dos métodos. Por simplificação, usei os mesmos delocamentos iniciais para 
todos os nós livres. Valores utilizados : $u_i = -1$ e $v_i = 5$, $\forall i$ com $i=1,2,...n_{nodes}$ 

Exemplo de definição do ponto inicial com a premissa acima para o caso 4 molas:

\begin{python}
  P0 = np.array([-1, 5, -1, 5, -1, 5])
\end{python}

\textbf{Utilizei o seguinte controle numérico para resolução da questão 2(b):}
\begin{itemize}
  \item Número máximo de passos (ou iterações): 200
  \item Tolerância para convergência do gradiente: $10^{-3}$
  \item Tolerância para convergência da busca unidirecional: $10^{-6}$
  \item $\Delta\alpha$ do passo constante: $10^{-2}$
\end{itemize}

O gráfico abaixo mostra o número de passos por método e por discretizaçao do número de molas. Nele é possível
constatar que os métodos Univariante e Steepest Descent já não conseguem convergir para um sistema de 4 molas em diante.
O método de Powell até converge para 4 molas, mas a partir de 6 molas já não consegue mais convergir. O método Fletcher-Reeves,
na sensibilidade feita nesse trabalho comvergiu até 10 molas, e no caso com 20 molas não convergiu, apesar de ter chegado
bem próximo dos resultados dos métodos que convergiram. Os métodos BFGS e
Newton-Raphson convergiram para todos os casos, com um número relativamente baixo de passos.

\begin{figure}[H]
  \centering
  \includegraphics[scale=0.35]{figuras/q2b_passos.png}
  \caption{Número de passos, por método OSR, para diferentes números de molas. Questão 2b.}
\end{figure}

As tabelas abaixo resumem o estudo de convergência do valor para o deslocamento final do Ponto $A$, representado pelo
vetor $\{u_A, v_a\}^t$. Pode-se notar que os métodos que não convergem, apresentam resultados divergentes dos que convergiram,
como esperado. Também é possível notar a diferença entre posição prevista para o ponto $A$ caso a discretização seja muito
pequena, 2 molas por exemplo, e caso seja muito alta, 20 molas por exemplo, lembrando sempre de olhar o resultado do BFGS
e Newton-Raphson para essa comparação, dado que foram os métodos que atingiram convergência para todas as discretizações.

\begin{table}[H]
  \begin{center}
    \begin{tabular}{c|c|c|c}
      \multicolumn{4}{c}{Deslocamento do Ponto $A$ (nó central)}\\
      \hline
      \textbf{Método} & \textbf{\# 2 molas} & \textbf{4 molas} & \textbf{6 molas} \\
      \hline
      Univariante       & $\{-0.2051, 7.7890\}^t$ &	$\{-0.0867, 7.1702\}^t$ &	$\{-0.1927, 7.1189\}^t$ \\
      Powell	          & $\{-0.2051, 7.7890\}^t$ &	$\{-0.0863, 7.1700\}^t$ &	$\{-0.3735, 6.8136\}^t$ \\
      Steepest Descent  & $\{-0.2051, 7.7890\}^t$ &	$\{-0.0860, 7.1676\}^t$ &	$\{-0.0725, 6.9551\}^t$ \\
      Fletcher-Reeves   & $\{-0.2051, 7.7890\}^t$ &	$\{-0.0863, 7.1700\}^t$ &	$\{-0.0692, 7.0765\}^t$ \\
      BFGS	            & $\{-0.2051, 7.7890\}^t$ &	$\{-0.0863, 7.1700\}^t$ &	$\{-0.0692, 7.0765\}^t$ \\
      Newton-Raphson    & $\{-0.2051, 7.7890\}^t$ &	$\{-0.0863, 7.1700\}^t$ &	$\{-0.0692, 7.0765\}^t$
    \end{tabular}
  \end{center}
  \caption{Resumo dos resultados obtidos na questão 2b}
\end{table}

\begin{table}[H]
  \begin{center}
    \begin{tabular}{c|c|c|c}
      \multicolumn{4}{c}{Deslocamento do Ponto $A$ (nó central)}\\
      \hline
      \textbf{Método}   & \textbf{8 molas} & \textbf{10 molas} & \textbf{20 molas} \\
      \hline
      Univariante       & $\{-0.4678, 7.1400\}^t$	& $\{-0.6909, 7.2969\}^t$ &	$\{-1.0657, 6.7105\}^t$ \\
      Powell	          & $\{-0.5929, 7.5628\}^t$	& $\{-0.7793, 6.6867\}^t$ &	$\{-1.0733, 6.7864\}^t$ \\
      Steepest Descent  & $\{-0.1084, 6.6119\}^t$	& $\{-0.1652, 6.2609\}^t$ &	$\{-0.5561, 5.4032\}^t$ \\
      Fletcher-Reeves   & $\{-0.0635, 7.0450\}^t$	& $\{-0.0610, 7.0306\}^t$ &	$\{-0.0575, 7.0123\}^t$ \\
      BFGS	            & $\{-0.0635, 7.0450\}^t$	& $\{-0.0610, 7.0306\}^t$ &	$\{-0.0575, 7.0116\}^t$ \\
      Newton-Raphson    & $\{-0.0635, 7.0450\}^t$	& $\{-0.0610, 7.0306\}^t$ &	$\{-0.0575, 7.0116\}^t$
    \end{tabular}
  \end{center}
  \caption{Resumo dos resultados obtidos na questão 2b}
\end{table}

A figura abaixo resume o tempo de execução de cada método para cada discretização do número de molas.
Importante destacar o tempo maior para os métodos de Powell e Newton-Raphson. O tempo do método de Newton-Raphson 
acredito que possa ter influência do uso do pacote numdifftools para cálculo da Hessiana, e o tempo para cálculo de sua inversa.
Porém, o método de Powell sofreu novamente, principalmente no caso 4 molas, com uma direção com módulo pequeno e
consequente $\alpha$ grande do passo constante, \textbf{reforçando novamente a ideia de se normalizar as direções na busca unidirecional
 para evitar problemas desse tipo.}

\begin{figure}[H]
  \centering
  \includegraphics[scale=0.4]{figuras/q2b_tempo.png}
  \caption{Tempo de Execução, por método OSR, para diferentes números de molas. Questão 2b.}
\end{figure}

A figura abaixo mostra o comportamento do valor da função a cada iteração dos métodos para o caso
com 20 molas, evidenciando o quão rápido Flecther-Reeves,
BFGS e Newton-Raphson se aproximam do mínimo.  

Um detalhe que também aparece, mas está sutil e pode passar despercebido dada a escala em $y$ utilizada,
é que o primeiro passo do Newton-Raphson está aumentando consideravelmente o valor da função. Após investigação do resultado,
notei que esse primeiro passo está usando uma direção com módulo muito grande, da ordem de $5.10^9$ e retornando um 
$\alpha$ da seção áurea da ordem de $3.10^{-7}$. Como o passo usado é $10^{-2}$, é possível constatar que o $\Delta \alpha$
absoluto, ao invés de ficar próximo ao ${10^{-2}}$ desejado, atinge valores acima de $1$, dadas as ordens de grandeza dos
parâmetros. Pode ser que o problema seja o cálculo da Hessiana nesse primeiro passo, apesar de que para os demais passos,
o Newton-Raphson se comporta adequadamente. \textbf{Esse exemplo é mais um reforço para que seja usado direções normalizadas.}

\begin{figure}[H]
  \centering
  \includegraphics[scale=0.45]{figuras/q2b_fxpassos_20molas.PNG}
  \caption{Gráficos de $f(x_1,x_2)$ versus passo da minimização, por método. Questão 2b com 20 molas}
\end{figure}

\begin{figure}[H]
  \centering
  \begin{subfigure}[b]{\textwidth}
    \includegraphics[width=0.49\textwidth]{figuras/q2bUnivGraf.png}
    \includegraphics[width=0.49\textwidth]{figuras/q2bPowellGraf.png}
  \end{subfigure}
  \begin{subfigure}[b]{\textwidth}
    \includegraphics[width=0.49\textwidth]{figuras/q2bSteepDGraf.png}
    \includegraphics[width=0.49\textwidth]{figuras/q2bFRGraf.png}
  \end{subfigure}
  \begin{subfigure}[b]{\textwidth}
    \includegraphics[width=0.49\textwidth]{figuras/q2bBFGSGraf.png}
    \includegraphics[width=0.49\textwidth]{figuras/q2bNRGraf.png}
  \end{subfigure}
  \caption{Posição dos nós, por método, para diferentes números de molas}
\end{figure}

\end{document}