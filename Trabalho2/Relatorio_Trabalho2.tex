\documentclass[10pt, a4paper]{article}
% \usepackage[english]{babel}
\usepackage[brazilian]{babel}
\usepackage[utf8]{inputenc}
% \usepackage[T1]{fontenc}
\usepackage{lipsum}

% matlab code
% \usepackage{matlab-prettifier}
%\usepackage[numbered,framed]{matlab-prettifier}
\usepackage{pythonhighlight}
\renewcommand{\lstlistingname}{Anexo} % Listing->Code
\let\ph\mlplaceholder % shorter macro
\definecolor{codegreen}{rgb}{0,0.6,0}
\definecolor{codegray}{rgb}{0.5,0.5,0.5}
\definecolor{codepurple}{rgb}{0.58,0,0.82}
\definecolor{backcolour}{rgb}{0.95,0.95,0.92}
\lstdefinestyle{myStyle}{
    language=Matlab,
    breaklines=true,
    frame=single,
    numbers=none,
    basicstyle=\ttfamily\footnotesize,
%     basicstyle=\footnotesize\ttfamily,
    keywordstyle=\bfseries\color{magenta},
    commentstyle=\color{codegreen},
    identifierstyle=\color{blue},
    backgroundcolor=\color{backcolour},
    stringstyle=\color{codepurple},
}
\usepackage{adjustbox}

% For subfigure use
\usepackage[font=small,labelfont=bf]{caption}
\usepackage{subcaption}

% Set page size and margins
% Replace `letterpaper' with`a4paper' for UK/EU standard size
\usepackage[a4paper,top=2cm,bottom=2cm,left=2cm,right=2cm,marginparwidth=2cm]{geometry}

% tabelas
\usepackage{array}
\usepackage{tabularx}
\usepackage{booktabs}

\usepackage{float}

% Useful packages
\usepackage{amsmath}

\usepackage{graphicx}
%\graphicspath{{figures/}} %Setting the graphicspath
\usepackage[colorlinks=true, allcolors=blue]{hyperref}
\usepackage{cleveref}
\newcommand{\crefrangeconjunction}{--}
\DeclareMathOperator{\sen}{sen}


\begin{document}

\def\TITLE{Trabalho 2}
\def\SUBTITLE{Opção 01}
\def\DISCIPLINE{MEC 2403 - Otimização, Algoritmos e Aplicações na Engenharia Mecânica}
\def\PROFESSOR{Ivan Menezes}
\def\AUTHOR{Gustavo Henrique Gomes dos Santos}
\def\CONTACT{gustavohgs@gmail.com}
\def\DATE{junho de 2023}

\title{\textbf{\TITLE} \\ \DISCIPLINE}
\author{\AUTHOR}
\date{\DATE}

\begin{titlepage}
      \begin{center}
          \vspace*{1cm}

          \Huge
          \textbf{\TITLE}

          \Large
          \textbf{\SUBTITLE}

          \vspace{0.5cm}
          \LARGE
          \DISCIPLINE

          \vspace{1.5cm}

          \textbf{\AUTHOR \\ {\tt \CONTACT}}

          \vfill
          Professor: \PROFESSOR

          \vspace{0.8cm}

          \includegraphics[width=0.2\textwidth]{../general/puc.jpg}

          \Large
          Departamento de Engenharia Mecânica\\
          PUC-RJ Pontifícia Universidade Católica do Rio de Janeiro\\
          \DATE

      \end{center}
  \end{titlepage}

\maketitle

\section{Introdução}

\subsection{Objetivos}

Esse trabalho tem como objetivo a implementação e teste dos seguintes métodos indiretos de otimização com restrição :

\begin{enumerate}
  \item Penalidade
  \item Barreira
\end{enumerate}

Para a etapa de sequência de otimização sem restrição, da qual esses métodos fazem uso, foram utilizados os métodos
implementados no trabalho 1. São eles:

\begin{enumerate}
  \item Univariante
  \item Powell
  \item Steepest Descent
  \item Fletcher-Reeves
  \item BFGS
  \item Newton-Raphson
\end{enumerate}

Estes métodos, por sua vez, fazem uso dos seguintes algoritmos de busca unidimensional também implementados em trabalhos anteriores :

\begin{enumerate}
  \item Passo constante
  \item Seção áurea
\end{enumerate}

\section{Implementação}

Foi utilizada a linguagem de programação Python para elaboração deste trabalho.
A implementação consiste de um arquivo com o código principal, um segundo arquivo com os algoritmos dos métodos de otimização com restrição,
um terceiro arquivo que encapsula os métodos e a convergência da otimização sem restrição
e um quarto arquivo com os algoritmos da busca unidimensional.

\subsection{Código Principal}

O código principal trata das definições do ponto inicial, função a ser minimizada 
e as restrições de igualdade e desigualdade.
A escolha de quais métodos OCR e OSR serão utilizados e a definição dos parâmetros numéricos também são feitos no código principal.
Além disso, o controle de passos da otimização OC e a verificação de convergência.

\vspace{5mm}
Os seguintes pacotes foram utilizados na implementação do código principal, já inclusos os arquivos .py com
as implementações dos métodos OSR, busca unidimensional e métodos OCR.

\begin{python}
  import numpy as np
  import matplotlib.pyplot as plt
  import osr_methods as osr
  import line_search_methods as lsm
  import ocr_methods as ocr  
\end{python}

\vspace{5mm}
Definição do ponto inicial, que irá variar em cada teste realizado com diferentes funções e método OCR.

\begin{python}
  x = np.array([3., 2.])
\end{python}

\vspace{5mm}
Escolha dos métodos de OCR e OSR.

\begin{python}
  # Metodos OCR
  # 1 - Penalidade
  # 2 - Barreira
  metodo_ocr = 1

  if (metodo_ocr == 1):
      n_met_ocr = "Penalidade"
  elif (metodo_ocr == 2):
      n_met_ocr = "Barreira"

  # Metodos OSR
  # 1 - Univariante
  # 2 - Powell
  # 3 - Stepest Descent
  # 4 - Newton-Raphson
  # 5 - Fletcher-Reeves
  # 6 - BFGS
  metodo_osr = 4

  if (metodo_osr == 1):
      n_met = 'Univariante'
  elif (metodo_osr == 2):
      n_met = 'Powell'
  elif (metodo_osr == 3):
      n_met = 'Steepest Descent'
  elif (metodo_osr == 4):
      n_met = 'Newton-Raphson'
  elif (metodo_osr == 5):
      n_met = 'Fletcher-Reeves'
  elif (metodo_osr == 6):
      n_met = 'BFGS'  
\end{python}

\vspace{5mm}
Controle numérico

\begin{python}
  # numero maximo de iteracoes na OSR
  maxiter = 1000

  # tolerancia para convergencia do gradiente na OSR
  tol_conv = 1E-6

  # tolerancia para a busca unidirecional na OSR
  tol_search = 1E-7

  # delta alpha do passo constante na OSR
  line_step = 1E-2

  #epsilon da maquina
  eps = 1E-10

  #parametros ocr
  if metodo_ocr == 1:
      #penalidade
      r = 1
      beta = 10
  elif metodo_ocr == 2:
      #barreira
      r = 10
      beta = 0.1

  #tolerancia OCR
  tol = 1E-6

  ctrl_num_osr = [maxiter, tol_conv, tol_search, line_step, eps]  
\end{python}

\vspace{5mm}
Definição de $f(\overrightarrow{x})$, $\overrightarrow{\nabla} f(\overrightarrow{x})$ e $\textbf{H}(\overrightarrow{x})$.
As reticências no código abaixo serão substituiídas pelos valores adequados a cada função.

\begin{python}
  def f(x):
    return ...

  def grad_f(x):
    return ...

  def hess_f(x):
    hess = np.zeros((2,2), dtype=float)
    hess[0,:] = ...
    hess[1,:] = ...
    return hess
\end{python}

\vspace{5mm}
Definição das restrições $c_l(\overrightarrow{x})$, $\overrightarrow{\nabla}c_l(\overrightarrow{x})$,
$h_k(\overrightarrow{x})$, $\overrightarrow{\nabla}h_k(\overrightarrow{x})$, $\textbf{W}_{c_l}(\overrightarrow{x})$ e
$\textbf{W}_{h_k}(\overrightarrow{x})$, sendo W a hessiana da restrição.
As reticências no código abaixo serão substituídas pelos valores adequados a cada função. Podem ser definidas quantas restrições
forem desejadas, apesar do exemplo abaixo estar com apenas uma de desigualdade e uma de igualdade.

\begin{python}

  def h1(x):
    return ...

  def grad_h1(x):
    return ...

  def hess_h1(x):
    hess = np.zeros((2,2), dtype=float)
    hess[0,:] = ...
    hess[1,:] = ...
    return hess

  def c1(x):
    return ...

  def grad_c1(x):
    return ...

  def hess_c1(x):
    hess = np.zeros((2,2), dtype=float)
    hess[0,:] = ...
    hess[1,:] = ...
    return hess
\end{python}

\vspace{5mm}
Agrupamento das restrições em listas(ou arrays) para servirem de input para o restante do programa. 
E montagem de um array auxiliar para montagem da função $\phi$ no caso do método OCR de penalidade.

\begin{python}
  h_list = [h1]
  grad_h_list = [grad_h1]
  hess_h_list = [hess_h1]

  c_list = [c1]
  grad_c_list = [grad_c1]
  hess_c_list = [hess_c1]

  #para o metodo de penalidade
  #controle de quais cls irao montar a phi
  c_mont = []
  if metodo_ocr == 1:
      for c in c_list:
          if c(x) > 0:
              c_mont.append(1)
          else:
              c_mont.append(0)
              
  params = [f, grad_f, hess_f, h_list, grad_h_list, hess_h_list, c_list, grad_c_list, hess_c_list, c_mont]
\end{python}

\vspace{5mm}
Verificação de convergência.

\begin{enumerate}
  \item Penalidade : $\frac{1}{2}r_p^kp(\overrightarrow{x}^{k+1}) < tol$, sendo $p(\overrightarrow{x}) = \sum_{k=1}^{m} h_k^2(\overrightarrow{x}) + \sum_{l=1}^{p}\{max[0,c_l(\overrightarrow{x})]\}^2$
  \item Barreira   : $r_b^kb(\overrightarrow{x}^{k+1}) < tol$, sendo $b = \sum_{l=1}^{p} -\frac{1}{c_l(\overrightarrow{x})}$
\end{enumerate}

A cada iteração do loop no código abaixo, o valor de $r$ é atualizado pelo fator $\beta$, e os parâmetros necessários para montagem
da função $\phi$ pelos métodos da penalidade e da barreira também são atualizados. Após cada atualização e verificação de convergência, 
a otimização sem restrição é chamada, e esse processo é repetido até que seja atingido o critério de convergência da otimização com
restrição.

Para o método da barreira também é feita uma avaliação
se o ponto final da iteração continua dentro das restrições. Caso negativo, o passo é refeito com um passo reduzido no método do passo
constante da busca unidimensional.



\begin{python}
  if metodo_ocr == 1:
    parc = (1/2)*r*ocr.p_penal(x, params)
  elif metodo_ocr == 2:
    parc = r*ocr.b_bar(x, params)
    
  listP_OCR = []
  listP_OCR.append(x)

  listResultsOSR = []

  passos_OCR = 0
  redo = 0
  print(n_met)
  while(parc > tol):
    passos_OCR = passos_OCR + 1
    if passos_OCR > 1:
        r = beta*r
        if metodo_ocr == 1:
            params[-1] = []
            for c in c_list:
                if c(x) > 0:
                    params[-1].append(1)
                else:
                    params[-1].append(0)
    listP_OSR, passos_OSR, conv_OSR, flag_conv_OSR, tempoExec_OSR = osr.osr_ctrl(x, params, r, ctrl_num_osr, metodo_ocr, metodo_osr)
    
    if metodo_ocr == 2:
        redo = 0
        for c in c_list:
            if c(listP_OSR[-1]) > 0:
                redo = 1
                break
    if (redo == 0):
        ctrl_num_osr[3] = line_step
        x = listP_OSR[-1]
        listP_OCR.append(x)
        listResultsOSR.append([listP_OSR, params, r, metodo_ocr, metodo_osr])
        if metodo_ocr == 1:
            parc = (1/2)*r*ocr.p_penal(x, params)
        elif metodo_ocr == 2:
            parc = r*ocr.b_bar(x, params)
        print(f'{passos_OCR}: x={x}, r={r:.4e}, passos={passos_OSR}, conv_OCR={parc:.4e}, conv_OSR={conv_OSR:.4e}, tempo={tempoExec_OSR}')
    elif (redo == 1):
        print(f'Refazendo passo {passos_OCR} com delta alpha = {0.1*ctrl_num_osr[3]}')
        passos_OCR = passos_OCR - 1
        r = r/beta
        ctrl_num_osr[3] = 0.1*ctrl_num_osr[3]    
\end{python}

\subsection{Métodos OCR}

Os algoritmos referentes ao uso dos métdos OCR da penalidade e da barreira foram implementados em um arquivo denominado ocr\_methods.py.

\vspace{3mm}
O seguinte pacote é necessário nesse arquivo:

\begin{python}
  import numpy as np
\end{python}

\subsubsection{Método de Penalidade}

\textbf{Pseudo-Função Objetivo:}

\vspace{3mm}
$\phi(\overrightarrow{x}, r_p) = f(\overrightarrow{x}) + \frac{1}{2} r_p \sum_{k=1}^{m} h_k^2(\overrightarrow{x}) 
+ \frac{1}{2} r_p \sum_{l=1}^{p} \{max[0, c_l(\overrightarrow{x})]\}^2$

\vspace{5mm}
\textbf{Cálculo do gradiente:}

\vspace{3mm}
Sendo $p(\overrightarrow{x}) = \sum_{k=1}^{m} h_k^2(\overrightarrow{x}) + \sum_{l=1}^{p}\{max[0,c_l(\overrightarrow{x})]\}^2$, temos então que
$\overrightarrow{\nabla} \phi(\overrightarrow{x}, r_p) = \overrightarrow{\nabla} f(\overrightarrow{x}) + 
\frac{1}{2} r_p \overrightarrow{\nabla} p(\overrightarrow{x})$.

\vspace{3mm}
$\overrightarrow{\nabla} p(\overrightarrow{x}) = 2\sum_{k=1}^{m} \{h_k(\overrightarrow{x}) \overrightarrow{\nabla} h_k(\overrightarrow{x}) \}
+ 2\sum_{l=1}^{p} \{c_{l}^{f}(\overrightarrow{x}) c_l(\overrightarrow{x}) \overrightarrow{\nabla} c_l(\overrightarrow{x}) \} $,
sendo $c_l^f(\overrightarrow{x}) =
\begin{cases}
  1,& \text{se } c_l(\overrightarrow{x}) > 0 \\
  0,              & \text{se } c_l(\overrightarrow{x}) \leq 0
\end{cases}$

\vspace{5mm}
\textbf{Cálculo da Hessiana:}

\vspace{3mm}
$H_{\phi}(\overrightarrow{x}) = H_f(\overrightarrow{x}) + \frac{1}{2} r_p H_p(\overrightarrow{x})$

\vspace{3mm}
$H_{p_{i\text{x}j}} = \frac{\partial ^2 p}{\partial x_i \partial x_j} = 
2\sum_{k=1}^{m} \{\frac{\partial h_k}{\partial x_j} \frac{\partial h_k}{\partial x_i}\} 
+ 2\sum_{k=1}^{m} \{h_k \frac{\partial ^2 h_k}{\partial x_j \partial x_i}\}
+2\sum_{l=1}^{p} \{c_l^f \frac{\partial c_l}{\partial x_j} \frac{\partial c_l}{\partial x_i}\}
+ 2\sum_{l=1}^{p} \{c_l^f c_l \frac{\partial ^2 c_l}{\partial x_j \partial x_i}\}$

\vspace{3mm}
Interessante notar que $\frac{\partial c_l}{\partial x_j}$,
$\frac{\partial c_l}{\partial x_i}$,
$\frac{\partial h_k}{\partial x_j}$ e
 $\frac{\partial h_k}{\partial x_i}$ são componentes conhecidos de 
$\overrightarrow{\nabla} h_k$ e $\overrightarrow{\nabla} c_l$. Além disso,
$\frac{\partial ^2 c_l}{\partial x_j \partial x_i}$ 
e $\frac{\partial ^2 h_k}{\partial x_j \partial x_i}$
são termos das hessianas das restrições e que também são conhecidos.
Dessa forma, temos todos os inputs necessários para cálculo da hessiana de 
$p(\overrightarrow{x})$ e por conseguinte a hessiana de 
$\phi(\overrightarrow{x}, r)$.

\vspace{3mm}
\begin{python}
  #Metodo da Penalidade
  def p_penal(x, params):
      #leitura dos parametros
      h_list = params[3]
      c_list = params[6]
      c_mont = params[9]
      
      p = 0
      for h in h_list:
          p = p + (h(x))**2
      
      for i in np.arange(len(c_list)):
          p = p + c_mont[i]*c_list[i](x)**2
          
      return  p

  def phi_penal(x, params, r):
      #leitura dos parametros
      f = params[0]
      h_list = params[3]
      c_list = params[6]
      c_mont = params[9]
      
      p = 0
      for h in h_list:
          p = p + (h(x))**2
      
      for i in np.arange(len(c_list)):
          p = p + c_mont[i]*c_list[i](x)**2
          
      return f(x) + (1/2)*r*p

  def grad_phi_penal(x, params, r):
      #leitura dos parametros
      grad_f = params[1]
      h_list = params[3]
      grad_h_list = params[4]
      c_list = params[6]
      grad_c_list = params[7]
      c_mont = params[9]
      
      dimens = x.size
      grad_p = np.zeros(dimens, dtype=float)
      
      for i in np.arange(len(h_list)):
          grad_p = grad_p + 2*h_list[i](x)*grad_h_list[i](x)
      for j in np.arange(len(c_list)):
          grad_p = grad_p + 2*c_mont[j]*c_list[j](x)*grad_c_list[j](x)
          
      return grad_f(x) + (1/2)*r*grad_p

  def hess_phi_penal(x, params, r):
      #leitura dos parametros
      hess_f = params[2]
      h_list = params[3]
      grad_h_list = params[4]
      hess_h_list = params[5]
      c_list = params[6]
      grad_c_list = params[7]
      hess_c_list = params[8]
      c_mont = params[9]    
      
      dimens = x.size    
      hessian_p = np.zeros((dimens, dimens), dtype=float)
      
      for i in np.arange(dimens):    
          for j in np.arange(dimens):
              for k in np.arange(len(grad_h_list)):
                  hessian_p[i,j] = hessian_p[i,j] + 2*grad_h_list[k](x)[i]*grad_h_list[k](x)[j]
              for l in np.arange(len(grad_cl_list)):
                  hessian_p[i,j] = hessian_p[i,j] + 2*c_mont[l]*grad_c_list[l](x)[j]*grad_c_list[l](x)[i]
      
      for k in np.arange(len(h_list)):
          hessian_p = hessian_p + 2*h_list[k](x)*hess_h_list[k](x)
      
      for k in np.arange(len(c_list)):
          hessian_p = hessian_p + 2*c_mont[k]*c_list[k](x)*hess_c_list[k](x)
      
      return hess_f(x) + (1/2)*r*hessian_p
\end{python}

\subsubsection{Método de Barreira}

\textbf{Pseudo-Função Objetivo:}

\vspace{3mm}
$\phi(\overrightarrow{x}, r_b) = f(\overrightarrow{x}) + 
r_b \sum_{l=1}^{m} -\frac{1}{c_l(\overrightarrow{x})}$

\vspace{5mm}
\textbf{Cálculo do gradiente:}

\vspace{3mm}
Sendo $b(\overrightarrow{x}) = \sum_{l=1}^{m} -\frac{1}{c_l(\overrightarrow{x})}$
, temos então que
$\overrightarrow{\nabla} \phi(\overrightarrow{x}, r_b) = 
\overrightarrow{\nabla} f(\overrightarrow{x}) + 
r_b \overrightarrow{\nabla} b(\overrightarrow{x})$.

\vspace{3mm}
$\overrightarrow{\nabla} b =
\sum_{l=1}^{m} \frac{\overrightarrow{\nabla} c_l}{c_l^2}$

\vspace{5mm}
\textbf{Cálculo da Hessiana:}

\vspace{3mm}
$H_{\phi}(\overrightarrow{x}) = H_f(\overrightarrow{x}) 
+ r_b H_b(\overrightarrow{x})$

\vspace{3mm}
$H_{b_{i\text{x}j}} = -2 \sum_{l=1}^{m} \{ \frac{1}{c_l^3}\frac{\partial c_l}{\partial x_i} \frac{\partial c_l}{\partial x_j} \}
+ \sum_{l=1}^{m} \{ \frac{1}{c_l^2} \frac{\partial ^2 c_l}{\partial x_i \partial x_j}            \}$

\vspace{3mm}
$\frac{\partial c_l}{\partial x_j}$ e
$\frac{\partial c_l}{\partial x_i}$ são componentes conhecidos de $\overrightarrow{\nabla} c_l$.
Além disso,
$\frac{\partial^2 c_l}{\partial x_j \partial x_i}$ são termos da hessiana das restrições e que também são conhecidos.
Dessa forma, temos todos os inputs necessários para cálculo da hessiana de 
$b(\overrightarrow{x})$ e por conseguinte a hessiana de 
$\phi(\overrightarrow{x}, r)$.

\begin{python}
  #### Metodo da Barreira 
  def phi_bar(x, params, r):
      #leitura dos parametros
      f = params[0]
      c_list = params[6]
      
      b = 0    
      for c in c_list:
          b = b - 1/cl(x)
                  
      return f(x) + r*b

  def b_bar(x, params):
      #leitura dos parametros
      c_list = params[6]
      
      b = 0
      for c in c_list:
          b = b - 1/c(x)
              
      return b

  def grad_phi_bar(x, params, r):
      #leitura dos parametros
      grad_f = params[1]
      c_list = params[6]
      grad_c_list = params[7]
      
      dimens = x.size
      grad_b = np.zeros(dimens, dtype=float)
      
      for i in np.arange(len(c_list)):
          grad_b = grad_b + (c_list[i](x))**(-2)*grad_c_list[i](x)
              
      return grad_f(x) + r*grad_b

  def hess_phi_bar(x, params, r):
      #leitura dos parametros
      hess_f = params[2]
      c_list = params[6]
      grad_c_list = params[7]
      hess_c_list = params[8]
      
      dimens = x.size    
      hessian_b = np.zeros((dimens, dimens), dtype=float)
      
      for i in np.arange(dimens):    
          for j in np.arange(dimens):
              for k in np.arange(len(c_list)):
                  hessian_b[i,j] = hessian_b[i,j] - 2*((c_list[k](x))**(-3))*grad_c_list[k](x)[i]*grad_c_list[k](x)[j]
      
      for k in np.arange(len(cl_list)):
          hessian_b = hessian_b + ((c_list[k](x))**(-2))*hess_c_list[k](x)
      
      return hess_f(x) + r*hessian_b
\end{python}

\subsection{Métodos e Otimização OSR}

Os algoritmos de otimização sem restrição, Univariante, Powell, Steepest Descent, Newton-Raphson, Flecther-Reeves
e BFGS foram implementados em um arquivo denominado osr\_methods.py. Uma função chamada osr\_ctrl também está presente
nesse arquivo e faz a interface entre o código principal e os métodos de OSR.  O código principal chama essa função a cada iteração
OCR, e ela por sua vez faz todo o tratamento da OSR, chamando as funções $\phi$ do módulo OCR apresentado na seção anterior E
retornando os resultados para o código principal.

Esses códigos foram discutidos com mais detalhes no trabalho 1.

\begin{python}
  import numpy as np
  import osr_methods as osr
  import line_search_methods as lsm
  import ocr_methods as ocr
  from timeit import default_timer as timer

  def univariante(passo, dimens):
      #indice do vetor = (resto da divisao do passo pela dimensao) - 1
      #primeira posicao do vetor no python tem indice 0
      indice = passo%dimens - 1
      
      if (indice == -1) :
          #indice = -1 indica que se trata da ultima posicao do array
          #no pyton esse indice eh o tamanho do vetor - 1
          indice = dimens - 1
          
      #define a direcao canonica a ser utilizada
      ek = np.zeros(dimens)
      ek[indice] = 1
      
      return ek
      
  def powell(P, P0, direcoes, passos, ciclos, dimens):
      #indice do vetor = (resto da divisao do passo pela dimensao) - 1
      #primeira posicao do vetor no python tem indice 0
      indice = passos%(dimens + 1) - 1
      
      if (indice == -1):
          #indice = -1 indica que se trata da ultima posicao do array
          #no pyton esse indice eh o tamanho do vetor - 1
          #direcao n + 1 do ciclo = Patual - P0
          dir = P - P0
          direcoes[dimens - 1] = dir        
      elif (indice == 0):
          #indice = 0 significa que vamos usar a primeira direcao do conjunto
          #representa o inicio de um novo ciclo
          ciclos = ciclos + 1

          if (ciclos%(dimens+2) == 0):
              #se ciclo for multipl de dimens + 2, conjunto de direcoes = canonicas
              direcoes = np.eye(dimens, dtype=float)
          P0 = P.copy()
          dir = direcoes[indice].copy()
          
      else:
          dir = direcoes[indice].copy()
          direcoes[indice-1] = dir
    
      return dir, direcoes, P0, ciclos            

  def newtonRaphson(grad_P, hessian_f):
      return -np.linalg.inv(hessian_f).dot(grad_P)

  def steepestDescent(grad):
      return -grad

  def fletcherReeves(dir_last, grad, grad_last, passo):
      if passo == 1:
          grad_last = grad.copy()
          return -grad, grad_last
      else:
          beta = (np.linalg.norm(grad)/np.linalg.norm(grad_last))**2
          grad_last = grad.copy()
          return -grad + beta*dir_last, grad_last
      
  def bfgs(P, P_last, grad, grad_last, S_last, passo, dimens):
      if (passo == 1):
          dir = -S_last.dot(grad)
      else:
          delta_x_k = P - P_last
          delta_g_k = grad - grad_last
          
          #para o numpy, vetor 1-D linha e vetor coluna sao a mesma coisa (nao e necessrio transpor)
          #matrizes
          A = np.outer(delta_x_k, np.transpose(delta_x_k))
          B = S_last.dot(np.outer(delta_g_k, np.transpose(delta_x_k)))
          C = np.outer(delta_x_k, np.transpose(S_last.dot(delta_g_k)))
          
          #Escalares        
          d = np.transpose(delta_x_k).dot(delta_g_k)
          e = np.transpose(delta_g_k).dot(S_last.dot(delta_g_k))
                  
          S = S_last + (d + e)*A/(d**2) - (B + C)/d
          dir = -S.dot(grad)
          S_last = S.copy()
      P_last = P
      grad_last = grad
      return dir, P_last, grad_last, S_last

  def osr_ctrl(P0, params, r, ctrl_num, metodo_ocr, metodo_osr):
      #controle numerico
      maxiter = ctrl_num[0]
      tol_conv = ctrl_num[1]
      tol_search = ctrl_num[2]
      line_step = ctrl_num[3]
      eps = ctrl_num[4]
      
      metodo = metodo_osr
          
      #inicializacoes auxiliares dos metodos de OSR
      passos = 0
      dimens = P0.size
      Pmin = P0.copy()
      listPmin = []
      listPmin.append(Pmin)
      
      if metodo_ocr == 1:
          grad = ocr.grad_phi_penal(Pmin, params, r)
      elif metodo_ocr == 2:
          grad = ocr.grad_phi_bar(Pmin, params, r)
      
      norm_grad = np.linalg.norm(grad)
      flag_conv = True

      if (metodo == 2):
          direcoes = np.eye(dimens, dtype=float)
          ciclos = 0
          P1 = P0.copy()
      elif (metodo == 5):
          #o metodo recebe a direcao anterior 
          #inicializo a direcao com um vetor de zeros mas que nunca e usado
          #uso apenas para enviar como parametro na primeira iteracao do metodo, o qual atualiza o valor de dir para a iteracao seguinte
          dir = np.zeros((1, dimens))
          grad_last = grad.copy()
      elif(metodo == 6):
          S_last = np.eye(dimens)
          grad_last = grad.copy()
          P_last = P0.copy()
      
      #calculo do Pmin
      start = timer()
      while (norm_grad > tol_conv):
          if (passos == maxiter):
              flag_conv = False
              break
          passos = passos + 1
          if (metodo == 1):
              dir = osr.univariante(passos, dimens)
          elif (metodo == 2):
              dir, direcoes, P1, ciclos = osr.powell(Pmin, P1, direcoes,passos, ciclos, dimens)
          elif (metodo == 3):
              dir = osr.steepestDescent(grad)
          elif (metodo == 4):
              if metodo_ocr == 1:
                  hess = ocr.hess_phi_penal(Pmin, params, r)
              elif metodo_ocr == 2:
                  hess = ocr.hess_phi_bar(Pmin, params,r)
              dir = osr.newtonRaphson(grad, hess)
          elif (metodo == 5):
              dir, grad_last = osr.fletcherReeves(dir, grad, grad_last, passos)
          elif (metodo == 6):
              dir, P_last, grad_last, S_last = osr.bfgs(Pmin, P_last, grad, grad_last, S_last, passos, dimens)
          dir = dir/np.linalg.norm(dir)
          intervalo = lsm.passo_cte(dir, Pmin, params, r, metodo_ocr, eps, line_step)
          alpha = lsm.secao_aurea(intervalo, dir, Pmin, params, r, metodo_ocr, tol_search)
          Pmin = Pmin + alpha*dir
          listPmin.append(Pmin)
          
          
          if metodo_ocr == 1:
              grad = ocr.grad_phi_penal(Pmin, params, r)
          elif metodo_ocr == 2:
              grad = ocr.grad_phi_bar(Pmin, params, r)
              
          norm_grad = np.linalg.norm(grad)
          
      end = timer()
      tempoExec = end - start
      
      return listPmin, passos, norm_grad, flag_conv, tempoExec
\end{python}


\subsection{Busca Unidirecional}

Os algoritmos dos métodos do Passo Constante e da Seção Áurea foram implementados em um arquivo denominado line\_search\_methods.py.
Códigos discutidos no trabalho 1. Pequena adaptação realizada para trabalhar com as funções $\phi$ dos métodos OCR.


\begin{python}
  import ocr_methods as ocr
  import numpy as np
  
  def passo_cte(direcao, P0, params, r, metodo_ocr, eps = 1E-8, step = 0.01):
      #line search pelo metodo do passo constante
      
      #define o sentido correto de busca
      if metodo_ocr == 1:
          f1 = ocr.phi_penal(P0 - eps*(direcao/np.linalg.norm(direcao)), params, r)
          f2 = ocr.phi_penal(P0 + eps*(direcao/np.linalg.norm(direcao)), params, r)
      elif metodo_ocr == 2:
          f1 = ocr.phi_bar(P0 - eps*(direcao/np.linalg.norm(direcao)), params, r)
          f2 = ocr.phi_bar(P0 + eps*(direcao/np.linalg.norm(direcao)), params, r)
          
      if (f1 > f2):
          sentido_busca = direcao.copy()
          flag = 0
      else:
          sentido_busca = -direcao.copy()
          flag = 1
          
      P = P0.copy()
      P_next = P + step*sentido_busca
      alpha = 0
      
      if metodo_ocr == 1:
          f1 = ocr.phi_penal(P, params, r)
          f2 = ocr.phi_penal(P_next, params, r)
      elif metodo_ocr == 2:
          f1 = ocr.phi_bar(P, params, r)
          f2 = ocr.phi_bar(P_next, params, r)
        
      while (f1 > f2):           
          alpha = alpha + step
          P = P0 + alpha*sentido_busca
          P_next = P0 + (alpha+step)*sentido_busca        
          
          if metodo_ocr == 1:
              f1 = ocr.phi_penal(P, params, r)
              f2 = ocr.phi_penal(P_next, params, r)
              f_eps = ocr.phi_penal(P - eps*(sentido_busca/np.linalg.norm(sentido_busca)), params, r)
          elif metodo_ocr == 2:
              f1 = ocr.phi_bar(P, params, r)
              f2 = ocr.phi_bar(P_next, params, r)
              f_eps = ocr.phi_bar(P - eps*(sentido_busca/np.linalg.norm(sentido_busca)), params, r)
              
          if (f_eps < f1):
              alpha = alpha - step
              break
      
      intervalo = np.array([alpha, alpha + step])
      
      if(flag == 1):
          intervalo = -intervalo
          
      #retorna o intervalo de busca = [alpha min, alpha min + step]                 
      return intervalo
      
  def secao_aurea(intervalo, direcao, P0, params, r, metodo_ocr, tol=0.00001):
      #line search pelo metodo da secao aurea
      
      #verifica o sentido da busca
      if(intervalo[1] < 0):
          intervalo = -intervalo
          sentido_busca = -direcao.copy()
          flag = 1
      else:
          sentido_busca = direcao.copy()
          flag = 0
      
      #atribui os limites superior e inferior da busca a variaveis internas do metodo
      alpha_upper = intervalo[1]
      alpha_lower = intervalo[0]
      beta = alpha_upper - alpha_lower
      
      #razao aurea
      Ra = (np.sqrt(5)-1)/2
      
      # define os pontos de analise de f com base na razao aurea
      alpha_e = alpha_lower + (1-Ra)*beta
      alpha_d = alpha_lower + Ra*beta 
      
      #primeira iteracao avalia f nos 2 pontos selecionados pela razao aurea
      if metodo_ocr == 1:
          f1 = ocr.phi_penal(P0 + alpha_e*sentido_busca, params, r)
          f2 = ocr.phi_penal(P0 + alpha_d*sentido_busca, params, r)
      elif metodo_ocr == 2:
          f1 = ocr.phi_bar(P0 + alpha_e*sentido_busca, params, r)
          f2 = ocr.phi_bar(P0 + alpha_d*sentido_busca, params, r)
      
      #loop enquanto a convergencia nao for obtida
      while (beta > tol):
          if (f1 > f2):
              #caso positivo, define novo intervalo variando de alpha_e ate alpha_upper
              # e aproveita os valores anteriores de alpha_d e f2 como novos alpha_e e f1
              alpha_lower = alpha_e
              f1 = f2
              alpha_e = alpha_d
              
              #calcula novo alpha_d e f2=f(alpha_d)
              beta = alpha_upper - alpha_lower
              alpha_d = alpha_lower + Ra*beta
              
              if metodo_ocr == 1:
                  f2 = ocr.phi_penal(P0 + alpha_d*sentido_busca, params, r)
              elif metodo_ocr == 2:
                  f2 = ocr.phi_bar(P0 + alpha_d*sentido_busca, params, r) 
                  
          else:
              #caso negativo, define novo intervalo variando de alpha_lower ate alpha_d
              # e aproveita os valores anteriores de alpha_e e f1 como novos alpha_d e f2
              alpha_upper = alpha_d
              f2 = f1
              alpha_d = alpha_e
              
              #calcula novo alpha_e e f1=f(alpha_e)
              beta = alpha_upper - alpha_lower
              alpha_e = alpha_lower + (1-Ra)*beta
              
              if metodo_ocr == 1:
                  f1 = ocr.phi_penal(P0 + alpha_e*sentido_busca, params, r)
              elif metodo_ocr == 2:
                  f1 = ocr.phi_bar(P0 + alpha_e*sentido_busca, params, r)
              
      # calcula Pmin e alpha min apos convergencia
      alpha_med = (alpha_lower + alpha_upper)/2
      alpha_min = alpha_med
      
      if (flag == 1):
          alpha_min = -alpha_min
      
      return alpha_min
\end{python}


\section{Teste da Implementação}

\subsection{Problema 1}

\begin{center}
  $\begin{cases}
    \textbf{Min} \hspace{4mm}  f(x_1, x_2) = (x_1 - 2)^4 + (x_1 - 2x_2)^2 \\
    \textbf{s.t.:} \hspace{4mm} x_1^2 - x_2 \leq 0
  \end{cases}$
\end{center}
Obs.: Adotar $r_p^0 = 1$, $\beta = 10$ e $x^0 = \{3,2\}$ para o método de penalidade e
$r_b^0 = 10$, $\beta = 0.1$ e $x^0 = \{0,1\}$ para o método de barreira.

\vspace{3mm}
$\overrightarrow{\nabla} f (\overrightarrow{x}) = \{4(x_1 - 2)^3 + 2(x_1 - 2x_2), -4(x_1 - 2x_2)\}$

\vspace{3mm}
$H_{f_{1\text{x}1}} = 12(x_1 - 2)^2 + 2$, \hspace{4mm}
$H_{f_{1\text{x}2}} = -4$, \hspace{4mm}
$H_{f_{2\text{x}1}} = -4$, \hspace{4mm}
$H_{f_{2\text{x}2}} = 8$

\vspace{3mm}
\textbf{Definição da função, seu gradiente e sua hessiana, no código principal:}

\begin{python}
  def f(x):
      return (x[0]-2)**4 + (x[0] - 2*x[1])**2

  def grad_f(x):
      return np.array([4*(x[0]-2)**3 + 2*(x[0] - 2*x[1]), 2*(x[0] - 2*x[1])*(-2)])

  def hess_f(x):
      hess = np.zeros((2,2), dtype=float)
      hess[0,:] = np.array([12*(x[0]-2)**2 + 2, -4.])
      hess[1,:] = np.array([-4., 8.])
      return hess
\end{python}

\vspace{3mm}
$c(\overrightarrow{x}) = x_1^2 - x_2$

\vspace{2mm}
$\overrightarrow{\nabla} c (\overrightarrow{x}) = \{2x_1, -1\}$

\vspace{2mm}
$H_{c_{1\text{x}1}} = 2$, \hspace{4mm}
$H_{c_{1\text{x}2}} = 0$, \hspace{4mm}
$H_{c_{2\text{x}1}} = 0$, \hspace{4mm}
$H_{c_{2\text{x}2}} = 0$

\vspace{3mm}
\textbf{Definição da restrição, seu gradiente e sua hessiana, no código principal:}

\begin{python}
  def c1(x):
      return x[0]**2 - x[1]

  def grad_c1(x):
      return np.array([2*x[0], -1.])

  def hess_c1(x):
      hess = np.zeros((2,2), dtype=float)
      hess[0,:] = np.array([2., 0.])
      hess[1,:] = np.array([0., 0])
      return hess
\end{python}

\subsubsection{Penalidade - Prob. 1}

Definição do ponto inicial $x^0 = \{3,2\}$ no código principal :

\begin{python}
  x = np.array([3., 2.])
\end{python}

\vspace{3mm}
\textbf{Controle numérico e parâmetros:}

\begin{itemize}
  \item Máximas iterações na OSR : $1000$
  \item Tolerância OSR: $10^{-6}$
  \item Tolerância Seção Áurea: $10^{-7}$
  \item $\Delta \alpha$: $10^{-2}$
  \item $\epsilon$: $10^{-10}$
  \item Tolerância OCR: $10^{-6}$
  \item $r_p^0 = 1$
  \item $\beta = 10$
\end{itemize}

\begin{figure}[H]
  \centering
    \includegraphics[width=0.8\textwidth]{fig_p1/Penalidade_f.pdf}
  \caption{Curvas de nível de $f(x_1,x_2)$, restrições e otimização realizada. }
\end{figure}


\vspace{5mm}
\begin{table}[H]
  \begin{center}
    \begin{tabular}{c|c|c|c|c|c|c}
      \multicolumn{7}{c}{\textbf{Prob. 1 - Penalidade - Univariante}}\\
      \hline
      \textbf{Iter} & \textbf{$P_{min}$} & \textbf{r} & \textbf{\# Passos} & \textbf{Conv\_OCR} & \textbf{Conv\_OSR} & \textbf{t(s)}\\
      \hline
        1& [1.25174114 0.7304246 ]& 1e+00& 27& 3.5e-01& 9.6e-07& 0.028   \\
        2& [1.02501305 0.81147611]& 1e+01& 1000& 2.9e-01& 3.7e-06& 0.230\\
        3& [0.95576688 0.88122316]& 1e+02& 1000& 5.2e-02& 3.5e-06& 0.189\\
        4& [0.94659013 0.89267782]& 1e+03& 1000& 5.6e-03& 1.6e-03& 0.214\\
        5& [0.94526659 0.89319248]& 1e+04& 1000& 5.7e-04& 1.5e-02& 0.190\\
        6& [0.94513054 0.89323808]& 1e+05& 1000& 5.7e-05& 1.5e-02& 0.195\\
        7& [0.94511456 0.89323814]& 1e+06& 1000& 5.7e-06& 3.3e-02& 0.180\\
        8& [0.94511298 0.8932382 ]& 1e+07& 1000& 6.2e-07& 3.0e-01& 0.228\\
    \end{tabular}
  \end{center}
  \caption{Resultados obtidos para o problema 1, método de penalidade, univariante para $x^0=\{3,2\}$}
\end{table}

\begin{figure}[H]
  \centering
  \begin{subfigure}[b]{\textwidth}
    \includegraphics[width=0.49\textwidth]{fig_p1/Penalidade_Univariante_1.pdf}
    \includegraphics[width=0.49\textwidth]{fig_p1/Penalidade_Univariante_2.pdf}
  \end{subfigure}
  \caption{Exemplo com 2 primeiros passos da OCR com método OSR Univariante. }
\end{figure}


\vspace{5mm}
\begin{table}[H]
  \begin{center}
    \begin{tabular}{c|c|c|c|c|c|c}
      \multicolumn{7}{c}{\textbf{Prob. 1 - Penalidade - Powell}}\\
      \hline
      \textbf{Iter} & \textbf{$P_{min}$} & \textbf{r} & \textbf{\# Passos} & \textbf{Conv\_OCR} & \textbf{Conv\_OSR} & \textbf{t(s)}\\
      \hline
        1& [1.25174105 0.73042442]& 1e+00& 6& 3.5e-01& 2.2e-07& 0.027     \\
        2& [1.02501313 0.81147619]& 1e+01& 7& 2.9e-01& 9.1e-07& 0.005\\
        3& [0.95576689 0.88122318]& 1e+02& 53& 5.2e-02& 5.1e-07& 0.023\\
        4& [0.94663397 0.89276033]& 1e+03& 38& 5.6e-03& 4.8e-07& 0.013\\
        5& [0.94568845 0.89398972]& 1e+04& 30& 5.7e-04& 2.8e-07& 0.009\\
        6& [0.94559354 0.89411344]& 1e+05& 78& 5.7e-05& 3.0e-08& 0.019\\
        7& [0.94558401 0.89412573]& 1e+06& 1000& 5.8e-06& 5.2e-02& 0.264\\
        8& [0.94558315 0.89412714]& 1e+07& 1000& 5.9e-07& 1.4e-01& 0.217\\
    \end{tabular}
  \end{center}
  \caption{Resultados obtidos para o problema 1, método de penalidade, powell para $x^0=\{3,2\}$}
\end{table}

\begin{figure}[H]
  \centering
  \begin{subfigure}[b]{\textwidth}
    \includegraphics[width=0.49\textwidth]{fig_p1/Penalidade_Powell_1.pdf}
    \includegraphics[width=0.49\textwidth]{fig_p1/Penalidade_Powell_2.pdf}
  \end{subfigure}
  \caption{Exemplo com 2 primeiros passos da OCR com método OSR Powell. }
\end{figure}

\vspace{5mm}
\begin{table}[H]
  \begin{center}
    \begin{tabular}{c|c|c|c|c|c|c}
      \multicolumn{7}{c}{\textbf{Prob. 1 - Penalidade - Steepest Descent}}\\
      \hline
      \textbf{Iter} & \textbf{$P_{min}$} & \textbf{r} & \textbf{\# Passos} & \textbf{Conv\_OCR} & \textbf{Conv\_OSR} & \textbf{t(s)}\\
      \hline
        1& [1.2517411  0.73042453]& 1e+00& 25& 3.5e-01& 6.2e-07& 0.025    \\
        2& [1.02501316 0.81147628]& 1e+01& 1000& 2.9e-01& 1.6e-06& 0.254\\
        3& [0.9557669  0.88122324]& 1e+02& 1000& 5.2e-02& 1.2e-05& 0.248\\
        4& [0.94663395 0.89276024]& 1e+03& 1000& 5.6e-03& 1.1e-04& 0.210\\
        5& [0.94568838 0.89398961]& 1e+04& 1000& 5.7e-04& 1.3e-04& 0.250\\
        6& [0.94556416 0.89405783]& 1e+05& 1000& 5.7e-05& 1.1e-02& 0.203\\
        7& [0.94555192 0.89406498]& 1e+06& 1000& 6.0e-06& 1.8e-01& 0.239\\
        8& [0.94555068 0.89406568]& 1e+07& 1000& 8.9e-07& 1.8e+00& 0.212\\
    \end{tabular}
  \end{center}
  \caption{Resultados obtidos para o problema 1, método de penalidade, steepest descent para $x^0=\{3,2\}$}
\end{table}

\begin{figure}[H]
  \centering
  \begin{subfigure}[b]{\textwidth}
    \includegraphics[width=0.49\textwidth]{fig_p1/Penalidade_Steepest Descent_1.pdf}
    \includegraphics[width=0.49\textwidth]{fig_p1/Penalidade_Steepest Descent_2.pdf}
  \end{subfigure}
  \caption{Exemplo com 2 primeiros passos da OCR com método OSR Steepest Descent. }
\end{figure}

\vspace{5mm}
\begin{table}[H]
  \begin{center}
    \begin{tabular}{c|c|c|c|c|c|c}
      \multicolumn{7}{c}{\textbf{Prob. 1 - Penalidade - Newton-Raphson}}\\
      \hline
      \textbf{Iter} & \textbf{$P_{min}$} & \textbf{r} & \textbf{\# Passos} & \textbf{Conv\_OCR} & \textbf{Conv\_OSR} & \textbf{t(s)}\\
      \hline
        1& [1.25174109 0.73042445]& 1e+00& 3& 3.5e-01& 2.7e-07& 0.035     \\
        2& [1.02501318 0.81147627]& 1e+01& 4& 2.9e-01& 8.9e-08& 0.009\\
        3& [0.95576692 0.88122322]& 1e+02& 4& 5.2e-02& 1.0e-07& 0.008\\
        4& [0.94663397 0.89276032]& 1e+03& 1000& 5.6e-03& 4.6e-06& 0.668\\
        5& [0.94568841 0.89398971]& 1e+04& 1000& 5.7e-04& 1.3e-03& 0.422\\
        6& [0.94559354 0.89411344]& 1e+05& 1000& 5.7e-05& 5.1e-04& 0.502\\
        7& [0.94558405 0.89412582]& 1e+06& 1000& 5.7e-06& 2.0e-04& 0.526\\
        8& [0.9455831  0.89412707]& 1e+07& 1000& 5.7e-07& 4.2e-03& 0.401\\
    \end{tabular}
  \end{center}
  \caption{Resultados obtidos para o problema 1, método de penalidade, Newton-Raphson para $x^0=\{3,2\}$}
\end{table}

\begin{figure}[H]
  \centering
  \begin{subfigure}[b]{\textwidth}
    \includegraphics[width=0.49\textwidth]{fig_p1/Penalidade_Newton-Raphson_1.pdf}
    \includegraphics[width=0.49\textwidth]{fig_p1/Penalidade_Newton-Raphson_2.pdf}
  \end{subfigure}
  \caption{Exemplo com 2 primeiros passos da OCR com método OSR Newton-Raphson. }
\end{figure}

\vspace{5mm}
\begin{table}[H]
  \begin{center}
    \begin{tabular}{c|c|c|c|c|c|c}
      \multicolumn{7}{c}{\textbf{Prob. 1 - Penalidade - Fletcher-Reeves}}\\
      \hline
      \textbf{Iter} & \textbf{$P_{min}$} & \textbf{r} & \textbf{\# Passos} & \textbf{Conv\_OCR} & \textbf{Conv\_OSR} & \textbf{t(s)}\\
      \hline
        1& [1.24978283 0.72770155]& 1e+00& 1000& 3.5e-01& 1.9e-02& 0.338\\
        2& [1.0247375  0.81081343]& 1e+01& 1000& 2.9e-01& 5.2e-03& 0.305\\
        3& [0.95598843 0.88169493]& 1e+02& 1000& 5.2e-02& 1.0e-02& 0.253\\
        4& [0.94673281 0.89294715]& 1e+03& 1000& 5.6e-03& 2.2e-03& 0.292\\
        5& [0.94569677 0.89400545]& 1e+04& 1000& 5.7e-04& 2.9e-04& 0.259\\
        6& [0.94557152 0.89407169]& 1e+05& 1000& 5.7e-05& 2.0e-02& 0.250\\
        7& [0.94555891 0.89407834]& 1e+06& 1000& 5.5e-06& 1.4e-01& 0.235\\
        8& [0.94555764 0.89407898]& 1e+07& 1000& 3.7e-07& 1.4e+00& 0.244\\
    \end{tabular}
  \end{center}
  \caption{Resultados obtidos para o problema 1, método de penalidade, Flecther-Reeves para $x^0=\{3,2\}$}
\end{table}

\begin{figure}[H]
  \centering
  \begin{subfigure}[b]{\textwidth}
    \includegraphics[width=0.49\textwidth]{fig_p1/Penalidade_Fletcher-Reeves_1.pdf}
    \includegraphics[width=0.49\textwidth]{fig_p1/Penalidade_Fletcher-Reeves_2.pdf}
  \end{subfigure}
  \caption{Exemplo com 2 primeiros passos da OCR com método OSR Fletcher-Reeves }
\end{figure}

\vspace{5mm}
\begin{table}[H]
  \begin{center}
    \begin{tabular}{c|c|c|c|c|c|c}
      \multicolumn{7}{c}{\textbf{Prob. 1 - Penalidade - BFGS}}\\
      \hline
      \textbf{Iter} & \textbf{$P_{min}$} & \textbf{r} & \textbf{\# Passos} & \textbf{Conv\_OCR} & \textbf{Conv\_OSR} & \textbf{t(s)}\\
      \hline
        1& [1.25174106 0.73042443]& 1e+00& 6& 3.5e-01& 4.5e-08& 0.021      \\
        2& [1.02501318 0.81147628]& 1e+01& 4& 2.9e-01& 5.3e-08& 0.003\\
        3& [0.95576696 0.88122325]& 1e+02& 1000& 5.2e-02& 8.7e-06& 0.278\\
        4& [0.94663396 0.89276031]& 1e+03& 4& 5.6e-03& 6.6e-07& 0.001\\
        5& [0.94568843 0.8939897 ]& 1e+04& 1000& 5.7e-04& 6.2e-06& 0.244\\
        6& [0.94559353 0.89411343]& 1e+05& 1000& 5.7e-05& 3.1e-03& 0.259\\
        7& [0.94558406 0.89412585]& 1e+06& 1000& 5.7e-06& 1.0e-03& 0.276\\
        8& [0.94558308 0.89412702]& 1e+07& 1000& 5.7e-07& 7.1e-03& 0.271\\
    \end{tabular}
  \end{center}
  \caption{Resultados obtidos para o problema 1, método de penalidade, BFGS para $x^0=\{3,2\}$}
\end{table}

\begin{figure}[H]
  \centering
  \begin{subfigure}[b]{\textwidth}
    \includegraphics[width=0.49\textwidth]{fig_p1/Penalidade_BFGS_1.pdf}
    \includegraphics[width=0.49\textwidth]{fig_p1/Penalidade_BFGS_2.pdf}
  \end{subfigure}
  \caption{Exemplo com 2 primeiros passos da OCR com método OSR BFGS. }
\end{figure}

\subsubsection{Barreira - Prob. 1}

Definição do ponto inicial $x^0 = \{0,1\}$ no código principal :

\begin{python}
  x = np.array([0., 1.])
\end{python}

\vspace{3mm}
\textbf{Controle numérico e parâmetros:}

\begin{itemize}
  \item Máximas iterações na OSR : $1000$
  \item Tolerância OSR: $10^{-6}$
  \item Tolerância Seção Áurea: $10^{-7}$
  \item $\Delta \alpha$: $10^{-2}$
  \item $\epsilon$: $10^{-10}$
  \item Tolerância OCR: $10^{-6}$
  \item $r_b^0 = 10$
  \item $\beta = 0.1$
\end{itemize}

\begin{figure}[H]
  \centering
    \includegraphics[width=0.8\textwidth]{fig_p1/Barreira_f.pdf}
  \caption{Curvas de nível de $f(x_1,x_2)$, restrições e otimização realizada. }
\end{figure}


\vspace{5mm}
\begin{table}[H]
  \begin{center}
    \begin{tabular}{c|c|c|c|c|c|c|c}
      \multicolumn{8}{c}{\textbf{Prob. 1 - Barreira - Univariante}}\\
      \hline
      \textbf{Iter} & \textbf{$P_{min}$} & \textbf{r} & $\Delta \alpha$ &\textbf{\# Passos} & \textbf{Conv\_OCR} & \textbf{Conv\_OSR} & \textbf{t(s)}\\
      \hline
        1& [0.70794442 1.53149922]& 1e+01& 1e-02 &208& 9.7e+00& 9.3e-07& 0.051    \\
        2& [0.82820088 1.10979838]& 1e+00& 1e-02 &76& 2.4e+00& 4.2e-07& 0.020\\
        3& [0.89886443 0.96384102]& 1e-01& 1e-02 &144& 6.4e-01& 8.4e-07& 0.040\\
        4& [0.92935334 0.91630707]& 1e-02& 1e-02 &1000& 1.9e-01& 5.4e-06& 0.357\\
        5& [0.94027885 0.90115413]& 1e-03& 1e-02 &1000& 5.9e-02& 1.2e-05& 0.181\\
        6& [0.94393558 0.89644099]& 1e-04& 1e-03 &1000& 1.8e-02& 1.7e-03& 0.137\\
        7& [0.94528564 0.89528469]& 1e-05& 1e-03 &1000& 5.8e-03& 8.6e-03& 0.135\\
        8& [0.94574387 0.89497567]& 1e-06& 1e-04 &1000& 1.8e-03& 1.2e-02& 0.115\\
        9& [0.94589637 0.89489206]& 1e-07& 1e-04 &1000& 5.8e-04& 1.3e-02& 0.102\\
        10& [0.94593827 0.89485365]& 1e-08& 1e-05 &1000& 1.8e-04& 1.5e-02& 0.098\\
        11& [0.94595766 0.89485312]& 1e-09& 1e-05 &1000& 5.8e-05& 8.6e-03& 0.086\\
        12& [0.94596395 0.89485321]& 1e-10& 1e-06 &1000& 1.8e-05& 7.2e-02& 0.065\\
        13& [0.94596577 0.89485297]& 1e-11& 1e-06 &1000& 5.7e-06& 1.6e-01& 0.082\\
        14& [0.94596639 0.89485294]& 1e-12& 1e-07 &1000& 1.9e-06& 3.3e-01& 0.063\\
        15& [0.94596659 0.89485294]& 1e-13& 1e-07 &1000& 6.5e-07& 1.8e+00& 0.064\\
    \end{tabular}
  \end{center}
  \caption{Resultados obtidos para o problema 1, método de barreira, univariante para $x^0=\{0,1\}$}
\end{table}

\begin{figure}[H]
  \centering
  \begin{subfigure}[b]{\textwidth}
    \includegraphics[width=0.49\textwidth]{fig_p1/Barreira_Univariante_1.pdf}
    \includegraphics[width=0.49\textwidth]{fig_p1/Barreira_Univariante_2.pdf}
  \end{subfigure}
  \caption{Exemplo com 2 primeiros passos da OCR com método OSR Univariante. }
\end{figure}


\vspace{5mm}
\begin{table}[H]
  \begin{center}
    \begin{tabular}{c|c|c|c|c|c|c|c}
      \multicolumn{8}{c}{\textbf{Prob. 1 - Barreira - Powell}}\\
      \hline
      \textbf{Iter} & \textbf{$P_{min}$} & \textbf{r} & $\Delta \alpha$ &\textbf{\# Passos} & \textbf{Conv\_OCR} & \textbf{Conv\_OSR} & \textbf{t(s)}\\
      \hline
        1& [0.70794439 1.53149919]& 1e+01& 1e-02 &8& 9.7e+00& 5.6e-07& 0.010       \\
        2& [0.82820086 1.10979833]& 1e+00& 1e-02 &25& 2.4e+00& 8.4e-07& 0.009\\
        3& [0.89886443 0.963841  ]& 1e-01& 1e-02 &20& 6.4e-01& 2.8e-07& 0.006\\
        4& [0.92935322 0.91630682]& 1e-02& 1e-02 &9& 1.9e-01& 9.3e-07& 0.002\\
        5& [0.94027827 0.90115302]& 1e-03& 1e-02 &12& 5.9e-02& 8.0e-07& 0.003\\
        6& [0.9438872 0.8963501]& 1e-04& 1e-03 &44& 1.8e-02& 4.0e-07& 0.013\\
        7& [0.9450449  0.89483025]& 1e-05& 1e-03 &1000& 5.8e-03& 5.1e-04& 0.367\\
        8& [0.94541263 0.89434951]& 1e-06& 1e-04 &54& 1.8e-03& 5.1e-07& 0.007\\
        9& [0.94552911 0.89419752]& 1e-07& 1e-04 &24& 5.8e-04& 4.1e-07& 0.005\\
        10& [0.94556594 0.89414941]& 1e-08& 1e-05 &393& 1.8e-04& 4.0e-07& 0.065\\
        11& [0.94557753 0.89413409]& 1e-09& 1e-05 &1000& 5.8e-05& 4.5e-03& 0.154\\
        12& [0.94558129 0.89412942]& 1e-10& 1e-06 &30& 1.8e-05& 4.7e-07& 0.002\\
        13& [0.94558246 0.89412794]& 1e-11& 1e-06 &1000& 5.7e-06& 1.9e-01& 0.140\\
        14& [0.94558283 0.89412746]& 1e-12& 1e-07 &1000& 1.8e-06& 4.0e-01& 0.111\\
        15& [0.94558288 0.89412715]& 1e-13& 1e-07 &1000& 5.7e-07& 3.3e-01& 0.119\\
    \end{tabular}
  \end{center}
  \caption{Resultados obtidos para o problema 1, método de barreira, Powell para $x^0=\{0,1\}$}
\end{table}

\begin{figure}[H]
  \centering
  \begin{subfigure}[b]{\textwidth}
    \includegraphics[width=0.49\textwidth]{fig_p1/Barreira_Powell_1.pdf}
    \includegraphics[width=0.49\textwidth]{fig_p1/Barreira_Powell_2.pdf}
  \end{subfigure}
  \caption{Exemplo com 2 primeiros passos da OCR com método OSR Powell. }
\end{figure}

\vspace{5mm}
\begin{table}[H]
  \begin{center}
    \begin{tabular}{c|c|c|c|c|c|c|c}
      \multicolumn{8}{c}{\textbf{Prob. 1 - Barreira - Steepest Descent}}\\
      \hline
      \textbf{Iter} & \textbf{$P_{min}$} & \textbf{r} & $\Delta \alpha$ &\textbf{\# Passos} & \textbf{Conv\_OCR} & \textbf{Conv\_OSR} & \textbf{t(s)}\\
      \hline
        1& [0.70794438 1.53149917]& 1e+01& 1e-02 &52& 9.7e+00& 4.3e-07& 0.033       \\
        2& [0.8282009  1.10979841]& 1e+00& 1e-02 &112& 2.4e+00& 3.5e-07& 0.035\\
        3& [0.89886442 0.96384104]& 1e-01& 1e-02 &1000& 6.4e-01& 6.9e-06& 0.219\\
        4& [0.92935325 0.91630683]& 1e-02& 1e-02 &1000& 1.9e-01& 1.5e-05& 0.325\\
        5& [0.94027832 0.90115304]& 1e-03& 1e-02 &1000& 5.9e-02& 6.6e-05& 0.300\\
        6& [0.94388751 0.89635066]& 1e-04& 1e-03 &1000& 1.8e-02& 9.5e-05& 0.172\\
        7& [0.94504565 0.8948317 ]& 1e-05& 1e-03 &1000& 5.8e-03& 3.0e-04& 0.145\\
        8& [0.94549366 0.89450285]& 1e-06& 1e-04 &1000& 1.8e-03& 2.3e-03& 0.160\\
        9& [0.94563767 0.89440286]& 1e-07& 1e-04 &1000& 5.8e-04& 3.3e-03& 0.193\\
        10& [0.94568309 0.89437093]& 1e-08& 1e-05 &1000& 1.8e-04& 1.2e-02& 0.374\\
        11& [0.94569739 0.89436074]& 1e-09& 1e-05 &1000& 5.8e-05& 3.6e-02& 0.307\\
        12& [0.94570181 0.89435729]& 1e-10& 1e-06 &1000& 1.9e-05& 1.7e-01& 0.150\\
        13& [0.94570312 0.89435619]& 1e-11& 1e-06 &1000& 5.6e-06& 6.0e-01& 0.098\\
        14& [0.94570357 0.89435581]& 1e-12& 1e-07 &1000& 1.8e-06& 3.1e-01& 0.062\\
        15& [0.94570374 0.89435569]& 1e-13& 1e-07 &1000& 7.8e-07& 5.6e+00& 0.063\\
    \end{tabular}
  \end{center}
  \caption{Resultados obtidos para o problema 1, método de barreira, Steepest Descent para $x^0=\{0,1\}$}
\end{table}

\begin{figure}[H]
  \centering
  \begin{subfigure}[b]{\textwidth}
    \includegraphics[width=0.49\textwidth]{fig_p1/Barreira_Steepest Descent_1.pdf}
    \includegraphics[width=0.49\textwidth]{fig_p1/Barreira_Steepest Descent_2.pdf}
  \end{subfigure}
  \caption{Exemplo com 2 primeiros passos da OCR com método OSR Steepest Descent. }
\end{figure}

\vspace{5mm}
\begin{table}[H]
  \begin{center}
    \begin{tabular}{c|c|c|c|c|c|c|c}
      \multicolumn{8}{c}{\textbf{Prob. 1 - Barreira - Newton-Raphson}}\\
      \hline
      \textbf{Iter} & \textbf{$P_{min}$} & \textbf{r} & $\Delta \alpha$ &\textbf{\# Passos} & \textbf{Conv\_OCR} & \textbf{Conv\_OSR} & \textbf{t(s)}\\
      \hline
        1& [0.7079444 1.5314992]& 1e+01& 1e-02 &4& 9.7e+00& 1.2e-07& 0.014         \\
        2& [0.8282009  1.10979841]& 1e+00& 1e-02 &6& 2.4e+00& 1.0e-07& 0.011\\
        3& [0.89886444 0.96384105]& 1e-01& 1e-02 &1000& 6.4e-01& 3.0e-06& 0.395\\
        4& [0.9293532  0.91630686]& 1e-02& 1e-02 &1000& 1.9e-01& 2.3e-05& 0.344\\
        5& [0.94027832 0.90115304]& 1e-03& 1e-02 &1000& 5.9e-02& 6.9e-05& 0.389\\
        6& [0.94388719 0.89635012]& 1e-04& 1e-03 &1000& 1.8e-02& 1.4e-04& 0.295\\
        7& [0.9450449  0.89483025]& 1e-05& 1e-03 &1000& 5.8e-03& 4.7e-04& 0.327\\
        8& [0.94541264 0.89434953]& 1e-06& 1e-03 &23& 1.8e-03& 5.0e-07& 0.004\\
        9& [0.9455291 0.8941975]& 1e-07& 1e-04 &1000& 5.8e-04& 5.7e-05& 0.319\\
        10& [0.94556593 0.89414939]& 1e-08& 1e-04 &1000& 1.8e-04& 1.8e-04& 0.330\\
        11& [0.94557761 0.89413424]& 1e-09& 1e-05 &1000& 5.8e-05& 2.1e-05& 0.241\\
        12& [0.94558128 0.8941294 ]& 1e-10& 1e-05 &1000& 1.8e-05& 3.8e-04& 0.276\\
        13& [0.94558246 0.89412791]& 1e-11& 1e-06 &1000& 5.8e-06& 1.8e-03& 0.192\\
        14& [0.94558281 0.8941274 ]& 1e-12& 1e-06 &1000& 1.8e-06& 1.1e-02& 0.173\\
        15& [0.94558294 0.89412727]& 1e-13& 1e-07 &1000& 5.8e-07& 1.5e-03& 0.149\\
    \end{tabular}
  \end{center}
  \caption{Resultados obtidos para o problema 1, método de barreira, Newton-Raphson para $x^0=\{0,1\}$}
\end{table}

\begin{figure}[H]
  \centering
  \begin{subfigure}[b]{\textwidth}
    \includegraphics[width=0.49\textwidth]{fig_p1/Barreira_Newton-Raphson_1.pdf}
    \includegraphics[width=0.49\textwidth]{fig_p1/Barreira_Newton-Raphson_2.pdf}
  \end{subfigure}
  \caption{Exemplo com 2 primeiros passos da OCR com método OSR Newton-Raphson. }
\end{figure}

\vspace{5mm}
\begin{table}[H]
  \begin{center}
    \begin{tabular}{c|c|c|c|c|c|c|c}
      \multicolumn{8}{c}{\textbf{Prob. 1 - Barreira - Fletcher-Reeves}}\\
      \hline
      \textbf{Iter} & \textbf{$P_{min}$} & \textbf{r} & $\Delta \alpha$ &\textbf{\# Passos} & \textbf{Conv\_OCR} & \textbf{Conv\_OSR} & \textbf{t(s)}\\
      \hline
        1& [0.7083677  1.53228399]& 1e+01& 1e-02 &1000& 9.7e+00& 1.2e-02& 0.272   \\
        2& [0.82771558 1.10882661]& 1e+00& 1e-02 &1000& 2.4e+00& 1.1e-02& 0.255\\
        3& [0.89923489 0.96456733]& 1e-01& 1e-02 &1000& 6.4e-01& 7.5e-03& 0.164\\
        4& [0.92959411 0.9167657 ]& 1e-02& 1e-02 &1000& 1.9e-01& 4.3e-03& 0.174\\
        5& [0.94042115 0.90142408]& 1e-03& 1e-02 &1000& 5.9e-02& 2.6e-03& 0.192\\
        6& [0.94385431 0.89628806]& 1e-04& 1e-03 &1000& 1.8e-02& 6.6e-04& 0.157\\
        7& [0.94502008 0.89478343]& 1e-05& 1e-03 &1000& 5.8e-03& 6.4e-04& 0.199\\
        8& [0.94541561 0.8943552 ]& 1e-06& 1e-04 &1000& 1.8e-03& 1.5e-03& 0.132\\
        9& [0.94556948 0.89427382]& 1e-07& 1e-04 &1000& 5.8e-04& 4.4e-03& 0.171\\
        10& [0.94561808 0.89424809]& 1e-08& 1e-05 &1000& 1.8e-04& 1.9e-02& 0.123\\
        11& [0.94563354 0.89423997]& 1e-09& 1e-05 &1000& 5.8e-05& 4.8e-02& 0.137\\
        12& [0.94563835 0.89423743]& 1e-10& 1e-06 &1000& 1.8e-05& 2.3e-01& 0.073\\
        13& [0.94563992 0.89423665]& 1e-11& 1e-06 &1000& 5.6e-06& 5.8e-01& 0.127\\
        14& [0.94564043 0.89423638]& 1e-12& 1e-07 &1000& 1.8e-06& 2.2e-01& 0.085\\
        15& [0.94564061 0.89423629]& 1e-13& 1e-07 &1000& 8.0e-07& 6.4e+00& 0.101\\
    \end{tabular}
  \end{center}
  \caption{Resultados obtidos para o problema 1, método de barreira, Fletcher-Reeves para $x^0=\{0,1\}$}
\end{table}

\begin{figure}[H]
  \centering
  \begin{subfigure}[b]{\textwidth}
    \includegraphics[width=0.49\textwidth]{fig_p1/Barreira_Fletcher-Reeves_1.pdf}
    \includegraphics[width=0.49\textwidth]{fig_p1/Barreira_Fletcher-Reeves_2.pdf}
  \end{subfigure}
  \caption{Exemplo com 2 primeiros passos da OCR com método OSR Fletcher-Reeves }
\end{figure}

\vspace{5mm}
\begin{table}[H]
  \begin{center}
    \begin{tabular}{c|c|c|c|c|c|c|c}
      \multicolumn{8}{c}{\textbf{Prob. 1 - Barreira - BFGS}}\\
      \hline
      \textbf{Iter} & \textbf{$P_{min}$} & \textbf{r} & $\Delta \alpha$ &\textbf{\# Passos} & \textbf{Conv\_OCR} & \textbf{Conv\_OSR} & \textbf{t(s)}\\
      \hline
        1& [0.7079444 1.5314992]& 1e+01& 1e-02 &11& 9.7e+00& 1.4e-07& 0.017           \\
        2& [0.82820084 1.10979841]& 1e+00& 1e-02 &1000& 2.4e+00& 7.2e-06& 0.368\\
        3& [0.89886439 0.96384104]& 1e-01& 1e-02 &1000& 6.4e-01& 1.3e-05& 0.346\\
        4& [0.92935323 0.91630685]& 1e-02& 1e-02 &4& 1.9e-01& 7.5e-07& 0.003\\
        5& [0.94027829 0.90115306]& 1e-03& 1e-02 &4& 5.9e-02& 3.8e-07& 0.001\\
        6& [0.94388719 0.89635009]& 1e-04& 1e-02 &1000& 1.8e-02& 7.6e-06& 0.301\\
        7& [0.94504488 0.89483027]& 1e-05& 1e-03 &1000& 5.8e-03& 1.4e-05& 0.358\\
        8& [0.94541264 0.89434954]& 1e-06& 1e-04 &1000& 1.8e-03& 3.6e-04& 0.152\\
        9& [0.94552912 0.89419753]& 1e-07& 1e-04 &1000& 5.8e-04& 3.1e-04& 0.290\\
        10& [0.94556597 0.89414946]& 1e-08& 1e-04 &1000& 1.8e-04& 1.5e-03& 0.195\\
        11& [0.94557758 0.89413418]& 1e-09& 1e-05 &1000& 5.8e-05& 3.1e-04& 0.194\\
        12& [0.94558126 0.89412937]& 1e-10& 1e-05 &1000& 1.8e-05& 3.6e-04& 0.211\\
        13& [0.94558245 0.89412789]& 1e-11& 1e-06 &1000& 5.8e-06& 3.7e-04& 0.206\\
        14& [0.94558281 0.8941274 ]& 1e-12& 1e-06 &1000& 1.8e-06& 6.4e-04& 0.190\\
        15& [0.94558295 0.89412729]& 1e-13& 1e-07 &1000& 5.8e-07& 2.3e-03& 0.151\\
    \end{tabular}
  \end{center}
  \caption{Resultados obtidos para o problema 1, método de barreira, BFGS para $x^0=\{0,1\}$}
\end{table}

\begin{figure}[H]
  \centering
  \begin{subfigure}[b]{\textwidth}
    \includegraphics[width=0.49\textwidth]{fig_p1/Barreira_BFGS_1.pdf}
    \includegraphics[width=0.49\textwidth]{fig_p1/Barreira_BFGS_2.pdf}
  \end{subfigure}
  \caption{Exemplo com 2 primeiros passos da OCR com método OSR BFGS. }
\end{figure}

\end{document}